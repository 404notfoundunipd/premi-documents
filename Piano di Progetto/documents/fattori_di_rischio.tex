
\section{Analisi dei rischi}
Questo paragrafo ha il compito di analizzare in modo approfondito gli eventuali rischi che possono emergere durante il periodo di lavoro, in modo da ottimizzare l'avanzamento del progetto. Per l'analisi dei rischi è stata definita la seguente procedura:

\begin{itemize}
\item \textbf{Identificazione:} individuare i potenziali rischi che possono presentarsi durante l'avanzamento del processo;
\item \textbf{Analisi:} valutare la possibilità dell'occorrenza del rischio, valutarne le conseguenze sul progetto;
\item \textbf{Pianificazione di controllo:} istituire metodi di controllo e prevenzione per i rischi individuati, così da poterli evitare;
\item \textbf{Mitigazione:} definire delle contromisure di correzioni per mitigare le conseguenze deleterie di un rischio nel caso dovesse verificarsi. 
\end{itemize}

Di seguito, l'elenco dei possibili rischi identificati. Per ognuno di essi è stata indicata la probabilità di occorrenza e il grado di pericolosità, delineata una breve descrizione, definite delle strategie per il rilevamento e prevenzione e stabilite delle contromisure di correzioni: 

\subsection{Livello tecnologico}

\hypertarget{subsubsect:software}{}
\subsubsection{Strumenti software}

\begin{itemize}
\item \textbf{Probabilità di occorrenza:} Medio
\item \textbf{Grado di pericolosità:} Alto
\item \textbf{Descrizione:} il software utilizzato per la gestione del progetto è stato scelto e accettato preventivamente da tutti i componenti del gruppo di lavoro. Potrebbero però sorgere dei problemi di incompatibilità dovuti all'utilizzo di versioni differenti del software scelto. Inoltre alcune tecnologie adottate per lo sviluppo del progetto non sono conosciute da tutti i membri del gruppo, e non è comunque da escludere la possibilità di incontrare degli inconvenienti anche nell'utilizzo degli strumenti conosciuti;

\item \textbf{Strategie per il rilevamento e prevenzione:} durante la fase di analisi si è deciso di utilizzare una macchina virtuale$_G$ (VM) creata e configurata appositamente per la realizzazione del progetto. Una volta creata, la VM$_G$ è stata distribuita ai membri del gruppo in modo tale da permettere a tutti di lavorare con il medesimo ambiente di sviluppo ed evitare così problemi legati all'incompatibilità fra versioni differenti del software. \\
Il \ruoloResponsabile{} ha il compito di verificare il grado di conoscenza di ciascun componente, relativo alle tecnologie adottate;
 
\item \textbf{Contromisure di correzione:} nel caso in cui uno dei componenti si trovi impossibilitato ad usare la VM$_G$, sarà compito suo risolvere il problema riadattando il lavoro svolto alla versione corretta del software.
Per quanto riguarda il grado di conoscenza delle tecnologie, ciascun componente si impegnerà a colmare le proprie lacune in modo autonomo, attraverso lo studio della documentazione online fornita dagli enti che sviluppano le tecnologie adottate;

\item \textbf{Riscontro effettivo:} il gruppo non ha ancora riscontrato problemi al riguardo.
\end{itemize}


\subsubsection{Strumenti hardware}
\hypertarget{subsubsect:hardware}{}
\begin{itemize}
\item \textbf{Probabilità di occorrenza:} Molto basso
\item \textbf{Grado di pericolosità:} Basso

\item \textbf{Descrizione:} l'intero progetto, dalla realizzazione dei documenti al codice, sarà sviluppato su calcolatore, di conseguenza ogni membro del gruppo possiede l'hardware necessario allo sviluppo del progetto. Tuttavia i portatili in uso dai componenti del gruppo sono di tipo commerciale e non professionale, quindi è da tenere in considerazione la possibilità che si verifichino problemi che portino all'inutilizzo totale o parziale di alcuni calcolatori;

\item \textbf{Strategie per il rilevamento e prevenzione:} ciascun componente del gruppo avrà cura della propria attrezzatura. Oltre all'hardware personale, il gruppo ha accesso a tre laboratori che garantiscono l'hardware necessario allo sviluppo del progetto. Tuttavia, nel caso di guasti improvvisi delle macchine c'è il rischio di perdita o inaccessibilità ai dati. Per questo motivo tutto il materiale è presente in duplice copia, ovvero in locale e su una piattaforma di versionamento. Inoltre il lavoro di ogni componente è suddiviso in varie attività di minore entità; nel momento in cui si conclude un'attività, questa andrà caricata sulla piattaforma di versionamento e si proseguirà con l'attività successiva;
 
\item \textbf{Contromisure di correzione:} in caso di impossibilità da parte in un componente ad accedere al proprio calcolatore, egli potrà usufruire dei laboratori di facoltà e, grazie alla piattaforma di versionamento, potrà riprendere a lavorare in breve tempo ammortizzando la perdita di dati. Inoltre, la decisione di lavorare su VM$_G$ e il salvataggio della stessa in cloud$_G$, permette di avere sempre a disposizione l'ambiente di sviluppo ideale per la realizzazione del progetto, evitando così problemi e ritardi dovuti alla differente configurazione software dei calcolatori utilizzati;

\item \textbf{Riscontro effettivo:} alcuni componenti del gruppo non hanno la possibilità di lavorare sempre con l'hardware personale. Grazie alla VM e alla piattaforma di versionamento ogni componente del gruppo ha a disposizione tutto l'occorrente per lavorare indipendentemente dall'hardware che utilizza.
\end{itemize}

\subsection{Livello organizzativo}

\subsubsection{Valutazione dei costi}
\hypertarget{subsubsect:costi}{}
\begin{itemize}
\item \textbf{Probabilità di occorrenza:} Medio
\item \textbf{Grado di pericolosità:} Alto
\item \textbf{Descrizione:} durante la pianificazione è possibile che le tempistiche per l'esecuzione di alcune attività vengano calcolate in modo errato, soprattutto a causa dell'inesperienza del gruppo in ambito di sviluppo, progettazione, organizzazione e gestione degli imprevisti. In particolare, una sottostima dei tempi provoca un ritardo nella consegna dei materiali previsti con conseguente aumento dei costi;

\item \textbf{Strategie per il rilevamento e prevenzione:} la caratteristica dinamica del rischio impone che si debba controllare lo stato dei ticket periodicamente, in modo da verificare eventuali ritardi nello sviluppo delle attività. Sarà compito del \ruoloResponsabile{} verificare giornalmente lo stato delle attività in corso, ponendo particolare attenzione alle attività critiche in modo da evitare che subiscano ritardi;


\item \textbf{Contromisure di correzione:} si è deciso di prevedere per ogni attività con maggior criticità un periodo di slack$_G$, per evitare che un ritardo possa influenzare la durata totale del progetto. Se il periodo di slack$_G$ dovesse risultare insufficiente si rieseguirà una pianificazione che terrà conto del ritardo accumulato e che si cercherà di recuperare. Nel caso non sia possibile recuperare il ritardo accumulato si procederà analizzando caso per caso la soluzione migliore. Generalmente si dovrà scegliere se proporre un preventivo economico maggiorato rispetto a quello calcolato o se limitare le funzionalità desiderabili e opzionali;

\item \textbf{Riscontro effettivo:} nei primi giorni si sono verificati dei ritardi nel portare a termine alcune attività. Tuttavia, grazie a una pianificazione attenta, non si sono verificati aumenti di costi né slittamenti nei tempi di consegna.

\end{itemize}


\subsection{Livello dei requisiti}

\subsubsection{Cambio radicale dei requisiti}
\hypertarget{subsubsect:requisiti}{}
\begin{itemize}
\item \textbf{Probabilità di occorrenza:} Basso
\item \textbf{Grado di pericolosità:} Alto

\item \textbf{Descrizione:} nonostante il Proponente abbia fornito una prima descrizione del prodotto, egli potrebbe decidere di applicare delle variazioni ai requisiti del progetto;

\item \textbf{Strategie per il rilevamento e prevenzione:} sarà opportuno mantenere una adeguata comunicazione con il Proponente in modo da venire a conoscenza il prima possibile di eventuali cambi radicali dei requisiti ed evitare così incomprensioni;

\item \textbf{Contromisure di correzione:} in caso di richieste di variazioni da parte del Proponente si discuterà con lo stesso la possibilità di applicare le suddette modifiche. Si dovranno rivalutare tempi di sviluppo, preventivo e la possibilità di soddisfare i requisiti desiderabili e opzionali, tenendo in considerazione la possibilità di una rivalutazione profonda del lavoro svolto;

\item \textbf{Riscontro effettivo:} il gruppo non ha ancora riscontrato problemi al riguardo.
\end{itemize}

\subsection{Livello del personale}

\subsubsection{Indisposizione di uno o più membri}
\hypertarget{subsubsect:indisposizione}{}
\begin{itemize}
\item \textbf{Probabilità di occorrenza:} Medio
\item \textbf{Grado di pericolosità:} Medio

\item \textbf{Descrizione:} un componente del gruppo potrebbe trovarsi indisposto a lavorare per un certo periodo di tempo. Considerando il numero di persone che lavorano al progetto  e il periodo di tempo previsto per la sua realizzazione, è piuttosto probabile che almeno un componente sarà indisposto a lavorare. Tuttavia, l'eventualità che si presentino lunghi periodi di assenza o impossibilità totale di lavoro da parte di un componente è piuttosto remota;
\item \textbf{Strategie per il rilevamento e prevenzione:} in caso di indisposizione da parte di un membro del gruppo è opportuno avvisare tempestivamente il \ruoloResponsabile{}  così da permettere una rapida riorganizzazione del lavoro tra gli altri componenti del gruppo;

\item \textbf{Contromisure di correzione:} riorganizzazione il più possibile tempestiva del lavoro a breve termine, ed eventualmente disponibilità del componente assente di lavorare anche fuori sede (quando possibile). In caso di periodi di assenza particolarmente lunghi, il \ruoloResponsabile{} provvederà ad effettuare una riorganizzazione dei ruoli all'interno del gruppo;

\item \textbf{Riscontro effettivo:} il gruppo non ha ancora riscontrato problemi al riguardo.
\end{itemize}

\subsubsection{Conflitti interni al gruppo}
\hypertarget{subsubsect:conflitti}{}
\begin{itemize}
\item \textbf{Probabilità di occorrenza:} Medio
\item \textbf{Grado di pericolosità:} Medio alto

\item \textbf{Descrizione:} per tutti i componenti del team questa è la prima esperienza di lavoro collaborativo in un gruppo così numeroso e con un progetto di queste dimensioni. Questo potrebbe causare problemi di collaborazione con conseguente appesantimento del carico di lavoro e la nascita di un clima lavorativo non proficuo. La probabilità che si verifichino conflitti fra i membri aumenta sensibilmente in caso di elevata mole di lavoro e stress;

\item \textbf{Strategie per il rilevamento e prevenzione:} la collaborazione dei componenti del gruppo nelle varie fasi permetterà al \ruoloResponsabile{} di monitorare lo stato dei rapporti fra i membri del team. Sarà responsabilità di ogni singolo componente cercare di mantenere un comportamento il più possibile adeguato al corretto svolgimento del progetto. In caso di problemi inespressi, ogni componente dovrà cercare di evidenziare errori e inefficienze degli altri compagni in maniera costruttiva, in modo da correggere eventuali comportamenti errati o controproducenti;

\item \textbf{Contromisure di correzione:} nel caso di forti conflitti, il \ruoloResponsabile{} dovrà mediare l'incontro tra i componenti problematici. Se i contrasti tra le parti non venissero risolti o si dovessero verificare frequenti conflitti fra i due componenti, il \ruoloResponsabile{} avrà il compito di riorganizzare il lavoro in modo da minimizzare l'interazione fra i due;

\item \textbf{Riscontro effettivo:} il gruppo non ha ancora riscontrato problemi al riguardo.
\end{itemize}

\subsubsection{Inesperienza del gruppo}
\hypertarget{subsubsect:inesperienza}{}
\begin{itemize}
\item \textbf{Probabilità di occorrenza:} Alta
\item \textbf{Grado di pericolosità:} Alto

\item \textbf{Descrizione:} l'approccio al metodo di lavoro risulta nuovo. Progetti di questa entità richiedono capacità di pianificazione e di analisi che il gruppo non possiede a causa dell'inesperienza e l'utilizzo di alcuni strumenti software che nessun componente del gruppo ha mai utilizzato. Alcune conoscenze richieste richiedono tempo per essere apprese;

\item \textbf{Strategie per il rilevamento e prevenzione:} qualora si presentasse la necessità di utilizzare un nuovo strumento, verrà segnalato al \ruoloResponsabile. Egli affiderà la gestione del nuovo strumento alla persona ritenuta più adeguata a padroneggiarlo nel più breve tempo possibile, la quale, in caso di difficoltà nel reperire la documentazione necessaria, richiederà consigli agli altri membri del gruppo, mediante i metodi di comunicazione descritti nelle Norme di Progetto; 

\item \textbf{Contromisure di correzione:} ogni componente del gruppo si impegna a studiare il materiale richiesto per poter affrontare in modo ottimale il progetto. Nel caso la persona designata ad una determinata attività non riuscisse ad apprendere in tempi brevi le conoscenze necessarie a portarla a termine, essa verrà sostituita da un altro componente. Se l'apprendimento di un determinato strumento dovesse risultare impossibile in un tempo ridotto si provvederà alla ricerca di un nuovo strumento. In caso di ritardi sensibili dovuti all'errata gestione delle mansioni, si dovrà intervenire rivalutando la suddivisione
del lavoro e analizzando le cause che hanno portato alla dispersione delle risorse;

\item \textbf{Riscontro effettivo:} attualmente il gruppo non ha dovuto utilizzare strumenti sconosciuti, di conseguenza il rischio non si è ancora presentato.
\end{itemize}


\subsubsection{Impossibilità di incontrarsi periodicamente in uno stesso luogo}
\hypertarget{subsubsect:incontri}{}
\begin{itemize}
\item \textbf{Probabilità di occorrenza:} Medio
\item \textbf{Grado di pericolosità:} Basso
\item \textbf{Descrizione:} i membri del gruppo vivono in luoghi relativamente distanti fra loro. Inoltre, le sedi universitarie potrebbero essere irraggiungibili a causa della chiusura dell'ateneo o a causa di disservizi dei mezzi di trasporto.

\item \textbf{Strategie per il rilevamento e prevenzione:}  il lavoro sarà pianificato tenendo conto anche dei giorni di chiusura delle sedi universitarie, mentre la suddivisione dei compiti tra i componenti dovrà, per quanto possibile, garantire una sufficiente capacità di sviluppo individuale indipendente dal lavoro degli altri componenti;

\item \textbf{Contromisure di correzione:} nel caso di lunghi periodi di chiusura dell'ateneo e necessità di lavorare in gruppo si dovrà trovare un luogo di ritrovo alternativo e raggiungibile da tutti tramite mezzi pubblici o privati. Nei casi in cui questo non sia possibile, la piattaforma di versionamento del materiale e egli strumenti di comunicazione riportati nel documento Norme di Progetto, permetteranno, nel limite del possibile, di ovviare al problema mantenendo una comunicazione affidabile ed efficiente.

\item \textbf{Riscontro effettivo:} nel gruppo vi sono due studenti lavoratori che a causa di impegni inerenti al lavoro stesso non sempre possono essere presenti agli incontri del gruppo. Tuttavia, grazie ad un'attenta suddivisione dei compiti e agli strumenti di comunicazione che permettono di mantenere i componenti del gruppo costantemente in contatto, non si sono verificati problemi o ritardi.
\end{itemize}

\subsection{Riepilogo}
{\renewcommand\arraystretch{1.2} 
\begin{table}[h]
\begin{tabular}{|l|l|l|}
\hline
\textbf{Rischio} & \textbf{Probabilità} & \textbf{Pericolosità} \\
\hline
 \hyperlink{subsubsect:software}{Strumenti software} & Medio & Alto \\
\hline
\hyperlink{subsubsect:hardware}{Strumenti hardware} & Molto basso & Basso\\
\hline
\hyperlink{subsubsect:costi}{Valutazione dei costi} & Medio & Alto \\
\hline
\hyperlink{subsubsect:requisiti}{Cambio dei requisiti} & Basso & Alto \\
\hline
\hyperlink{subsubsect:indisposizione}{Indisposizione di uno o più membri} & Medio & Medio \\
\hline
\hyperlink{subsubsect:conflitti}{Conflitti interni al gruppo} & Medio & Medio Alto \\
\hline
\hyperlink{subsubsect:inesperienza}{Inesperienza del gruppo} & Alta & Alto \\
\hline
\hyperlink{subsubsect:incontri}{Impossibilità ad incontrarsi in uno stesso luogo} & Medio & Basso \\
\hline
\end{tabular}
\caption{Riepilogo dei rischi con la probabilità di occorrenza e il grado di pericolosità.}
\end{table}}


\newpage