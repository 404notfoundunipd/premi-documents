\section{Ciclo di vita}
Il modello di ciclo di vita adottato per Premi e' il \textbf{modello incrementale}, per le sue proprieta' qui elencate:

\begin{itemize}
	\item decompone il prodotto finale in un numero di componenti, ognuno dei quali viene costruito (e verificato) separatamente per ordine di importanza. Questo permette di verificare con piu' precisione che tutti requisiti vengano soddisfatti perche' sono facilmente associabili ad uno o piu' componenti. E' importante ricordare che la creazione di un sito web, suddiviso in pagine, gerarchie e funzionalita' si presta naturalmente ad una costruzione di tipo incrementale.
	\item Sposta le attivita' principali di sviluppo, analisi e progettazione architetturale ad alto livello all'inizio del ciclo garantendo una codifica allo stesso tempo controllata e snella. Anche gli incrementi vengono pianificati e questo aiuta a stimare costi e tempi di produzione.
	\item Favorisce la creazione di prototipi che consentono una maggiore visione di insieme e migliorano il dialogo con il committente.
	\item Ogni incremento riduce il rischio di fallimento perche' consolida la sezione coinvolta (ogni incremento produce una base da considerarsi stabile).
	\item Le risorse umane possono essere distribuite a rotazione su di un numero limitato di attivita', per brevi periodi di tempo, assumendo ruoli diversi. Questo e' in linea con la richiesta dei docenti di far ricoprire piu' ruoli ai componenti del gruppo, garantendo pero' assenza di conflitto di interessi tra i ruoli assunti.
\end{itemize}	
	
|
|
|
|
|            immagine modello incrementale
|
|
|
|
|


\section{Scadenze e Ruoli previsti}
Qui vengono presentate le date di consegna delle revisioni che il gruppo 404NotFound ha deciso di rispettare per la stesura di questo documento:

\begin{itemize}
	\item \textit{Revisione dei Requisiti} (RR): 16-02-2015 \\
	Data di consegna della documentazione: 23-01-2015
	\item \textit{Revisione di Progetto} (RP): 27-04-2015 \\
	Data di consegna della documentazione: 22-04-2015
	\item \textit{Revisione di Qualifica} (RQ): 29-05-2015 \\
	Data di consegna della documentazione: 24-05-2015
	\item \textit{Revisione di Accettazione} (RA): 18-06-2015 \\
	Data di consegna della documentazione: 17-06-2015
\end{itemize}

I ruoli previsti per la realizzazione del progetto sono:

\begin{table}[h]
\begin{center}
\begin{tabular}{|l|l|}
\hline
\textbf{Ruolo} & \textbf{Costo (€/ora)} \\
\hline
Responsabile & 30 \\
Analista & 25 \\
Progettista & 22 \\
Amministratore & 20 \\
Verificatore & 15 \\
Programmatore & 15 \\
\hline
\end{tabular}
\caption{Ruoli previsti e costo per ruolo.}
\end{table}


\section{Pianificazione delle attivita'}
Lo sviluppo di Premi, in linea con le scadenze sopra elencate, viene diviso in quattro macro-fasi:

\begin{itemize}
\item \textbf{Analisi} (AN) dal 1-12-2014 al 22-01-2015
\item \textbf{Progettazione Architetturale} (PA) dal 17-02-2015 al 21-04-2015
\item \textbf{Progettazione di Dettaglio e Codifica} (PDC) dal 22-04-2015 al 23-05-2015
\item \textbf{Verifica Finale e Validazione} (VV) dal 24-05-2015 al 16-06-2015
\end{itemize}

Ogni macro-fase e' a sua volta divisa nelle sue attivita' essenziali, e ogni attivita' e' composta da sotto-attivita' ne che disciplinano la realizzazione.

|
|
|
|  descrizione dei Gantt
|
|
|


\subsection{Analisi}
\textbf{Periodo:} dal 1-12-2014 al 22-01-2015
La macro-fase di Analisi inizia dalla formazione del gruppo e prosegue fino alla consegna della documentazione per la Revisione dei Requisiti. \\
I ruoli coinvolti sono quelli del \textit{Responsabile di Progetto}, \textit{Amministratore} e \textit{Analista}, mentre le attivita' principali sono la stesura e la verifica dei seguenti documenti:

\begin{itemize}
\item \textbf{Studio di fattibilita':} in questo documento vengono studiate le tecnologie interessate e la fattibilita' dei Capitolati per stabilire quale affrontare. Sulla scelta pesa molto anche l'interesse dei vari membri del gruppo ai temi proposti.
\item \textbf{Norme di Progetto:} emanate dall'\textit{Amministratore}, queste norme disciplinano tutte le attivita' del gruppo in ogni fase del Ciclo di Vita del software.
\item \textbf{Analisi dei Requisiti:} qui vengono descritti in modo dettagliato i requisiti emersi dal Capitolato e dal successivo incontro con il proponente. Ogni requisito aiuta a delineare le funzionalita' del prodotto finale.
\item \textbf{Piano di Progetto:} la stesura di questo documento e' compito del \textit{Responsabile di progetto} e punta a pianificare le attivita' e a distribuirle nell'arco di tempo stabilito dal gruppo calcolandone i costi totali. Vengono inoltre studiati i possibili rischi a cui il progetto va incontro e vengono suggerite le strategie per affrontarli.
\item \textbf{Piano di Qualifica:} delinea la strategia generale di Verifica e Validazione.
\item \textbf{Glossario:} questo documento contiene le definizioni dei termini e degli acronimi presenti negli altri documenti e ne facilita la compresione. Viene scritto in modo incrementale da tutti i redattori.
\item \textbf{Lettera di Presentazione:} ha il compito di presentare il gruppo al committente e rende ufficiale l'offerta di prendersi in carico il capitolato.
\end{itemize}

\subsubsection{diagrammi}


\subsection{Progettazione Architetturale}
\textbf{Periodo:} dal 17-02-2015 al 21-04-2015 \\
La macro-fase di Progettazione Architetturale inizia dalla Revisione dei Requisiti e prosegue fino alla consegna della documentazione per la Revisione di Progetto. \\
I ruoli coinvolti sono quelli del \textit{Responsabile di Progetto}, \textit{Amministratore}, \textit{Analista}, \textit{Progettista} e \textit{Verificatore}, mentre le attivita' principali sono la stesura e la verifica dei seguenti documenti:

\begin{itemize}
\item \textbf{correzione},se necessaria, dei documenti usciti dalla precedente fase (la Revisione dei Requisiti potrebbe imporre delle modifiche ad alcune sezioni).
\item \textbf{Specifica Tecnica:} ha lo scopo di definire l'architettura del prodotto finale,  attraverso lo studio dei componenti e l'esposizione dei Design Pattern utilizzati. Ogni componente viene associato ad uno o piu' requisiti.
\item \textbf{incremento e verifica} dei documenti usciti dalla precendente macro-fase.
\end{itemize}

\subsubsection{diagrammi}


\subsection{Progettazione di Dettaglio e Codifica}
\textbf{Periodo:} dal 22-04-2015 al 23-05-2015 \\
La macro-fase di Progettazione Architetturale inizia dalla Revisione di Progetto e prosegue fino alla consegna del prodotto per la Revisione di Qualifica. \\
L'attivita' di codifica viene divisa in due parti: la prima produce un prototipo contenente le funzionalita' di base, la seconda incrementa il prototipo e crea un prodotto completo. \\
I ruoli coinvolti sono quelli del \textit{Responsabile di Progetto}, \textit{Amministratore}, \textit{Progettista}, \textit{Verificatore} e \textit{Programmatore}.
Anche in questa fase e' prevista la stesura e la verifica di documenti:
\begin{itemize}
\item \textbf{correzione}, se necessaria, dei documenti usciti dalla precedente fase (la Revisione di Progetto potrebbe imporre delle modifiche ad alcune sezioni).
\item \textbf{Definizione di Prodotto:} questo documento mostra come si e' scelto di attuare le scelte progettuali definite nella Specifica Tecnica. Classi e componenti vengono descritti in modo dettagliato.
\item \textbf{Manuale Utente:} guida gli utenti all'utilizzo il prodotto.
\item \textbf{incremento e verifica} dei documenti usciti dalla precendente macro-fase.

\subsubsection{diagrammi}


\subsection{Verifica Finale e Validazione}
\textbf{Periodo:} dal 24-05-2015 al 16-06-2015 
La macro-fase di Verifica Finale e Validazione inizia dalla Revisione di Qualifica e conclude le attivita' di sviluppo del software. \\
In queste tre settimane verra' corretta, incrementata e verificata tutta la documentazione prodotta fino a quel momento con l'obiettivo di rilasciarne una versione da definirsi completa. \\
L'attivita' di \textbf{validazione e collaudo} si occupa di accertare che il prodotto realizzato sia conforme alle attese e che copra i requisiti previsti. \\

\subsubsection{diagrammi}


