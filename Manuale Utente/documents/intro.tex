\section{Introduzione}
\subsection{Scopo del documento}
Lo scopo del presente documento è quello di fornire all'utente finale una guida che spiegherà passo passo il funzionamento del prodotto Premi, illustrandone tutte le funzionalità.

\subsection{Utente finale del prodotto}
Il prodotto è rivolto a tutti gli utenti che utilizzano un browser$_G$ visuale e non è stata posta particolare attenzione alle categorie di utenti che fanno uso dello screenreader$_G$. In particolare il software verrà utilizzato da chiunque abbia la necessità di creare una presentazione, e cerchi uno strumento alternativo al noto e diffuso PowerPoint$_G$.

\subsection{Scopo del prodotto}
Lo scopo del progetto è la realizzazione di un software di presentazione di slide non basato sul modello di PowerPoint$_{G}$, sviluppato in tecnologia HTML5$_{G}$ e che funzioni sia su desktop che su dispositivo mobile. Il software dovrà permettere la creazione da parte dell'autore e la successiva presentazione del lavoro, fornendo effetti grafici di supporto allo storytelling e alla creazione di mappe mentali.

\subsection{Come leggere il manuale}
Il presente manuale è stato redatto allo scopo di guidare l'utente passo passo nell'utilizzo dell'applicazione \emph{Premi}, pertanto la sua struttura è lineare e suddivisa per argomenti. Ogni azione che l'utente vuole effettuare è illustrata in questa guida. L'utilizzo di molte immagini illustrative rende più intuitiva la spiegazione.

\subsection{Come segnalare problemi e malfunzionamenti}
In caso venissero riscontrati malfunzionamenti o comportamenti anomali dell'applicazione \emph{Premi}, comunicarli tramite messaggio email all'indirizzo \href{mailto:404notfound.unipd@gmail.com}{404notfound.unipd@gmail.com} con la seguente forma:
\begin{itemize}
\item \textbf{OGGETTO}: segnalazione\_bug
\item \textbf{TESTO}: descrizione dettagliata dell'anomalia riscontrata, riportando eventuali codici di errore visualizzati
\item \textbf{ALLEGATO}: se disponibile, screenshot dimostrante l'anomalia riscontrata. Saranno considerati solo file immagine con estensione .png$_G$ o .jpg/.jpeg$_G$.
\end{itemize}

\subsection{Glossario}
Al fine di evitare ogni ambiguità relativa al linguaggio e ai termini utilizzati nei documenti formali tutti i termini e gli acronimi presenti nel seguente documento che necessitano di definizione saranno seguiti da una ”G” in pedice e saranno riportati nella sezione apposita dell'appendice del presente documento. Tale sezione consiste in un listato ordinato di termini e acronimi con le rispettive definizioni e spiegazioni.

\subsection{Riferimenti}
\subsubsection{Informativi}
\begin{itemize}
	\item \textbf{W3C:} \href{http://www.w3c.it/it/1/ufficio-italiano-w3c.html}{http://www.w3c.it/it/1/ufficio-italiano-w3c.html};
	\item \textbf{Wikipedia:} \href{http://it.wikipedia.org/}{http://it.wikipedia.org/}; 
	\item \textbf{Wordreference:} \href{http://www.wordreference.com/}{http://www.wordreference.com/}.
\end{itemize}

\newpage