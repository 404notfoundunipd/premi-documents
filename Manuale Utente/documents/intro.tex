\section{Introduzione}
\subsection{Scopo del documento}
Lo scopo del presente documento è quello di fornire all'utente finale una guida che spiegherà passo passo il funzionamento del prodotto Premi, illustrandone tutte le funzionalità.

%************ DA COMPLETARE ****************
\subsection{Utente finale del prodotto}
Definire quali tipi di utenti sono previsti per l'utilizzo dell'applicazione Premi.

\subsection{Scopo del prodotto}
Lo scopo del progetto è la realizzazione di un software di presentazione di slide non basato sul modello di PowerPoint$_{G}$, sviluppato in tecnologia HTML5$_{G}$ e che funzioni sia su desktop che su dispositivo mobile. Il software dovrà permettere la creazione da parte dell'autore e la successiva presentazione del lavoro, fornendo effetti grafici di supporto allo storytelling e alla creazione di mappe mentali.

%************ DA COMPLETARE ****************
\subsection{Come leggere il manuale}
Descrivere com'è stato strutturato il manuale in modo da agevolarne la lettura da parte dell'utente.

%************ DA COMPLETARE ****************
\subsection{Come segnalare problemi e malfunzionamenti}
Fornire in questa sezione tutti gli strumenti e le linee guida per permettere all'utente di riportare eventuali bug e malfunzionamenti non previsti da parte dell'applicazione.

\subsection{Glossario}
Al fine di evitare ogni ambiguità relativa al linguaggio e ai termini utilizzati nei documenti formali tutti i termini e gli acronimi presenti nel seguente documento che necessitano di definizione saranno seguiti da una ”G” in pedice e saranno riportati nella sezione apposita dell'appendice del presente documento. Tale sezione consiste in un listato ordinato di termini e acronimi con le rispettive definizioni e spiegazioni.

\subsection{Riferimenti}
\subsubsection{Informativi}

\newpage