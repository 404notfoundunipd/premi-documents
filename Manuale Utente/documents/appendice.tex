\subsection{Appendice}

\subsubsection{Glossario}
\hfill\Huge{\textbf{B}}\\
\rule{16cm}{.6pt}
\normalsize
	\begin{longtable}{p{0.25\textwidth} p{0.75\textwidth}} 
	    \\
		    \textbf{Browser}: &	Programma che fornisce uno strumento per navigare in Internet e interagire con il World Wide Web.\\
	\end{longtable}
	
\hfill\Huge{\textbf{C}}\\
\rule{16cm}{.6pt}
\normalsize
	\begin{longtable}{p{0.25\textwidth} p{0.75\textwidth}} 
	    \\
		    \textbf{Checkpoint}: &	Il Checkpoint corrisponde ad un punto della presentazione in cui c'è la possibilità di entrare nel cammino di specializzazione o di continuare sequenzialmente con il frame successivo.\\
	\end{longtable}
	
	\hfill\Huge{\textbf{E}}\\
\rule{16cm}{.6pt}
\normalsize
	\begin{longtable}{p{0.25\textwidth} p{0.75\textwidth}} 
	    \\
		    \textbf{Editor}: & Editor in informatica è un programma per la visualizzazione e la modifica di testo o immagini.\\
	\end{longtable}
	
	\hfill\Huge{\textbf{F}}\\
\rule{16cm}{.6pt}
\normalsize
	\begin{longtable}{p{0.25\textwidth} p{0.75\textwidth}} 
	    \\
		    \textbf{Frame}: & Nel contesto della presentazione, il frame è un oggetto contenitore in grado di contenere oggetti grafici, come immagini, testo e shape. Inoltre ogni frame può rappresentare un nodo del cammino presentativo.
	\end{longtable}
	
	\hfill\Huge{\textbf{H}}\\
\rule{16cm}{.6pt}
\normalsize
	\begin{longtable}{p{0.25\textwidth} p{0.75\textwidth}} 
	    \\
		    \textbf{HTML5}: & HTML5 è un linguaggio di markup per la strutturazione delle pagine web, e da Ottobre 2014 pubblicato come W3C Recommendation.
	\end{longtable}
	
	\hfill\Huge{\textbf{I}}\\
\rule{16cm}{.6pt}
\normalsize
	\begin{longtable}{p{0.25\textwidth} p{0.75\textwidth}} 
	    \\
		    \textbf{Infografica}: & L'infografica è l'informazione proiettata in forma più grafica e visuale che testuale. Le immagini utilizzate, elaborate tramite computer su palette grafiche elettroniche, possono essere 2D o 3D, animate o fisse.
	\end{longtable}
	
	
\hfill\Huge{\textbf{J}}\\
\rule{16cm}{.6pt}
\normalsize
	\begin{longtable}{p{0.25\textwidth} p{0.75\textwidth}} 
	    \\
		    \textbf{Javascript}: & E` un linguaggio di programmazione interpretato sviluppato da Netscape orientato agli oggetti e agli eventi.\\
		    \\
		    \textbf{JPEG - JPG}: & JPEG (acronimo di Joint Photographic Experts Group) è un comitato ISO/CCITT che ha definito il primo standard internazionale di compressione dell'immagine digitale a tono continuo, sia a livelli di grigio che a colori. JPEG indica quindi anche il diffusissimo formato di compressione a perdita di informazioni ed è un formato aperto e ad implementazione gratuita.
	\end{longtable}
	
	\hfill\Huge{\textbf{P}}\\
\rule{16cm}{.6pt}
\normalsize
	\begin{longtable}{p{0.25\textwidth} p{0.75\textwidth}} 
	    \\
		    \textbf{PNG}: & In informatica, il Portable Network Graphics (abbreviato PNG) è un formato di file per memorizzare immagini.\\
		    \\
		    \textbf{Power Point}: & Microsoft Office PowerPoint è il programma di presentazione prodotto da Microsoft, fa parte della suite di software di produttività personale Microsoft Office, è tutelato da copyright e distribuito con licenza commerciale ed è disponibile per i sistemi operativi Windows e Macintosh. E` utilizzato principalmente per proiettare e quindi comunicare su schermo, progetti, idee, e contenuti potendo incorporare testo, immagini, grafici, filmati, audio e potendo presentare tutto questo con animazioni di alto livello.\\
	\end{longtable}
	
	\hfill\Huge{\textbf{S}}\\
\rule{16cm}{.6pt}
\normalsize
	\begin{longtable}{p{0.25\textwidth} p{0.75\textwidth}} 
	    \\
		    \textbf{Screen reader}: & Uno screen reader (letteralmente lettore dello schermo) è un'applicazione software che identifica ed interpreta il testo mostrato sullo schermo di un computer, presentandolo tramite sintesi vocale o attraverso un display braille.\\
	\end{longtable}
	
	
	
	
	
	
	
	
	
	
	
	
	
	
	

