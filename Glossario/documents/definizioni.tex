\section{Definizioni}
\begin{longtable}{lp{0.5\columnwidth}}
	
	\textbf{Facebook} & Facebook è un servizio di rete sociale$_G$, posseduto e gestito dalla corporation Facebook Inc.
	
	\\ \\
	
	\textbf{Freeware} & Il termine freeware indica un software che viene distribuito in modo gratuito.
	
	\\ \\
	
	\textbf{Gmail} & Gmail (Google Mail) è un servizio gratuito di posta elettronica via web (Webmail), POP3 e IMAP, fornito da Google.
	
	\\ \\
	
	\textbf{HTML5} & L'HTML5 è un linguaggio di markup per la strutturazione delle pagine web, e da Ottobre 2014 pubblicato come W3C Recommendation.
	
	\\ \\
		
	\textbf{Instant Messaging} & La messaggistica istantanea (in lingua inglese instant messaging) è una categoria di sistemi di comunicazione sincrona in tempo reale in rete, tipicamente Internet o una rete locale, che permette ai suoi utilizzatori lo scambio di brevi messaggi.
	
	\\ \\ 
	
	\textbf{JPEG - JPG} & JPEG (acronimo di Joint Photographic Experts Group) è un comitato ISO/CCITT che ha definito il primo standard internazionale di compressione dell'immagine digitale a tono continuo, sia a livelli di grigio che a colori.
	"JPEG" indica quindi anche il diffusissimo formato di compressione a perdita di informazioni ed è un formato aperto e ad implementazione gratuita.
	
	\\ \\
	
	\textbf{Mailing List} & La mailing list (letteralmente, lista di corrispondenza, dalla lingua inglese; traducibile in italiano sia con lista di distribuzione o di diffusione, sia con lista di discussione, a seconda degli usi) è un servizio/strumento offribile da una rete di computer verso vari utenti e costituito da un sistema organizzato per la partecipazione di più persone ad una discussione asincrona o per la distribuzione di informazioni utili agli interessati/iscritti attraverso l'invio di email ad una lista di indirizzi di posta elettronica di utenti iscritti.
	
	\\ \\
	
	\textbf{PDF} & Il Portable Document Format, comunemente abbreviato PDF, è un formato di file basato su un linguaggio di descrizione di pagina sviluppato da Adobe Systems nel 1993 per rappresentare documenti in modo indipendente dall'hardware e dal software utilizzati per generarli o per visualizzarli.
	
	\\ \\ 
	
	\textbf{PNG} & In informatica, il Portable Network Graphics (abbreviato PNG) è un formato di file per memorizzare immagini.
	
	\\ \\
	
	\textbf{Power Point} & Microsoft Office PowerPoint è il programma di presentazione prodotto da Microsoft, fa parte della suite di software di produttività personale Microsoft Office, è tutelato da copyright e distribuito con licenza commerciale ed è disponibile per i sistemi operativi Windows e Macintosh. È utilizzato principalmente per proiettare e quindi comunicare su schermo, progetti, idee, e contenuti potendo incorporare testo, immagini, grafici, filmati, audio e potendo presentare tutto questo con animazioni di alto livello.
	
	\\ \\
	
	\textbf{Rete Sociale} & Una rete sociale (in lingua inglese social network) consiste in un qualsiasi gruppo di individui connessi tra loro da diversi legami sociali. Per gli esseri umani i legami vanno dalla conoscenza casuale, ai rapporti di lavoro, ai vincoli familiari.
	
	\\ \\
	
	\textbf{Skype} & Skype è un software proprietario freeware$_G$ di messaggistica istantanea$_G$ e VoIP. Esso unisce caratteristiche presenti nei client più comuni (chat, salvataggio delle conversazioni, trasferimento di file) ad un sistema di telefonate basato su un network Peer-to-peer. 
	
	\\ \\
	
	\textbf{SVG} & Scalable Vector Graphics abbreviato in SVG, indica una tecnologia in grado di visualizzare oggetti di grafica vettoriale e, pertanto, di gestire immagini scalabili dimensionalmente.
	Più specificamente si tratta di un linguaggio derivato dall'XML, cioè di un'applicazione del metalinguaggio posto a base degli sviluppi del Web da parte del consorzio W3C, che si pone l'obiettivo di descrivere figure bidimensionali statiche e animate.
	
	\\ \\ 
	
	\textbf{WhatsApp} & WhatsApp è un'applicazione proprietaria di messaggistica istantanea$_G$ multi-piattaforma per smartphone. Oltre allo scambio di messaggi testuali è possibile inviare immagini, video, file audio e condividere la propria posizione (grazie all'uso di mappe integrate nel dispositivo) con chiunque abbia uno smartphone dotato di connessione a Internet e abbia installato l'applicazione.
		
\end{longtable}