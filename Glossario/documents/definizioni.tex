\section{Definizioni}
\begin{longtable}{p{5cm} p{0.6\columnwidth}}

	\textbf{\Huge{A}} & 
	
	\\ \\
	
	\textbf{\Huge{B}} & 

	\\ \\

	\textbf{Branch} & Branch rappresenta una linea indipendente di sviluppo di un software su cui poter lavorare 
senza entrare in conflitto con la versione principale o con il lavoro degli altri sviluppatori.

	\\ \\
	
	\textbf{Browser} & Programma che fornisce uno strumento per navigare in Internet e interagire con il World Wide Web.

	\\ \\	
	
	\textbf{\Huge{C}} & 
	
	\\ \\

	\textbf{Cloud} & Il Cloud o Cloud Storage è un modello di conservazione dati su computer in rete dove i dati stessi sono memorizzati su molteplici server virtuali generalmente ospitati presso strutture di terze parti o su server dedicati.	
	
	\\ \\
	
	\textbf{Commit} & Un commit (o, più raramente, install, submit, check-in o ci) si effettua quando si copiano le modifiche fatte su file locali nella directory del repository (il software di controllo versione controlla quali file sono stati modificati dall'ultima sincronizzazione).

	\\ \\

	\textbf{Controllo di versione \linebreak distribuito} & è un sistema che permette a più sviluppatori lavorare ad un progetto senza dover necessariamente essere connessi ad una rete. Anziché lavorare in un'unica repository comune, ogni sviluppatore
lavora su di una propria copia del progetto che può essere sincronizzata con l'originale in un secondo momento.

	\\ \\
	
	\textbf{CSS} & acronimo di Cascading Style Sheets, in italiano fogli di stile è un linguaggio di programmazione che definisce lo stile di formattazione delle pagine web.

	\\ \\
	
	\textbf{\Huge{D}} & 
	
	\\ \\
	
	\textbf{Drag and Drop} & Nell'interfaccia grafica di un computer, il drag-and-drop indica una successione di tre azioni, consistenti nel cliccare su un oggetto virtuale (quale una finestra o un'icona) per trascinarlo (in inglese: drag) in un'altra posizione, dove viene rilasciato (in inglese: drop).
	
	\\ \\
	
	\textbf{\Huge{E}} & 
	
	\\ \\

	\textbf{Editor} & Editor in informatica è un programma per la visualizzazione e la modifica di testo o immagini.

	\\ \\
	
	\textbf{\Huge{F}} & 
	
	\\ \\
	
	\textbf{Facebook} & Facebook è un servizio di rete sociale, posseduto e gestito dalla corporation Facebook Inc.
	
	\\ \\
	
	\textbf{Freeware} & Il termine freeware indica un software che viene distribuito in modo gratuito.
	
	\\ \\
	
	\textbf{\Huge{G}} & 
	
	\\ \\
	
	\textbf{Git} & Git è un sistema software di controllo di versione distribuito, creato da Linus Torvalds nel 2005 e 
disponibile per tutti i principali sistemi operativi.

	\\ \\
	
	\textbf{GitHub} & GitHub è un servizio di hosting di repository che usa il sistema di controllo di versione Git.
Per maggiori informazioni visitare il sito \url{http://www.github.com/}.

	\\ \\
	
	\textbf{Gmail} & Gmail (Google Mail) è un servizio gratuito di posta elettronica via web (Webmail), POP3 e IMAP, fornito da Google.
	
	\\ \\

	\textbf{Google Calendar} & Google Calendar è un sistema gratuito di calendari integrato nell'account Google.
Funziona come un'agenda nella quale inserire degli eventi, e il suo utilizzo può essere pubblico o privato.
	
	\\ \\

	\textbf{Google Drive} & Google Drive è un servizio web integrato nell'account Google per l'archiviazione e la condivisione di file e documenti.
Include, tra le altre funzionalità,  l'applicazione web Google Docs per visualizzare e modificare documenti direttamente 
dal browser.	
	
	\\ \\
	
	\textbf{\Huge{H}} & 
	
	\\ \\
	
	\textbf{HTML5} & L'HTML5 è un linguaggio di markup per la strutturazione delle pagine web, e da Ottobre 2014 pubblicato come W3C Recommendation.
	
	\\ \\
	
	\textbf{\Huge{I}} & 
	
	\\ \\
	
	\textbf{Infografica} & L'infografica è l'informazione proiettata in forma più grafica e visuale che testuale. Le immagini utilizzate, elaborate tramite computer su palette grafiche elettroniche, possono essere 2D o 3D, animate o fisse.
	
	\\ \\
		
	\textbf{Instant Messaging} & La messaggistica istantanea (in lingua inglese instant messaging) è una categoria di sistemi di comunicazione sincrona in tempo reale in rete, tipicamente Internet o una rete locale, che permette ai suoi utilizzatori lo scambio di brevi messaggi.
	
	\\ \\ 

	\textbf{ISO} & acronimo di Internatonal Organization for Standardization. È la più importante organizzazione internazionale per la definizione di norme tecniche. 

	\\ \\

	\textbf{IEC} & IEC acronimo di International Electronic Commission. È un organizzazione internazionale per la definizione di standard in materia di elettronica elettricità e tecnologie correllate. Alcuni dei suoi standard sono definiti assieme a ISO.
	
	\\ \\

	\textbf{ISO/IEC 14598} & ISO/IEC 14598 descrive il processo di valutazione del software.  
	
	\\ \\
	
	\textbf{\Huge{J}} & 
	
	\\ \\
	
	\textbf{JavaScript} & È un linguaggio di programmazione interpretato sviluppato da Netscape orientato agli oggetti e agli eventi.
	
	\\ \\	
	
	\textbf{JPEG - JPG} & JPEG (acronimo di Joint Photographic Experts Group) è un comitato ISO/CCITT che ha definito il primo standard internazionale di compressione dell'immagine digitale a tono continuo, sia a livelli di grigio che a colori.
	"JPEG" indica quindi anche il diffusissimo formato di compressione a perdita di informazioni ed è un formato aperto e ad implementazione gratuita.
	
	\\ \\
	
	\textbf{\Huge{K}} & 
	
	\\ \\
	
	\textbf{\Huge{L}} & 
	
	\\ \\
	
	\textbf{\Huge{M}} & 
	
	\\ \\
	
	\textbf{Macchina Virtuale} & In informatica il termine macchina virtuale (VM dall'inglese Virtual Machine) indica un software che, attraverso un processo di virtualizzazione, crea un ambiente virtuale che emula tipicamente il comportamento di una macchina fisica grazie all'assegnazione di risorse hardware (porzioni di disco rigido, RAM e risorse di processamento) ed in cui alcune applicazioni possono essere eseguite come se interagissero con tale macchina.
	
	\\ \\
	
	\textbf{Mailing List} & La mailing list (letteralmente, lista di corrispondenza, dalla lingua inglese; traducibile in italiano sia con lista di distribuzione o di diffusione, sia con lista di discussione, a seconda degli usi) è un servizio/strumento offribile da una rete di computer verso vari utenti e costituito da un sistema organizzato per la partecipazione di più persone ad una discussione asincrona o per la distribuzione di informazioni utili agli interessati/iscritti attraverso l'invio di email ad una lista di indirizzi di posta elettronica di utenti iscritti.
	
	\\ \\
	
	\textbf{Mega} & È un servizio di storage online che mette a disposizione uno spazio gratuito di 50GB. 
	
	\\ \\	
	
	\textbf{Merge} & All'interno di una repository, significa riunire un branch di copia con il branch originale, 
aggiornando solamente quei files che sono stati modificati nel branch di copia indipendentemente
dai cambiamenti che sono avvenuti nel branch originale durante il periodo di separazione dei due.

	\\ \\
	
	\textbf{Milestone (pietra miliare)} & è un termine usato per indicare delle tappe intermedie nello sviluppo di un progetto, che consentono di monitorare con maggiore efficacia il suo progresso.
	
	\\ \\
	
	\textbf{\Huge{N}} & 
	
	\\ \\
	
	\textbf{\Huge{O}} & 
	
	\\ \\
	
	\textbf{\Huge{P}} &
	
	\\ \\ 
	
	\textbf{PDF} & Il Portable Document Format, comunemente abbreviato PDF, è un formato di file basato su un linguaggio di descrizione di pagina sviluppato da Adobe Systems nel 1993 per rappresentare documenti in modo indipendente dall'hardware e dal software utilizzati per generarli o per visualizzarli.
	
	\\ \\ 
	
	\textbf{PNG} & In informatica, il Portable Network Graphics (abbreviato PNG) è un formato di file per memorizzare immagini.
	
	\\ \\
	
	\textbf{Power Point} & Microsoft Office PowerPoint è il programma di presentazione prodotto da Microsoft, fa parte della suite di software di produttività personale Microsoft Office, è tutelato da copyright e distribuito con licenza commerciale ed è disponibile per i sistemi operativi Windows e Macintosh. È utilizzato principalmente per proiettare e quindi comunicare su schermo, progetti, idee, e contenuti potendo incorporare testo, immagini, grafici, filmati, audio e potendo presentare tutto questo con animazioni di alto livello.
	
	\\ \\
	
	\textbf{\Huge{Q}} & 
	
	\\ \\
	
	\textbf{\Huge{R}} & 
	
	\\ \\
	
	\textbf{Repository} & Repository è uno spazio di archiviazione da cui è possibile recuperare software o codice sorgente;
	
	\\ \\
	
	\textbf{Rete Sociale} & Una rete sociale (in lingua inglese social network) consiste in un qualsiasi gruppo di individui connessi tra loro da diversi legami sociali. Per gli esseri umani i legami vanno dalla conoscenza casuale, ai rapporti di lavoro, ai vincoli familiari.
	
	\\ \\
	
	\textbf{\Huge{S}} & 
	
	\\ \\
	
	\textbf{Skype} & Skype è un software proprietario freeware di messaggistica istantanea e VoIP. Esso unisce caratteristiche presenti nei client più comuni (chat, salvataggio delle conversazioni, trasferimento di file) ad un sistema di telefonate basato su un network Peer-to-peer. 
	
	\\ \\
	
	\textbf{SVG} & Scalable Vector Graphics abbreviato in SVG, indica una tecnologia in grado di visualizzare oggetti di grafica vettoriale e, pertanto, di gestire immagini scalabili dimensionalmente.
	Più specificamente si tratta di un linguaggio derivato dall'XML, cioè di un'applicazione del metalinguaggio posto a base degli sviluppi del Web da parte del consorzio W3C, che si pone l'obiettivo di descrivere figure bidimensionali statiche e animate.
	
	\\ \\ 
	
	\textbf{\Huge{T}} & 
	
	\\ \\
	
	\textbf{Template} & Il termine inglese template in informatica indica un documento o programma nel quale, come in un foglio semicompilato cartaceo, su una struttura generica o standard esistono spazi temporaneamente "bianchi" da riempire successivamente.
	
	\\ \\
	
	\textbf{TexMaker} & è un strumento software utilizzato per produrre pdf mediante linguaggio \LaTeX{}.
	
	\\ \\
	
	\textbf{Ticket o trouble ticket} & Un trouble ticket indica letteralmente una richiesta di assistenza, tracciata da un sistema informatico di gestione delle richieste di assistenza, che per metonimia viene indicato con lo stesso termine.
In inglese questi sistemi vengono anche indicati con i termini: issue tracking system (ITS), trouble ticket system, support ticket, request management o incident ticket system.
	
	\\ \\
	
	\textbf{Timeout} & è un periodo di tempo predeterminato nel quale una data operazione deve essere terminata.
	
	\\ \\
	
	\textbf{\Huge{U}} & 
	
	\\ \\

	\textbf{Ubuntu} & Ubuntu è una distribuzione GNU/Linux, basata su Debian, nata nel 2004. La sua principale caratteristica è la focalizzazione sull'utente e la facilità di utilizzo. Essa viene pubblicata come software libero sotto licenza GNU GPL, è distribuita gratuitamente ed è liberamente modificabile. Ubuntu è orientata all'utilizzo desktop e pone una grande attenzione al supporto hardware.
	
	\\ \\
	
	\textbf{UML} & In ingegneria del software, UML (Unified Modeling Language, "linguaggio di modellazione unificato") è un linguaggio di modellazione e specifica basato sul paradigma object-oriented.
	
	\\ \\
	
	\textbf{URL} & La locuzione Uniform Resource Locator (in acronimo URL), nella terminologia delle telecomunicazioni e dell'informatica è una sequenza di caratteri che identifica univocamente l'indirizzo di una risorsa in Internet, tipicamente presente su un host server, come ad esempio un documento, un'immagine, un video, rendendola accessibile ad un client che ne faccia richiesta attraverso l'utilizzo di un web browser.
	
	\\ \\
	
	\textbf{Use Case} & funzione o servizio offerto dal sistema ad uno o più attori ovvero entità che interagiscono col sistema.
	
	\\ \\	

	
	\textbf{\Huge{V}} & 
	
	\\ \\
	
	\textbf{\Huge{W}} & 
	
	\\ \\
	
	\textbf{W3C} & Il World Wide Web Consortium, anche conosciuto come W3C, è un'organizzazione non governativa internazionale che ha come scopo quello di sviluppare tutte le potenzialità del World Wide Web. Al fine di riuscire nel proprio intento, la principale attività svolta dal W3C consiste nello stabilire standard tecnici per il World Wide Web inerenti sia i linguaggi di markup che i protocolli di comunicazione.
	
	\\ \\
	
	\textbf{WhatsApp} & WhatsApp è un'applicazione proprietaria di messaggistica istantanea multi-piattaforma per smartphone. Oltre allo scambio di messaggi testuali è possibile inviare immagini, video, file audio e condividere la propria posizione (grazie all'uso di mappe integrate nel dispositivo) con chiunque abbia uno smartphone dotato di connessione a Internet e abbia installato l'applicazione.
	
	\\ \\
	
	\textbf{\Huge{X}} & 
	
	\\ \\
	
	\textbf{\Huge{Y}} & 
	
	\\ \\
	
	\textbf{\Huge{Z}} & 
		
\end{longtable}