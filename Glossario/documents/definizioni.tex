
\cleardoublepage
\phantomsection
\addcontentsline{toc}{section}{\#}
\noindent\hrulefill\hspace{4mm}\textbf{\textsl{\Huge{\#}}}\hspace{4mm}\hrulefill

\vspace*{2\bigskipamount}

\begin{addmargin}[0em]{0em}	
	\textbf{\underline{3DJS}} 
\end{addmargin}
	
\medskip
\begin{addmargin}[5em]{1em}	
È una libreria Javascript per visualizzare interattivamente grafici di grandi dimensioni in ambienti web.
\end{addmargin}

\newpage

\cleardoublepage
\phantomsection
\addcontentsline{toc}{section}{A}
\noindent\hrulefill\hspace{4mm}\textbf{\textsl{\Huge{A}}}\hspace{4mm}\hrulefill

\vspace*{2\bigskipamount}

\begin{addmargin}[0em]{0em}
	\textbf{\underline{Alert}} 
\end{addmargin}
	
\medskip
\begin{addmargin}[5em]{1em}
Segnale o messaggio (finestra) di avviso, allarme, avvertimento.
\end{addmargin}

\bigskip
\begin{addmargin}[0em]{0em}
	\textbf{\underline{AngularJS}} 
\end{addmargin}
	
\medskip
\begin{addmargin}[5em]{1em}
È un framework open-source mantenuto da Google e da una comunità di sviluppatori e corporazioni, il cui scopo è quello di facilitare lo sviluppo di applicazioni web sempre più simili a quelle desktop.
\end{addmargin}

\newpage

\cleardoublepage
\phantomsection
\addcontentsline{toc}{section}{B}
\noindent\hrulefill\hspace{4mm}\textbf{\textsl{\Huge{B}}}\hspace{4mm}\hrulefill

\vspace*{2\bigskipamount}


\begin{addmargin}[0em]{0em}
	\textbf{\underline{Best Practice}} 
\end{addmargin}
	
\medskip
\begin{addmargin}[5em]{1em}
Miglior pratica o buona prassi, indica le esperienze più significative, o comunque quelle che hanno permesso di ottenere migliori risultati, in uno specifico contesto.
\end{addmargin}

\bigskip
\begin{addmargin}[0em]{0em}
	\textbf{\underline{BOM}} 
\end{addmargin}
	
\medskip
\begin{addmargin}[5em]{1em}
Il Byte Order Mark (BOM) è una piccola sequenza di byte che viene posizionata
all'inizio di un flusso di dati di puro testo (tipicamente un file) per indicarne il
tipo di codifica Unicode.
\end{addmargin}

\bigskip
\begin{addmargin}[0em]{0em}	
	\textbf{\underline{Brainstorming}}
\end{addmargin}
	
\medskip
\begin{addmargin}[5em]{1em}	
L'espressione brainstorming è una tecnica creativa di gruppo per far emergere idee volte alla risoluzione di un problema. Consiste, dato un problema, nell'organizzare una riunione in cui ogni partecipante propone liberamente soluzioni di ogni tipo (anche strampalate, paradossali o con poco senso apparente) al problema, senza che nessuna di esse venga minimamente censurata.
\end{addmargin}

\bigskip
\begin{addmargin}[0em]{0em}
	\textbf{\underline{Branch}} 
\end{addmargin}
	
\medskip
\begin{addmargin}[5em]{1em}
Branch rappresenta una linea indipendente di sviluppo di un software su cui poter lavorare 
senza entrare in conflitto con la versione principale o con il lavoro degli altri sviluppatori.
\end{addmargin}

\bigskip
\begin{addmargin}[0em]{0em}	
	\textbf{\underline{Break}}
\end{addmargin}
	
\medskip
\begin{addmargin}[5em]{1em}	
Il comando break viene utilizzato per uscire da un ciclo prima che la condizione di test divenga false. Quando si esce da un ciclo con un'istruzione break, l'esecuzione del programma continua con l'istruzione che segue il ciclo stesso.
\end{addmargin}

\bigskip
\begin{addmargin}[0em]{0em}	
	\textbf{\underline{Browser}}
\end{addmargin}
	
\medskip
\begin{addmargin}[5em]{1em}	
Programma che fornisce uno strumento per navigare in Internet e interagire con il World Wide Web.
\end{addmargin}

\newpage

\cleardoublepage
\phantomsection
\addcontentsline{toc}{section}{C}
\noindent\hrulefill\hspace{4mm}\textbf{\textsl{\Huge{C}}}\hspace{4mm}\hrulefill

\vspace*{2\bigskipamount}

\begin{addmargin}[0em]{0em}
	\textbf{\underline{Callback}}
\end{addmargin}

\medskip
\begin{addmargin}[5em]{1em}
	Una funzione callback è una parte di codice eseguibile che viene passata come argomento ad un'altra funzione, la quale è tenuta ad un certo punto a eseguire l'attributo non appena vengono soddisfatte determinate circostanze.
\end{addmargin}		

\bigskip
\begin{addmargin}[0em]{0em}
	\textbf{\underline{Checkpoint}}
\end{addmargin}

\medskip
\begin{addmargin}[5em]{1em}
	Il Checkpoint corrisponde ad un punto della presentazione in cui c'è la possibilità di entrare nel cammino di specializzazione o di continuare sequenzialmente con il frame successivo.  	
\end{addmargin}	

\bigskip
\begin{addmargin}[0em]{0em}
	\textbf{\underline{Client}}
\end{addmargin}

\medskip
\begin{addmargin}[5em]{1em}
	Un client in informatica, indica una componente che accede ai servizi o alle risorse di un'altra componente detta server. Si può quindi parlare di client riferendosi all'hardware oppure al software. Esso fa parte dunque dell'architettura logica di rete detta client-server.
Il termine client indica anche il software usato sul computer client per accedere alle funzionalità offerte dal server.	
\end{addmargin}		

\bigskip
\begin{addmargin}[0em]{0em}
	\textbf{\underline{Cloud}}
\end{addmargin}
	
\medskip
	\begin{addmargin}[5em]{1em}
Il Cloud o Cloud Storage è un modello di conservazione dati su computer in rete dove i dati stessi sono memorizzati su molteplici server virtuali generalmente ospitati presso strutture di terze parti o su server dedicati.	
\end{addmargin}	

\bigskip
\begin{addmargin}[0em]{0em}	
	\textbf{\underline{Commit}}
\end{addmargin}

\medskip
\begin{addmargin}[5em]{1em}	
Un commit (o, più raramente, install, submit, check-in o ci) si effettua quando si copiano le modifiche fatte su file locali nella directory del repository (il software di controllo versione controlla quali file sono stati modificati dall'ultima sincronizzazione).
\end{addmargin}

\bigskip
\begin{addmargin}[0em]{0em}
	\textbf{\underline{Controllo di versione distribuito}}
\end{addmargin}

\medskip
\begin{addmargin}[5em]{1em}	
È un sistema che permette a più sviluppatori lavorare ad un progetto senza dover necessariamente essere connessi ad una rete. Anziché lavorare in un'unica repository comune, ogni sviluppatore lavora su di una propria copia del progetto che può essere sincronizzata con l'originale in un secondo momento.
\end{addmargin}

\bigskip
\begin{addmargin}[0em]{0em}	
	\textbf{\underline{Cookie}}
\end{addmargin}
	
\medskip
\begin{addmargin}[5em]{1em}	
In informatica i cookie HTTP sono righe di testo usate per eseguire autenticazioni automatiche, tracciatura di sessioni e memorizzazione di informazioni specifiche riguardanti gli utenti che accedono al server. Nel dettaglio, sono stringhe di testo di piccola dimensione inviate da un server ad un Web client (di solito un browser) e poi rimandati indietro dal client al server (senza subire modifiche) ogni volta che il client accede alla stessa porzione dello stesso dominio web.
\end{addmargin}

\bigskip
\begin{addmargin}[0em]{0em}	
	\textbf{\underline{Chrome}}
\end{addmargin}
	
\medskip
\begin{addmargin}[5em]{1em}	
Google Chrome (detto anche semplicemente Chrome) è un browser sviluppato da Google, basato, a partire dalla versione 28, sul motore di rendering Blink.
\end{addmargin}

\bigskip
\begin{addmargin}[0em]{0em}	
	\textbf{\underline{CSS}}
\end{addmargin}

\medskip
	\begin{addmargin}[5em]{1em}	
Acronimo di Cascading Style Sheets, in italiano fogli di stile è un linguaggio di programmazione che definisce lo stile di formattazione delle pagine web.
\end{addmargin}

\newpage

\cleardoublepage
\phantomsection
\addcontentsline{toc}{section}{D}
\noindent\hrulefill\hspace{4mm}\textbf{\textsl{\Huge{D}}}\hspace{4mm}\hrulefill

\vspace*{2\bigskipamount}

\begin{addmargin}[0em]{0em}	
	\textbf{\underline{Database}}
\end{addmargin}

\medskip
\begin{addmargin}[5em]{1em}	
In italiano base di dati indica un archivio dati, o un insieme di archivi ben strutturati, in cui le informazioni in esso contenute sono strutturate e collegate tra loro secondo un particolare modello logico (relazionale, gerarchico, reticolare o a oggetti) in modo tale da consentire la gestione/organizzazione efficiente dei dati stessi e l'interfacciamento con le richieste dell'utente attraverso i cosiddetti query language (query di ricerca o interrogazione, inserimento, cancellazione, aggiornamento ecc...
\end{addmargin}	

\bigskip
\begin{addmargin}[0em]{0em}	
	\textbf{\underline{Design pattern}}
\end{addmargin}

\medskip
\begin{addmargin}[5em]{1em}	
In informatica, nell'ambito dell'ingegneria del software, un design pattern, è un concetto che può essere definito "una soluzione progettuale generale ad un problema ricorrente". Si tratta di una descrizione o modello logico da applicare per la risoluzione di un problema che può presentarsi in diverse situazioni durante le fasi di progettazione e sviluppo del software, ancor prima della definizione dell'algoritmo risolutivo della parte computazionale.
\end{addmargin}	

\bigskip
\begin{addmargin}[0em]{0em}	
	\textbf{\underline{Diagramma delle attività}}
\end{addmargin}
	
\medskip
\begin{addmargin}[5em]{1em}	
È un diagramma definito all'interno dello Unified Modeling Language (UML) che definisce le attività da svolgere per realizzare una data funzionalità. Il diagramma definisce inoltre i responsabili per le singole attività e i punti di decisione. 
\end{addmargin}	

\bigskip
\begin{addmargin}[0em]{0em}	
	\textbf{\underline{Dom}}
\end{addmargin}
	
\medskip
\begin{addmargin}[5em]{1em}	
	DOM (acronimo di Document Object Model), letteralmente modello a oggetti del documento, è una forma di rappresentazione dei documenti strutturati come modello orientato agli oggetti.
DOM è lo standard ufficiale del W3C per la rappresentazione di documenti strutturati in maniera da essere neutrali sia per la lingua che per la piattaforma. DOM è inoltre la base per una vasta gamma di interfacce di programmazione delle applicazioni; alcune di esse sono standardizzate dal W3C.
\end{addmargin}	

\bigskip
\begin{addmargin}[0em]{0em}	
	\textbf{\underline{Drag and Drop}}
\end{addmargin}
	
\medskip
\begin{addmargin}[5em]{1em}	
Nell'interfaccia grafica di un computer, il drag-and-drop indica una successione di tre azioni, consistenti nel cliccare su un oggetto virtuale (quale una finestra o un'icona) per trascinarlo (in inglese: drag) in un'altra posizione, dove viene rilasciato (in inglese: drop).
\end{addmargin}	
	
\newpage

\cleardoublepage
\phantomsection
\addcontentsline{toc}{section}{E}
\noindent\hrulefill\hspace{4mm}\textbf{\textsl{\Huge{E}}}\hspace{4mm}\hrulefill
\vspace*{2\bigskipamount}

\begin{addmargin}[0em]{0em}
	\textbf{\underline{Editor}} 
\end{addmargin}
	
\medskip
\begin{addmargin}[5em]{1em}	
Editor in informatica è un programma per la visualizzazione e la modifica di testo o immagini.
\end{addmargin}

\newpage

\cleardoublepage
\phantomsection
\addcontentsline{toc}{section}{F}
\noindent\hrulefill\hspace{4mm}\textbf{\textsl{\Huge{F}}}\hspace{4mm}\hrulefill
\vspace*{2\bigskipamount}

\begin{addmargin}[0em]{0em}		
	\textbf{\underline{Facebook}}
\end{addmargin}
	
\medskip
\begin{addmargin}[5em]{1em}	
Facebook è un servizio di rete sociale, posseduto e gestito dalla corporation Facebook Inc.
\end{addmargin}	

\bigskip
\begin{addmargin}[0em]{0em}
	\textbf{\underline{Frame}}
\end{addmargin}

\medskip
\begin{addmargin}[5em]{1em}	
Nel contesto della presentazione, il frame è un oggetto contenitore in grado di contenere oggetti grafici, come immagini, testo e shape. Inoltre ogni frame può rappresentare un nodo del cammino presentativo.
\end{addmargin}	

\bigskip
\begin{addmargin}[0em]{0em}	
	\textbf{\underline{Framework}}
\end{addmargin}

\medskip
\begin{addmargin}[5em]{1em}	
In informatica, e specificatamente nello sviluppo software, un framework è un'architettura logica di supporto (spesso un'implementazione logica di un particolare design pattern) su cui un software può essere progettato e realizzato, spesso facilitandone lo sviluppo da parte del programmatore.
\end{addmargin}	

\bigskip
\begin{addmargin}[0em]{0em}	
	\textbf{\underline{Freeware}} 
\end{addmargin}

\medskip
\begin{addmargin}[5em]{1em}	
Il termine freeware indica un software che viene distribuito in modo gratuito.
\end{addmargin}	

\newpage

\cleardoublepage
\phantomsection
\addcontentsline{toc}{section}{G}
\noindent\hrulefill\hspace{4mm}\textbf{\textsl{\Huge{G}}}\hspace{4mm}\hrulefill

\vspace*{2\bigskipamount}

\begin{addmargin}[0em]{0em}	
	\textbf{\underline{Gantt}} 
\end{addmargin}

\medskip
\begin{addmargin}[5em]{1em}	
Il diagramma di Gantt è uno strumento di supporto alla gestione dei progetti ed è costruito partendo da un asse orizzontale - a rappresentazione dell'arco temporale totale del progetto, suddiviso in fasi incrementali (ad esempio, giorni, settimane, mesi) - e da un asse verticale - a rappresentazione delle mansioni o attività che costituiscono il progetto. 
\end{addmargin}	

\bigskip
\begin{addmargin}[0em]{0em}	
	\textbf{\underline{Git}}
\end{addmargin}

\medskip
\begin{addmargin}[5em]{1em}	
Git è un sistema software di controllo di versione distribuito, creato da Linus Torvalds nel 2005 e disponibile per tutti i principali sistemi operativi.
\end{addmargin}

\bigskip
\begin{addmargin}[0em]{0em}	
	\textbf{\underline{GitHub}}
\end{addmargin}

\medskip
\begin{addmargin}[5em]{1em}	
GitHub è un servizio di hosting di repository che usa il sistema di controllo di versione Git.
Per maggiori informazioni visitare il sito \url{http://www.github.com/}.
\end{addmargin}

\bigskip
\begin{addmargin}[0em]{0em}	
	\textbf{\underline{Gmail}}
\end{addmargin}

\medskip
\begin{addmargin}[5em]{1em}	
Gmail (Google Mail) è un servizio gratuito di posta elettronica via web (Webmail), POP3 e IMAP, fornito da Google.
\end{addmargin}	

\bigskip
\begin{addmargin}[0em]{0em}
	\textbf{\underline{Google Calendar}}
\end{addmargin}

\medskip
\begin{addmargin}[5em]{1em}	
Google Calendar è un sistema gratuito di calendari integrato nell'account Google.
Funziona come un'agenda nella quale inserire degli eventi, e il suo utilizzo può essere pubblico o privato.
\end{addmargin}

\bigskip
\begin{addmargin}[0em]{0em}
	\textbf{\underline{Google Drive}}
\end{addmargin}

\medskip
\begin{addmargin}[5em]{1em}	
Google Drive è un servizio web integrato nell'account Google per l'archiviazione e la condivisione di file e documenti.
Include, tra le altre funzionalità,  l'applicazione web Google Docs per visualizzare e modificare documenti direttamente 
dal browser.	
\end{addmargin}	

\newpage

\cleardoublepage
\phantomsection
\addcontentsline{toc}{section}{H}
\noindent\hrulefill\hspace{4mm}\textbf{\textsl{\Huge{H}}}\hspace{4mm}\hrulefill

\vspace*{2\bigskipamount}



\begin{addmargin}[0em]{0em}	
	\textbf{\underline{Header}}
\end{addmargin}

\medskip
\begin{addmargin}[5em]{1em}		
	L'elemento \texttt{<HEAD>} in un file HTML è un contenitore di informazioni riguardanti il documento HTML. Tipicamente definisce il titolo, lo stile, collegamenti, script e altre informazioni.
\end{addmargin}	

\bigskip
\begin{addmargin}[0em]{0em}	
	\textbf{\underline{HTML5}}
\end{addmargin}

\medskip
\begin{addmargin}[5em]{1em}	
L'HTML5 è un linguaggio di markup per la strutturazione delle pagine web, e da Ottobre 2014 pubblicato come W3C Recommendation.
\end{addmargin}	

\newpage

\cleardoublepage
\phantomsection
\addcontentsline{toc}{section}{ I }
\noindent\hrulefill\hspace{4mm}\textbf{\textsl{\Huge{I}}}\hspace{4mm}\hrulefill

\vspace*{2\bigskipamount}

\begin{addmargin}[0em]{0em}	
	\textbf{\underline{ImpressJS}}
\end{addmargin}

\medskip
\begin{addmargin}[5em]{1em}	
	È un framework di presentazione basato sulla potenza delle trasformazioni e delle transizioni del CSS3 nei moderni browser e ispirato all'idea che sta sotto prezi.com.
\end{addmargin}	

\bigskip

\begin{addmargin}[0em]{0em}	
	\textbf{\underline{In-memory Database}}
\end{addmargin}

\medskip
\begin{addmargin}[5em]{1em}	
	In-memory Database (o IMDB) è un DBMS che gestisce i dati nella memoria centrale. Rispetto ai database che gestiscono i dati su memorie di massa, gli IMDB sono molto più veloci a discapito però della quantità di dati gestibili, che si riduce di molto. Inoltre gli IMDB necessitano comunque di un modo per il recupero dei dati in caso di guasti.
\end{addmargin}	

\bigskip
\begin{addmargin}[0em]{0em}	
	\textbf{\underline{Infografica}}
\end{addmargin}

\medskip
\begin{addmargin}[5em]{1em}		
	L'infografica è l'informazione proiettata in forma più grafica e visuale che testuale. Le immagini utilizzate, elaborate tramite computer su palette grafiche elettroniche, possono essere 2D o 3D, animate o fisse.
\end{addmargin}	

\bigskip
\begin{addmargin}[0em]{0em}		
	\textbf{\underline{Instant Messaging}}
\end{addmargin}

\medskip
\begin{addmargin}[5em]{1em}	
La messaggistica istantanea (in lingua inglese instant messaging) è una categoria di sistemi di comunicazione sincrona in tempo reale in rete, tipicamente Internet o una rete locale, che permette ai suoi utilizzatori lo scambio di brevi messaggi.
\end{addmargin}	

\bigskip
\begin{addmargin}[0em]{0em}		
	\textbf{\underline{InteractJS}}
\end{addmargin}

\medskip
\begin{addmargin}[5em]{1em}	
InteractJS è un modulo Javascript per il drag and drop, ridimensionamento e per i gestire i comandi multi-touch in modo rapido e semplice per i moderni browser.
\end{addmargin}

\bigskip
\begin{addmargin}[0em]{0em}
	\textbf{\underline{ISO}}
\end{addmargin}
	
\medskip
\begin{addmargin}[5em]{1em}	
Acronimo di Internatonal Organization for Standardization. È la più importante organizzazione internazionale per la definizione di norme tecniche. 
\end{addmargin}

\bigskip
\begin{addmargin}[0em]{0em}
	\textbf{\underline{IEC}}
\end{addmargin}
	
\medskip
\begin{addmargin}[5em]{1em}	
IEC acronimo di International Electronic Commission. È un organizzazione internazionale per la definizione di standard in materia di elettronica elettricità e tecnologie correlate. Alcuni dei suoi standard sono definiti assieme a ISO.
\end{addmargin}	

\bigskip
\begin{addmargin}[0em]{0em}
	\textbf{\underline{ISO/IEC 14598}}
\end{addmargin}
	
\medskip
\begin{addmargin}[5em]{1em}	
ISO/IEC 14598 descrive il processo di valutazione del software.  
\end{addmargin}	

\newpage

\cleardoublepage
\phantomsection
\addcontentsline{toc}{section}{J}
\noindent\hrulefill\hspace{4mm}\textbf{\textsl{\Huge{J}}}\hspace{4mm}\hrulefill

\vspace*{2\bigskipamount}

\begin{addmargin}[0em]{0em}	
	\textbf{\underline{JavaScript}}
\end{addmargin}

\medskip
\begin{addmargin}[5em]{1em}
È un linguaggio di programmazione interpretato sviluppato da Netscape orientato agli oggetti e agli eventi.
\end{addmargin}
	
\bigskip
\begin{addmargin}[0em]{0em}	
	\textbf{\underline{JPEG - JPG}}
\end{addmargin}

\medskip
\begin{addmargin}[5em]{1em}
JPEG (acronimo di Joint Photographic Experts Group) è un comitato ISO/CCITT che ha definito il primo standard internazionale di compressione dell'immagine digitale a tono continuo, sia a livelli di grigio che a colori.
"JPEG" indica quindi anche il diffusissimo formato di compressione a perdita di informazioni ed è un formato aperto e ad implementazione gratuita.
\end{addmargin}	

\bigskip
\begin{addmargin}[0em]{0em}	
	\textbf{\underline{JSON}}
\end{addmargin}

\medskip
\begin{addmargin}[5em]{1em}
JSON (acronimo di JavaScript Object Notation), è un formato adatto all'interscambio di dati fra applicazioni client-server. È basato sul linguaggio JavaScript Standard ECMA-262 3ª edizione dicembre 1999, ma ne è indipendente. Viene usato in AJAX come alternativa a XML/XSLT.
\end{addmargin}

\newpage

\cleardoublepage
\phantomsection
\addcontentsline{toc}{section}{L}
\noindent\hrulefill\hspace{4mm}\textbf{\textsl{\Huge{L}}}\hspace{4mm}\hrulefill

\vspace*{2\bigskipamount}

\begin{addmargin}[0em]{0em}	
	\textbf{\underline{Lazy Evaluation}}
\end{addmargin}

\medskip
\begin{addmargin}[5em]{1em}	
La Lazy Evaluation (valutazione pigra) è una tecnica che consiste nel ritardare valutazione di una espressione finché il risultato non è richiesto effettivamente, evitando così ripetute valutazioni della stessa espressione.
\end{addmargin}	


\bigskip

\begin{addmargin}[0em]{0em}	
	\textbf{\underline{LF}}
\end{addmargin}

\medskip
\begin{addmargin}[5em]{1em}	
I sistemi basati su ASCII utilizzano il Line feed (LF, 0x0A, 10 in decimale) oppure il Carriage return (CR, 0x0D, 13 in decimale) per indicare che una riga vada a capo.
\end{addmargin}	


\bigskip
\begin{addmargin}[0em]{0em}	
	\textbf{\underline{Libreria}}
\end{addmargin}

\medskip
\begin{addmargin}[5em]{1em}	
Una libreria, in Informatica, è un insieme di funzioni o strutture dati predefinite e predisposte per essere collegate ad un programma software attraverso opportuno collegamento.
\end{addmargin}	


\bigskip
\begin{addmargin}[0em]{0em}	
	\textbf{\underline{Lightning}}
\end{addmargin}
	
\medskip
\begin{addmargin}[5em]{1em}	
E' un plugin utile per sincronizzare il calendario di google.
\end{addmargin}


\newpage

\cleardoublepage
\phantomsection
\addcontentsline{toc}{section}{M}
\noindent\hrulefill\hspace{4mm}\textbf{\textsl{\Huge{M}}}\hspace{4mm}\hrulefill

\vspace*{2\bigskipamount}

\begin{addmargin}[0em]{0em}	
	\textbf{\underline{Macchina Virtuale}} 
\end{addmargin}

\medskip
\begin{addmargin}[5em]{1em}
In informatica il termine macchina virtuale (VM dall'inglese Virtual Machine) indica un software che, attraverso un processo di virtualizzazione, crea un ambiente virtuale che emula tipicamente il comportamento di una macchina fisica grazie all'assegnazione di risorse hardware (porzioni di disco rigido, RAM e risorse di processamento) ed in cui alcune applicazioni possono essere eseguite come se interagissero con tale macchina.
\end{addmargin}	

\bigskip
\begin{addmargin}[0em]{0em}	
	\textbf{\underline{Mailing List}}
\end{addmargin}

\medskip
\begin{addmargin}[5em]{1em}	
La mailing list (letteralmente, lista di corrispondenza, dalla lingua inglese; traducibile in italiano sia con lista di distribuzione o di diffusione, sia con lista di discussione, a seconda degli usi) è un servizio/strumento offribile da una rete di computer verso vari utenti e costituito da un sistema organizzato per la partecipazione di più persone ad una discussione asincrona o per la distribuzione di informazioni utili agli interessati/iscritti attraverso l'invio di email ad una lista di indirizzi di posta elettronica di utenti iscritti.
\end{addmargin}	

\bigskip
\begin{addmargin}[0em]{0em}	
	\textbf{\underline{Markup}} 
\end{addmargin}

\medskip
\begin{addmargin}[5em]{1em}
In generale un linguaggio di markup è un insieme di regole che descrivono i meccanismi di rappresentazione (strutturali, semantici o presentazionali) di un testo che, utilizzando convenzioni standardizzate, sono utilizzabili su più supporti. La tecnica di formattazione per mezzo di marcatori (o espressioni codificate) richiede quindi una serie di convenzioni, ovvero appunto di un linguaggio a marcatori di documenti. 	
\end{addmargin}	

\bigskip
\begin{addmargin}[0em]{0em}	
	\textbf{\underline{Material design}}
\end{addmargin}
	
\medskip
\begin{addmargin}[5em]{1em}
Sono delle linee guida date da Google per la creazione di interfacce web.	
\end{addmargin}	

\bigskip
\begin{addmargin}[0em]{0em}	
	\textbf{\underline{Mega}} 
\end{addmargin}
	
\medskip
\begin{addmargin}[5em]{1em}
È un servizio di storage online che mette a disposizione uno spazio gratuito di 50GB. 
\end{addmargin}	

\bigskip
\begin{addmargin}[0em]{0em}	
	\textbf{\underline{Merge}} 
\end{addmargin}

\medskip
\begin{addmargin}[5em]{1em}	
All'interno di una repository, significa riunire un branch di copia con il branch originale, aggiornando solamente quei files che sono stati modificati nel branch di copia indipendentemente
dai cambiamenti che sono avvenuti nel branch originale durante il periodo di separazione dei due.
\end{addmargin}

\bigskip
\begin{addmargin}[0em]{0em}	
	\textbf{\underline{MeteorJS}} 
\end{addmargin}

\medskip
\begin{addmargin}[5em]{1em}	
MeteorJS è una piattaforma completa per la costruzione di web e mobile app scritte in puro Javascript.
\end{addmargin}

\bigskip
\begin{addmargin}[0em]{0em}	
	\textbf{\underline{Milestone (pietra miliare)}} 
\end{addmargin}
	
\medskip
\begin{addmargin}[5em]{1em}	
È un termine usato per indicare delle tappe intermedie nello sviluppo di un progetto, che consentono di monitorare con maggiore efficacia il suo progresso.
\end{addmargin}	

\bigskip
\begin{addmargin}[0em]{0em}	
	\textbf{\underline{minimongo}} 
\end{addmargin}
	
\medskip
\begin{addmargin}[5em]{1em}	
È una re-implementazione dell'intero (o quasi) database MongoDB$_G$ che sfrutta un database Javascript all'interno della memoria principale. Viene utilizzato da MeteorJS$_G$ come database locale del client per la visualizzazione i tempo reale delle modifiche apportate ai dati, effettuando in un secondo momento la sincronizzazione con il database del server.
\end{addmargin}	


\bigskip
\begin{addmargin}[0em]{0em}	
	\textbf{\underline{Mockup}} 
\end{addmargin}
	
\medskip
\begin{addmargin}[5em]{1em}	
È un modello in scala o a dimensione reale di un progetto o un dispositivo, il cui scopo è quello di promuovere, descrivere o valutare il prodotto finale. Se un Mockup possiede almeno una parte delle funzionalità del prodotto finale e consente il loro collaudo allora viene definito \textit{prototipo}.
\end{addmargin}	

\bigskip
\begin{addmargin}[0em]{0em}	
	\textbf{\underline{MongoDB}} 
\end{addmargin}
	
\medskip
\begin{addmargin}[5em]{1em}	
MongoDB è un DBMS non relazionale, orientato ai documenti. Classificato come un database di tipo NoSQL, MongoDB si allontana dalla struttura tradizionale basata su tabelle dei database relazionali in favore di documenti in stile JSON con schema dinamico (MongoDB chiama il formato BSON), rendendo l'integrazione di dati di alcuni tipi di applicazioni più facile e veloce. 
\end{addmargin}	

\bigskip
\begin{addmargin}[0em]{0em}	
	\textbf{\underline{Mozilla Thunderbird}} 
\end{addmargin}

\medskip
\begin{addmargin}[5em]{1em}	
È il programma di posta elettronica sviluppato da Mozilla, pensato come naturale complemento del browser web Firefox. 
Come Firefox, Thunderbird è un software libero, utilizzabile liberamente da chiunque lo desideri.
\end{addmargin}	

\bigskip
\begin{addmargin}[0em]{0em}	
	\textbf{\underline{MVC}} 
\end{addmargin}

\medskip
\begin{addmargin}[5em]{1em}	
Model-View-Controller, è un design pattern architetturale molto diffuso nello sviluppo di sistemi software, in particolare nell'ambito della programmazione ad oggetti, in grado di separare la logica di presentazione dei dati dalla logica di business. Il pattern è basato sulla separazione dei compiti fra i componenti software che interpretano tre ruoli principali: model fornisce i metodi per accedere ai dati utili all'applicazione, view visualizza i dati contenuti nel model e si occupa dell'interazione con utenti e agenti, controller riceve i comandi dell'utente (in genere attraverso il view) e li attua modificando lo stato degli altri due componenti.
\end{addmargin}	

\bigskip
\begin{addmargin}[0em]{0em}	
	\textbf{\underline{MVVP}} 
\end{addmargin}

\medskip
\begin{addmargin}[5em]{1em}
Model-View-ViewModel, derivazione del pattern Model-View-Controller che sposta l'elaborazione e il modo con cui vengono visualizzati i dati nel ViewModel, e sempre tramite il ViewModel crea un collegamento diretto tra View e Model.
\end{addmargin}	

\newpage

\cleardoublepage
\phantomsection
\addcontentsline{toc}{section}{N}
\noindent\hrulefill\hspace{4mm}\textbf{\textsl{\Huge{N}}}\hspace{4mm}\hrulefill

\vspace*{2\bigskipamount}

\begin{addmargin}[0em]{0em}	
	\textbf{\underline{Namespace}} 
\end{addmargin}

\medskip
\begin{addmargin}[5em]{1em}
	Un namespace, in italiano spazio dei nomi, è nella terminologia relativa all'informatica una collezione di nomi di entità, definite dal programmatore, omogeneamente usate in uno o più file sorgente. Lo scopo dei namespace è quello di evitare confusione ed equivoci nel caso siano necessarie molte entità con nomi simili, fornendo il modo di raggruppare i nomi per categorie.
\end{addmargin}	

\bigskip

\begin{addmargin}[0em]{0em}	
	\textbf{\underline{Node.js}} 
\end{addmargin}

\medskip
\begin{addmargin}[5em]{1em}	
Node.js è un framework relativo all'utilizzo lato server di Javascript. Il modello su cui si basa Node.js è quello event-driven: Node richiede al sistema operativo di ricevere notifiche al verificarsi di determinati eventi, e rimane quindi in attesa fino alla notifica stessa; solo in tale momento torna attivo per eseguire le istruzioni previste nella funzione di callback$_G$, così chiamata perché da eseguire una volta ricevuta la notifica che il risultato dell'elaborazione del sistema operativo è disponibile. 
\end{addmargin}	

	
\newpage

\cleardoublepage
\phantomsection
\addcontentsline{toc}{section}{O}
\noindent\hrulefill\hspace{4mm}\textbf{\textsl{\Huge{O}}}\hspace{4mm}\hrulefill

\vspace*{2\bigskipamount}

\begin{addmargin}[0em]{0em}	
	\textbf{\underline{Open-Source}} 
\end{addmargin}
	
\medskip
\begin{addmargin}[5em]{1em}	
Open source, in informatica, indica un software di cui gli autori (più precisamente i detentori dei diritti) rendono pubblico il codice sorgente, favorendone il libero studio e permettendo a programmatori indipendenti di apportarvi modifiche. Questa possibilità è regolata tramite l'applicazione di apposite licenze d'uso.	
\end{addmargin}	
	
\newpage

\cleardoublepage
\phantomsection
\addcontentsline{toc}{section}{P}
\noindent\hrulefill\hspace{4mm}\textbf{\textsl{\Huge{P}}}\hspace{4mm}\hrulefill

\vspace*{2\bigskipamount}

\begin{addmargin}[0em]{0em}	
	\textbf{\underline{Package}} 
\end{addmargin}

\medskip
\begin{addmargin}[5em]{1em}	
È un meccanismo per organizzare classi in gruppi logici, principalmente (ma non solo) per definire namespace distinti per diversi contesti e per riunire classi(o entità analoghe, quali interfacce ed enumerazioni) logicamente correlate.
\end{addmargin}	

\bigskip

\begin{addmargin}[0em]{0em}	
	\textbf{\underline{PDF}} 
\end{addmargin}

\medskip
\begin{addmargin}[5em]{1em}	
Il Portable Document Format, comunemente abbreviato PDF, è un formato di file basato su un linguaggio di descrizione di pagina sviluppato da Adobe Systems nel 1993 per rappresentare documenti in modo indipendente dall'hardware e dal software utilizzati per generarli o per visualizzarli.
\end{addmargin}	

\bigskip
\begin{addmargin}[0em]{0em}	
	\textbf{\underline{PNG}} 
\end{addmargin}
\medskip
\begin{addmargin}[5em]{1em}
In informatica, il Portable Network Graphics (abbreviato PNG) è un formato di file per memorizzare immagini.
\end{addmargin}	

\bigskip
\begin{addmargin}[0em]{0em}	
	\textbf{\underline{Power Point}}
\end{addmargin} 

\medskip
\begin{addmargin}[5em]{1em}
Microsoft Office PowerPoint è il programma di presentazione prodotto da Microsoft, fa parte della suite di software di produttività personale Microsoft Office, è tutelato da copyright e distribuito con licenza commerciale ed è disponibile per i sistemi operativi Windows e Macintosh. È utilizzato principalmente per proiettare e quindi comunicare su schermo, progetti, idee, e contenuti potendo incorporare testo, immagini, grafici, filmati, audio e potendo presentare tutto questo con animazioni di alto livello.
\end{addmargin}	

\bigskip
\begin{addmargin}[0em]{0em}	
	\textbf{\underline{ProjectLibre}} 
\end{addmargin}
	
\medskip
\begin{addmargin}[5em]{1em}	
È un software gratuito per la gestione di progetti che permette di tenere traccia delle attività e delle varie fasi di lavoro.
\end{addmargin}

\bigskip
\begin{addmargin}[0em]{0em}
	\textbf{\underline{Publish-Subscribe}} 
\end{addmargin}
	
\medskip
\begin{addmargin}[5em]{1em}
È il design pattern$_G$ utilizzato da MeteorJS per la comunicazione tra il database del server MongoDB$_G$ e il database del client minimongo$_G$
\end{addmargin}	

\bigskip
\begin{addmargin}[0em]{0em}
	\textbf{\underline{Pull request}} 
\end{addmargin}
	
\medskip
\begin{addmargin}[5em]{1em}
È il meccanismo utilizzato nel software Git per avvisare i contributori di un repository delle modifiche che si vogliono apportare ai file in esso contenuti.	
\end{addmargin}	

\newpage
	
\cleardoublepage
\phantomsection
\addcontentsline{toc}{section}{R}
\noindent\hrulefill\hspace{4mm}\textbf{\textsl{\Huge{R}}}\hspace{4mm}\hrulefill

\vspace*{2\bigskipamount}

\bigskip
\begin{addmargin}[0em]{0em}	
	\textbf{\underline{Repository}}
\end{addmargin} 

\medskip
\begin{addmargin}[5em]{1em}
Repository è uno spazio di archiviazione da cui è possibile recuperare software o codice sorgente.
\end{addmargin}	

\bigskip
\begin{addmargin}[0em]{0em}	
	\textbf{\underline{Rete Sociale}} 
\end{addmargin}
	
\medskip
\begin{addmargin}[5em]{1em}
Una rete sociale (in lingua inglese social network) consiste in un qualsiasi gruppo di individui connessi tra loro da diversi legami sociali. Per gli esseri umani i legami vanno dalla conoscenza casuale, ai rapporti di lavoro, ai vincoli familiari.
\end{addmargin}	
	
\newpage
	
\cleardoublepage
\phantomsection
\addcontentsline{toc}{section}{S}
\noindent\hrulefill\hspace{4mm}\textbf{\textsl{\Huge{S}}}\hspace{4mm}\hrulefill

\vspace*{2\bigskipamount}

\begin{addmargin}[0em]{0em}
	\textbf{\underline{Server}}
\end{addmargin} 
	
\medskip
\begin{addmargin}[5em]{1em}	
	In informatica il termine server, indica genericamente un componente o sottosistema informatico di elaborazione che fornisce, a livello logico e a livello fisico, un qualunque tipo di servizio ad altre componenti che ne fanno richiesta attraverso una rete di computer, all'interno di un sistema informatico o direttamente in locale su un computer.
Al termine server, così come per il termine client, possono dunque riferirsi sia la componente hardware che la componente software che forniscono le funzionalità o servizi.
\end{addmargin}

\bigskip
\begin{addmargin}[0em]{0em}
	\textbf{\underline{Shape}}
\end{addmargin} 
	
\medskip
\begin{addmargin}[5em]{1em}	
Con il termine Shape si vuole identificare un oggetto grafico che può avere forme diverse e che ha lo scopo di abbellire il frame in cui viene inserito.
\end{addmargin}	

\bigskip
\begin{addmargin}[0em]{0em}
	\textbf{\underline{Singleton}}
\end{addmargin} 
	
\medskip
\begin{addmargin}[5em]{1em}	
Il singleton è un design pattern creazionale che ha lo scopo di garantire che di una determinata classe venga creata una e una sola istanza, e di fornire un punto di accesso globale a tale istanza.
\end{addmargin}	

\bigskip
\begin{addmargin}[0em]{0em}
	\textbf{\underline{Slack}}
\end{addmargin} 
	
\medskip
\begin{addmargin}[5em]{1em}	
Il periodo di slack rappresenta il tempo durante il quale un'attività può essere ritardata senza ritardare l'intero progetto di cui fa parte.
\end{addmargin}	

\bigskip
\begin{addmargin}[0em]{0em}		
	\textbf{\underline{SmartGit}}
\end{addmargin} 
	
\medskip
\begin{addmargin}[5em]{1em}	
SmartGit è un programma ad interfaccia grafica che permette di eseguire i comandi Git.	
\end{addmargin}	

\bigskip
\begin{addmargin}[0em]{0em}		
	\textbf{\underline{Skype}}
\end{addmargin}
	 
\medskip
\begin{addmargin}[5em]{1em}	
Skype è un software proprietario freeware di messaggistica istantanea e VoIP. Esso unisce caratteristiche presenti nei client più comuni (chat, salvataggio delle conversazioni, trasferimento di file) ad un sistema di telefonate basato su un network Peer-to-peer.  
\end{addmargin}	

\bigskip
\begin{addmargin}[0em]{0em}		
	\textbf{\underline{SVG}}
	\end{addmargin} 
	
\medskip
\begin{addmargin}[5em]{1em}	
Scalable Vector Graphics abbreviato in SVG, indica una tecnologia in grado di visualizzare oggetti di grafica vettoriale e, pertanto, di gestire immagini scalabili dimensionalmente.
Più specificamente si tratta di un linguaggio derivato dall'XML, cioè di un'applicazione del metalinguaggio posto a base degli sviluppi del Web da parte del consorzio W3C, che si pone l'obiettivo di descrivere figure bidimensionali statiche e animate.
\end{addmargin}	

\bigskip
\begin{addmargin}[0em]{0em}		
	\textbf{\underline{Switch}}
	\end{addmargin} 
	
\medskip
\begin{addmargin}[5em]{1em}	
Il costrutto di controllo switch e' una struttura che permette di controllare più condizioni in un unico 'blocco' di codice. Più' esattamente, lo switch consente di valutare un'espressione confrontandone il valore all'interno di un ristretto dominio di valori.
\end{addmargin}	

\newpage
	
\cleardoublepage
\phantomsection
\addcontentsline{toc}{section}{T}
\noindent\hrulefill\hspace{4mm}\textbf{\textsl{\Huge{T}}}\hspace{4mm}\hrulefill

\vspace*{2\bigskipamount}	

\begin{addmargin}[0em]{0em}		
	\textbf{\underline{Template}}
	\end{addmargin}
	 
\medskip
\begin{addmargin}[5em]{1em}	
Il termine inglese template in informatica indica un documento o programma nel quale, come in un foglio semicompilato cartaceo, su una struttura generica o standard esistono spazi temporaneamente "bianchi" da riempire successivamente.
\end{addmargin}

\bigskip
\begin{addmargin}[0em]{0em}		
	\textbf{\underline{TexMaker}}
\end{addmargin} 
	
\medskip
\begin{addmargin}[5em]{1em}	
È un strumento software utilizzato per produrre pdf mediante linguaggio \LaTeX{}.
\end{addmargin}

\bigskip
\begin{addmargin}[0em]{0em}		
	\textbf{\underline{Ticket o trouble ticket}}
\end{addmargin}
	
\medskip
\begin{addmargin}[5em]{1em}	 
Un trouble ticket indica letteralmente una richiesta di assistenza, tracciata da un sistema informatico di gestione delle richieste di assistenza, che per metonimia viene indicato con lo stesso termine.
In inglese questi sistemi vengono anche indicati con i termini: issue tracking system (ITS), trouble ticket system, support ticket, request management o incident ticket system.
\end{addmargin}

\bigskip
\begin{addmargin}[0em]{0em}	
	\textbf{\underline{Timeout}}
\end{addmargin}

\medskip
\begin{addmargin}[5em]{1em}
È un periodo di tempo predeterminato nel quale una data operazione deve essere terminata.
\end{addmargin}

\newpage

\cleardoublepage
\phantomsection
\addcontentsline{toc}{section}{U}
\noindent\hrulefill\hspace{4mm}\textbf{\textsl{\Huge{U}}}\hspace{4mm}\hrulefill
\vspace*{2\bigskipamount}	

\begin{addmargin}[0em]{0em}
	\textbf{\underline{Ubuntu}}
\end{addmargin}

\medskip
\begin{addmargin}[5em]{1em}
Ubuntu è una distribuzione GNU/Linux, basata su Debian, nata nel 2004. La sua principale caratteristica è la focalizzazione sull'utente e la facilità di utilizzo. Essa viene pubblicata come software libero sotto licenza GNU GPL, è distribuita gratuitamente ed è liberamente modificabile. Ubuntu è orientata all'utilizzo desktop e pone una grande attenzione al supporto hardware.
\end{addmargin}	

\bigskip
\begin{addmargin}[0em]{0em}	
	\textbf{\underline{UML}}
\end{addmargin}

\medskip
\begin{addmargin}[5em]{1em}
In ingegneria del software, UML (Unified Modeling Language, "linguaggio di modellazione unificato") è un linguaggio di modellazione e specifica basato sul paradigma object-oriented.
\end{addmargin}	

\bigskip
\begin{addmargin}[0em]{0em}	
	\textbf{\underline{Urigo: Angular-Meteor}}
\end{addmargin}

\medskip
\begin{addmargin}[5em]{1em}
Urigo è l'unione tra i due framework$_G$ AngularJS e MeteorJS. Questa unione è stata realizzata convertendo AngularJS in package per Meteor, e l'interazione tra i due avviene tramite dei Singleton$_G$ che fungono da servizi.
\end{addmargin}

\bigskip
\begin{addmargin}[0em]{0em}	
	\textbf{\underline{URL}}
\end{addmargin}

\medskip
\begin{addmargin}[5em]{1em}
La locuzione Uniform Resource Locator (in acronimo URL), nella terminologia delle telecomunicazioni e dell'informatica è una sequenza di caratteri che identifica univocamente l'indirizzo di una risorsa in Internet, tipicamente presente su un host server, come ad esempio un documento, un'immagine, un video, rendendola accessibile ad un client che ne faccia richiesta attraverso l'utilizzo di un web browser.
\end{addmargin}
 
\bigskip
\begin{addmargin}[0em]{0em}
	\textbf{\underline{Use Case}} 
\end{addmargin}

\medskip
\begin{addmargin}[5em]{1em}
Funzione o servizio offerto dal sistema ad uno o più attori ovvero entità che interagiscono col sistema.
\end{addmargin}	

\newpage

\cleardoublepage
\phantomsection
\addcontentsline{toc}{section}{V}
\noindent\hrulefill\hspace{4mm}\textbf{\textsl{\Huge{V}}}\hspace{4mm}\hrulefill
\vspace*{2\bigskipamount}

\begin{addmargin}[0em]{0em}
	\textbf{\underline{ViewModel}}
\end{addmargin}

\medskip
\begin{addmargin}[5em]{1em}
Vedi MVVP.
\end{addmargin}

\cleardoublepage
\phantomsection
\addcontentsline{toc}{section}{W}
\noindent\hrulefill\hspace{4mm}\textbf{\textsl{\Huge{W}}}\hspace{4mm}\hrulefill
\vspace*{2\bigskipamount}

\begin{addmargin}[0em]{0em}
	\textbf{\underline{W3C}}
\end{addmargin}

\medskip
\begin{addmargin}[5em]{1em}
Il World Wide Web Consortium, anche conosciuto come W3C, è un'organizzazione non governativa internazionale che ha come scopo quello di sviluppare tutte le potenzialità del World Wide Web. Al fine di riuscire nel proprio intento, la principale attività svolta dal W3C consiste nello stabilire standard tecnici per il World Wide Web inerenti sia i linguaggi di markup che i protocolli di comunicazione.
\end{addmargin}

\bigskip
\begin{addmargin}[0em]{0em}
	\textbf{\underline{Warning}}
\end{addmargin} 

\medskip
\begin{addmargin}[5em]{1em}
Un messaggio di Warning è una finestra di dialogo, un pop-up o una notifica che comunica all'utente una condizione che potrebbe causare problemi in futuro.
\end{addmargin}
	
\bigskip
\begin{addmargin}[0em]{0em}
	\textbf{\underline{WhatsApp}}
\end{addmargin} 

\medskip
\begin{addmargin}[5em]{1em}
WhatsApp è un'applicazione proprietaria di messaggistica istantanea multi-piattaforma per smartphone. Oltre allo scambio di messaggi testuali è possibile inviare immagini, video, file audio e condividere la propria posizione (grazie all'uso di mappe integrate nel dispositivo) con chiunque abbia uno smartphone dotato di connessione a Internet e abbia installato l'applicazione.
\end{addmargin}

\newpage

\cleardoublepage
\phantomsection
\addcontentsline{toc}{section}{X}
\noindent\hrulefill\hspace{4mm}\textbf{\textsl{\Huge{X}}}\hspace{4mm}\hrulefill
\vspace*{2\bigskipamount}

\begin{addmargin}[0em]{0em}
	\textbf{\underline{XHR}}
\end{addmargin} 

\medskip
\begin{addmargin}[5em]{1em}
 XHR (acronimo di XMLHttpRequest) è un set di API che possono essere usate da JavaScript, JScript, VBScript e altri linguaggi di scripting dei browser per trasferire XML o altri dati da e verso un web server tramite HTTP. Il più grande vantaggio di XMLHTTP è la possibilità di aggiornare dinamicamente una pagina web senza ricaricare l'intera pagina.
\end{addmargin}

		
