%-------  diagramma della classe%
\subsubsection{premi/client/}
\begin{figure}[H]
\begin{center}
\includegraphics[scale=0.55]{img/diacla/.png}
\caption{Diagramma della classe premi/client/}
\end{center}
\end{figure}


\begin{description}
%-------  descrizione della classe%
\item[Descrizione] \hfill
	Cos'e', cosa fa e a cosa serve
	
	
%-------  lista delle classi ereditate%	
\item[Classi ereditate] \hfill
	\begin{itemize}
		\item classe/ereditata
	\end{itemize}
	
	
%-------  lista delle classi associate%	
\item[Dipendenze] \hfill
	\begin{itemize}
		\item \textbf{classe/associata}: per fare cosa?
	\end{itemize}
	
	
%-------  lista degli Attributi%	
\item[Attributi] \hfill
	\begin{description}
		\item[\textbf{- int attributo1			}] \hfill
			Descrizione dell'attributo
		\item[\textbf{- int attributo2			}] \hfill
			Descrizione dell'attributo
	\end{description}
	
	
%-------  lista dei metodi%	
\item[Metodi] \hfill

	% -- inizio metodo -- %
	\begin{description}
		\item[\textbf{\color{blue}+ void operation0()			}] \hfill
			Descrizione del metodo
			
		\begin{description}
			% -- lista argomenti del metodo -- %
			\item[Argomenti] \hfill
				\begin{itemize}
				
					\item \textbf{nomeArgomento : tipoArgomento			} \hfill
					Descrizione argomento
					
				\end{itemize}
			% -- note aggiuntive sul metodo -- %
			\item[Note] \hfill
			\begin{itemize}
					\item Deve essere esplicitamente marcato come costante (?)
					\item Deve possedere qualche caratteristica
					\item Metodo ridefinito
			\end{itemize}
		\end{description}
	\end{description}
	% -- fine metodo -- %		

\end{description}






%-------  diagramma di un template %
\subsubsection{premi/client/}

\begin{description}
%-------  descrizione del template%
\item[Descrizione] \hfill
	Cos'e', cosa fa e a cosa serve
\item[Note] \hfill
	\begin{itemize}
			\item Deve possedere un bottone per richiamare il metodo nello scope..
			\item Mostra una variabile nello scope attraverso....
			\item Dev'essere fatto in questo modo....
	\end{itemize}
\end{description}