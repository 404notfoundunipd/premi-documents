\clearpage
\section{Specifica componenti}
Vengono qui descritti i metodi e gli attributi di ogni componente dell'applicazione. \\
\subsection{Collezioni di MongoDB} \label{collezioni}
Le informazioni in Meteor vengono salvate all'interno di collezioni di documenti, i quali non sono altro che collezioni di coppie di chiavi e valori. Sono facilmente manipolabili poiché si convertono naturalmente in oggetti JavaScript$_G$, e documenti della stessa collezione non devono per forza contenere le stesse chiavi (MongoDB possiede uno schema dinamico). \\
All'avvio di MeteorJS$_G$ vengono create, se non sono già presenti, quattro collezioni di documenti:
\begin{itemize}
\item \textbf{Presentations} contiene tutte le presentazioni create dagli utenti;
\item \textbf{Trails} contiene i percorsi di tutte le presentazioni;
\item \textbf{Frames} contiene le slides di tutte le presentazioni;
\item \textbf{Infographics} contiene l'infografica di ogni presentazione;
\end{itemize}

\subsection{Template}
Componenti marcati come \textit{template} sono speciali pagine HTML che non possiedono metodi e attributi propri, ma utilizzano quelli forniti dal loro controller dedicato attraverso lo \textit{\$scope} (l'oggetto di AngularJS$_G$ che funge da ViewModel$_G$); per questo motivo saranno solamente descritti tramite note e raccomandazioni che specificano come essi devano essere costruiti.

\subsection{Directive}
Componenti marcati come \textit{directive} aggiungono marcatori ad un elemento del DOM (un attributo, un tag, un commento o una classe CSS), e dicono al compilatore di AngularJS$_G$ che genera la view di applicare un certo comportamento all'elemento, o di trasformare l'elemento in qualcos'altro. \\ Possono interagire con tutti i componenti necessari alla compilazione: template, \$scope e controller.

\subsection{Servizi di AngularJS$_G$} \label{servizi}
I servizi di AngularJS sono oggetti intercambiabili che vengono collegati tra loro attraverso il pattern Dependency Injection$_G$. Sono quasi sempre dei Singleton$_G$, e vengono istanziati solo quando un componente dipende da loro.
I principali servizi utilizzati per lo sviluppo di questa applicazione sono:
\begin{itemize}
\item \textbf{\$scope} è il collegamento tra i controller e le view dell'applicazione. Il controller modella lo \$scope con attributi e metodi pubblici, che la view utilizza per mostrare e modificare il database in tempo reale. Ogni applicazione che utilizza AngularJS ha un solo \$scope principale chiamato \$rootScope, ma può avere svariati \$scope figli aggiunti in una struttura ad albero che rispecchia esattamente la struttura a "stati" delle view del sito. \\
Nei diagrammi dei package di questo documento lo \$scope è rappresentato simbolicamente come classe esterna posta tra la view e il controller, mentre a livello implementativo viene invece definito all'interno del controller associato. Nella definizione dei controller quindi: 
\begin{itemize}
\item \textbf{tutti i gli attributi e i metodi pubblici dei controller vanno inseriti nello \$scope};
\item \textbf{tutti i gli attributi e i metodi privati dei controller appartengono al controller}.
\end{itemize}
Questo esempio mostra la dichiarazione di un metodo (privato) in un controller che accede ad attributi pubblici dello \$scope:
\color{blue}\begin{lstlisting}[frame=single]
var var_init = function () {
	$scope.emailState = "";
	$scope.passwordState = "";
};
\end{lstlisting}\color{black}
quest'altro esempio invece mostra la dichiarazione di un metodo (pubblico) dello \$scope all'interno del controller che accede al metodo privato del controller mostrato prima:
\color{blue}\begin{lstlisting}[frame=single]
$scope.initVariables = function () {
	var_init();
};
\end{lstlisting}\color{black}
Lo \$scope ha accesso completo al controller, mentre la view può solo utilizzare quello che è stato definito nello \$scope;
\item \textbf{\$rootScope} è lo \$scope "radice" da cui discendono tutti gli altri \$scope dell'applicazione;
\item \textbf{\$meteor} è un servizio creato da Urigo: Angular-Meteor$_G$ per permettere di accedere alle funzionalità di Meteor all'interno dei moduli di AngularJS;
\item \textbf{\$state} rappresenta lo stato, o la posizione, in cui l'utente si trova all'interno dell'applicazione. Viene fornito dal package esterno AngularUI Router per AngularJS, che vede il re-indirizzamento dell'utente all'interno dell'applicazione come lo spostamento attraverso una macchina a stati;
\item \textbf{\$stateProvider} è il servizio dentro cui vengono registrati gli stati dell'applicazione;
\item \textbf{\$stateParams} è una lista di parametri passati dallo stato precedente a quello corrente;
\item \textbf{\$location} rende l'URL nella barra degli indirizzi del browser disponibile all'applicazione: modifiche all'URL nella barra degl indirizzi si riflettono dentro \$location e viceversa;
\item \textbf{\$locationProvider} permette la configurazione del servizio \$locationProvider.

\end{itemize}

\subsection{Gestione degli Account Utente}
Il Framework MeteorJS$_G$ si occupa della gestione dell'account degli utenti che utilizzano l'applicazione. Le informazioni dell'utente sono consultabili attraverso la variabile \texttt{currentUser}  in \$rootScope, mentre all'interno dei metodi del server che pubblicano le informazioni all'utente è necessario usare \texttt{this.userid}.
