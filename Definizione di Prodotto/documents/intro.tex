\section{Introduzione}
\subsection{Scopo del documento}
Questo documento approfondisce la definizione della struttura e dei componenti di Premi già discussa nella \ST{}. Ogni componente verrà descritto in modo sufficientemente dettagliato da consentire ai programmatori di sviluppare il software in modo coerente con
 quanto progettato finora.

\subsection{Scopo del Progetto}
Lo scopo del progetto è la realizzazione di un software di presentazione di slide non basato sul modello di PowerPoint$_{G}$, sviluppato in tecnologia HTML5$_{G}$ e che funzioni sia su desktop che su dispositivo mobile. Il software dovrà permettere la creazione da parte dell'autore e la successiva presentazione del lavoro, fornendo effetti grafici di supporto allo storytelling e alla creazione di mappe mentali. 

\subsection{Glossario}
Al fine di evitare ogni ambiguità relativa al linguaggio e ai termini utilizzati nei documenti formali tutti i termini e gli acronimi presenti nel seguente documento che necessitano di definizione saranno seguiti da una ”G” in pedice e saranno riportati in un documento esterno denominato \Glo{}.pdf. Tale documento accompagna e completa il presente e consiste in un listato ordinato di termini e acronimi con le rispettive definizioni e spiegazioni.

\subsection{Riferimenti}
\subsubsection{Normativi}
\begin{itemize}
	\item \textbf{Norme di Progetto:} \textit{\NdP};
	\item \textbf{Specifica Tecnica:} \textit{\ST}
	\item \textbf{Capitolato d'appalto C4:} Premi: Software di presentazione ``better than Prezi'' - \href{http://www.math.unipd.it/~tullio/IS-1/2014/Progetto/C4.pdf}{http://www.math.unipd.it/$\sim$tullio/IS-1/2014/Progetto/C4.pdf}.
\end{itemize}
\subsubsection{Informativi}
\begin{itemize}
	\item \textbf{Slide dell'insegnamento Ingegneria del Software modulo A}:\\ \href{http://www.math.unipd.it/~tullio/IS-1/2014/}{http://www.math.unipd.it/$\sim$tullio/IS-1/2014/};
	\item \textbf{Slide dell'insegnamento Ingegneria del Software modulo B}:\\ \href{http://www.math.unipd.it/~rcardin/sweb.html}{http://www.math.unipd.it/$\sim$rcardin/sweb.html};
	\item \textbf{Guida a Urigo: Angular-Meteor}: \\
\href{https://github.com/Urigo/angular-meteor}{https://github.com/Urigo/angular-meteor};
	\item \textbf{Ingegneria del software - Ian Sommerville - 8a Edizione (2007)}:
		\begin{itemize}
		\item[-] Part 4: Development;
		\end{itemize}
\end{itemize}