\section{Standard di Progetto}
\subsection{Standard di progettazione architetturale}
I diagrammi inseriti in questo documento seguono lo standard UML$_G$ 2.x. Per ulteriori informazioni sugli standard utilizzati si rimanda ai documenti \ST{} e \NdP{}.
\subsection{Standard di documentazione del codice}
Per gli standard di documentazione del codice consultare la sezione apposita delle \NdP{}.
\subsection{Standard di denominazione di entità e relazioni}
Ogni package, classe, metodo, attributo, template o semplice file di codice deve avere un nome chiaro ed esplicito che rappresenti la funzione che esso svolge all'interno del software. \\
In particolare:
\begin{itemize}
	\item i package dei componenti principali del client devono avere le loro classi interne suddivise in tre ulteriori package interni: \texttt{views} per i template delle viste, \texttt{controllers} per i controllers, e \texttt{lib} per le classi che fungono da modello per i dati dell'applicazione
	\item componenti marcati \texttt{template} sono denominati con uno o due termini che indicano la loro funzione, seguiti da \texttt{.ng}
	\item componenti marcati \texttt{controller} hanno il nome del template ad essi associato seguito da \texttt{Ctrl}
\end{itemize}
I termini scelti devono essere in lingua inglese, ed è preferibile non utilizzare abbreviazioni se la lunghezza del nome non risulta eccessiva. \\
Per ulteriori informazioni si rimanda al documento \NdP{}.
\subsection{Standard di programmazione}
Per gli standard di programmazione consultare il documento \NdP{} nella sezione apposita. 
\subsection{Strumenti di lavoro}
Gli strumenti di lavoro utilizzati sono descritti in dettaglio nel documento \NdP{}.