\section{Introduzione}
\subsection{Scopo del documento}
Lo scopo del presente documento è quello di fornire tutte le istruzioni necessarie per una corretta installazione del server ospitante il progetto \emph{Premi} e il suo corretto avvio.

\subsection{Scopo del prodotto}
Lo scopo del progetto è la realizzazione di un software di presentazione di slide non basato sul modello di PowerPoint$_{G}$, sviluppato in tecnologia HTML5$_{G}$ e che funzioni sia su desktop che su dispositivo mobile. Il software dovrà permettere la creazione da parte dell'autore e la successiva presentazione del lavoro, fornendo effetti grafici di supporto allo storytelling e alla creazione di mappe mentali.

\subsection{Come leggere il manuale}
Per una corretta installazione, sarà sufficiente seguire le istruzioni riportate nel presente documento in sequenza, seguendo l'ordine delle sezioni.

\subsection{Assistenza tecnica}
In caso di problemi durante l'installazione dell'applicazione, è possibile richiedere assistenza tecnica inviando un messaggio email all'indirizzo \href{mailto:404notfound.unipd@gmail.com}{404notfound.unipd@gmail.com} con la seguente forma:
\begin{itemize}
\item \textbf{OGGETTO}: richiesta\_assistenza
\item \textbf{TESTO}: descrizione dettagliata del problema riscontrato in fase di installazione
\item \textbf{ALLEGATO}: se disponibile, screenshot del problema riscontrato. Saranno considerati solo file immagine con estensione .png$_G$ o .jpg/.jpeg$_G$.
\end{itemize}
Un nostro tecnico provvederà a fornire una risposta nel più breve tempo possibile.

\subsection{Glossario}
Al fine di evitare ogni ambiguità relativa al linguaggio e ai termini utilizzati nei documenti formali tutti i termini e gli acronimi presenti nel seguente documento che necessitano di definizione saranno seguiti da una ``G'' in pedice e saranno riportati nella sezione apposita dell'appendice del presente documento. Tale sezione consiste in un listato ordinato di termini e acronimi con le rispettive definizioni e spiegazioni.

\subsection{Riferimenti}
\subsubsection{Informativi}
\begin{itemize}
	\item \textbf{W3C:} \href{http://www.w3c.it/it/1/ufficio-italiano-w3c.html}{http://www.w3c.it/it/1/ufficio-italiano-w3c.html};
	\item \textbf{Wikipedia:} \href{http://it.wikipedia.org/}{http://it.wikipedia.org/}; 
	\item \textbf{Wordreference:} \href{http://www.wordreference.com/}{http://www.wordreference.com/}.
\end{itemize}

\newpage