\newpage
\section{Appendice}

\subsection{Glossario}
\hfill\Huge{\textbf{B}}\\
\rule{16cm}{.6pt}
\normalsize
	\begin{longtable}{p{0.25\textwidth} p{0.75\textwidth}} 
	    \\
		    \textbf{Browser}: &	Programma che fornisce uno strumento per navigare in Internet e interagire con il World Wide Web.\\
	\end{longtable}

\hfill\Huge{\textbf{C}}\\
\rule{16cm}{.6pt}
\normalsize
	\begin{longtable}{p{0.25\textwidth} p{0.75\textwidth}} 
	    \\
		    \textbf{Chrome}: & Google Chrome (detto anche semplicemente Chrome) è un browser sviluppato da Google, basato, a partire dalla versione 28, sul motore di rendering Blink.\\
	\end{longtable}
	
	\hfill\Huge{\textbf{D}}\\
\rule{16cm}{.6pt}
\normalsize
	\begin{longtable}{p{0.25\textwidth} p{0.75\textwidth}} 
	    \\
		    \textbf{Debian}: & Debian è un sistema operativo per computer composto solo da software libero, anche se può usare, tramite l'aggiunta di appositi repository, anche software proprietario o software libero basato su software non libero. La forma principale, Debian GNU/Linux utilizza Linux come kernel (la parte centrale di un sistema operativo) e programmi di utilità provenienti dal progetto GNU. Per questo prende il nome di GNU/Linux.\\
	\end{longtable}
	
\hfill\Huge{\textbf{J}}\\
\rule{16cm}{.6pt}
\normalsize
	\begin{longtable}{p{0.25\textwidth} p{0.75\textwidth}} 
	    \\
		    \textbf{Javascript}: & E` un linguaggio di programmazione interpretato sviluppato da Netscape orientato agli oggetti e agli eventi.\\
		    \\
		    \textbf{JPEG - JPG}: & JPEG (acronimo di Joint Photographic Experts Group) è un comitato ISO/CCITT che ha definito il primo standard internazionale di compressione dell'immagine digitale a tono continuo, sia a livelli di grigio che a colori. JPEG indica quindi anche il diffusissimo formato di compressione a perdita di informazioni ed è un formato aperto e ad implementazione gratuita.
	\end{longtable}
	
\hfill\Huge{\textbf{M}}\\
\rule{16cm}{.6pt}
\normalsize
	\begin{longtable}{p{0.25\textwidth} p{0.75\textwidth}} 
	    \\
		    \textbf{MeteorJS}: & MeteorJS è una piattaforma completa per la costruzione di web e mobile app scritte in puro Javascript.\\
	\end{longtable}
	
\hfill\Huge{\textbf{O}}\\
\rule{16cm}{.6pt}
\normalsize
	\begin{longtable}{p{0.25\textwidth} p{0.75\textwidth}} 
	    \\
		    \textbf{Open Source}: & Open source, in informatica, indica un software di cui gli autori (più precisamente i detentori dei diritti) rendono pubblico il codice sorgente, favorendone il libero studio e permettendo a programmatori indipendenti di apportarvi modifiche. Questa possibilità è regolata tramite l'applicazione di apposite licenze d'uso.\\
	\end{longtable}
	
	\hfill\Huge{\textbf{P}}\\
\rule{16cm}{.6pt}
\normalsize
	\begin{longtable}{p{0.25\textwidth} p{0.75\textwidth}} 
	    \\
		    \textbf{PNG}: & In informatica, il Portable Network Graphics (abbreviato PNG) è un formato di file per memorizzare immagini.\\
		    \\
		    \textbf{Power Point}: & Microsoft Office PowerPoint è il programma di presentazione prodotto da Microsoft, fa parte della suite di software di produttività personale Microsoft Office, è tutelato da copyright e distribuito con licenza commerciale ed è disponibile per i sistemi operativi Windows e Macintosh. E` utilizzato principalmente per proiettare e quindi comunicare su schermo, progetti, idee, e contenuti potendo incorporare testo, immagini, grafici, filmati, audio e potendo presentare tutto questo con animazioni di alto livello.\\
	\end{longtable}
	
\hfill\Huge{\textbf{R}}\\
\rule{16cm}{.6pt}
\normalsize
	\begin{longtable}{p{0.25\textwidth} p{0.75\textwidth}} 
	    \\
		    \textbf{Repository}: & Repository è uno spazio di archiviazione da cui è possibile recuperare software o codice sorgente.\\
	\end{longtable}	
	
	\hfill\Huge{\textbf{S}}\\
\rule{16cm}{.6pt}
\normalsize
	\begin{longtable}{p{0.25\textwidth} p{0.75\textwidth}} 
	    \\
		    \textbf{Screen reader}: & Uno screen reader (letteralmente lettore dello schermo) è un'applicazione software che identifica ed interpreta il testo mostrato sullo schermo di un computer, presentandolo tramite sintesi vocale o attraverso un display braille.\\
		    \\
		    \textbf{Server}: & In informatica il termine server, indica genericamente un componente o sottosistema informatico di elaborazione che fornisce, a livello logico e a livello fisico, un qualunque tipo di servizio ad altre componenti che ne fanno richiesta attraverso una rete di computer, all'interno di un sistema informatico o direttamente in locale su un computer. Al termine server, così come per il termine client, possono dunque riferirsi sia la componente hardware che la componente software che forniscono le funzionalità o servizi.\\
	\end{longtable}
	
\hfill\Huge{\textbf{U}}\\
\rule{16cm}{.6pt}
\normalsize
	\begin{longtable}{p{0.25\textwidth} p{0.75\textwidth}}
	\textbf{Ubuntu}: & Ubuntu è una distribuzione GNU/Linux, basata su Debian, nata nel 2004. La sua principale caratteristica è la focalizzazione sull'utente e la facilità di utilizzo. Essa viene pubblicata come software libero sotto licenza GNU GPL, è distribuita gratuitamente ed è liberamente modificabile. Ubuntu è orientata all'utilizzo desktop e pone una grande attenzione al supporto hardware.\\
	    \\
		    \textbf{URL}: & La locuzione Uniform Resource Locator (in acronimo URL), nella terminologia delle telecomunicazioni e dell'informatica è una sequenza di caratteri che identifica univocamente l'indirizzo di una risorsa in Internet, tipicamente presente su un host server, come ad esempio un documento, un'immagine, un video, rendendola accessibile ad un client che ne faccia richiesta attraverso l'utilizzo di un web browser.\\
	\end{longtable}

	
	
	
	
	
	
	
	
	
	
	
	
	
	

