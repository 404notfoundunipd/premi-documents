\section{Esempio citazioni}
Una citazione in display è
un testo che \LaTeX{} compone
su linee a sè:
\begin{quote}\begin{footnotesize} 
 
	Il cielo stellato sopra di me,\\
	la legge morale dentro di me.
	\flushright
		--- Autore Citazione, fonte.
 
\end{footnotesize} 
\end{quote}
Come si può osservare, la
citazione è centrata e
separata dal resto del testo.

\section{Esempio tabelle}
% Ogni spazio diviso da | rappresenta una colonna
% |l| -> allineamento cella a sinistra
% |c| -> allineamento cella al centro
% |r| -> allineamento cella a destra
\begin{table}[h]
\begin{tabular}{|l|l|l|}
\hline
\textbf{Prima colonna} & \textbf{seconda colonna} & \textbf{terza colonna} \\
\hline %disegna una linea orrizzontale
Primo & Secondo & Terzo \\
\hline
Quarto & Quinto & Sesto \\
\hline
\end{tabular}
\caption{Didascalia della tabella.}
\end{table}

\section{Esempio codice}
\subsection{Java}
\captionof{lstlisting}{Esempio codice Java}
\begin{lstlisting}[language=Java]
	/**
	* Esempio di codice Java
	*/
	public static void main(String[] args){
		System.out.println("Hello world!");
	}
\end{lstlisting}

\subsection{C++}
\captionof{lstlisting}{Esempio codice C++}
\begin{lstlisting}[language=C++]
	/**
	* Esempio di codice C++
	*/
	int main(String[] args){
		cout<<"Hello world!";
	}
\end{lstlisting}

\section{Esempio di link}
\href{http://www.google.com/}{Google Inc.}