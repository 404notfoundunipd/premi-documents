\section{Introduzione}

\subsection{Scopo del documento}
In seguito all'analisi del capitolato d'appalto e all'incontro con il proponente è emerso un insieme di requisiti, i quali verranno elencati e descritti in modo dettagliato nel presente documento.
Il seguente documento ha quindi lo scopo di evidenziare le funzionalità che il prodotto offrirà.

\subsection{Scopo del Progetto}
Lo scopo del progetto è la realizzazione di un software di presentazione di slide non basato sul modello di PowerPoint$_{G}$, sviluppato in tecnologia HTML5$_{G}$ e che funzioni sia su desktop che su dispositivo mobile. Il software dovrà permettere la creazione da parte dell'autore e la successiva presentazione del lavoro, fornendo effetti grafici di supporto allo storytelling e alla creazione di mappe mentali. 

\subsection{Glossario}
Al fine di evitare ogni ambiguità relativa al linguaggio e ai termini utilizzati nei documenti formali tutti i termini e gli acronimi presenti nel seguente documento che necessitano di definizione saranno seguiti da una ”G” in pedice e saranno riportati in un documento esterno denominato Glossario.pdf. Tale documento accompagna e completa il presente e consiste in un listato ordinato di termini e acronimi con le rispettive definizioni e spiegazioni.

\subsection{Riferimenti}
\subsubsection{Normativi}
\begin{itemize}
	\item \textbf{Norme di Progetto:} NormeDiProgetto\_v1.0.pdf;
	\item \textbf{Capitolato d’appalto C4:} Premi: Software di presentazione ``better than Prezi''\\ \href{http://www.math.unipd.it/~tullio/IS-1/2014/Progetto/C4.pdf}{http://www.math.unipd.it/$\sim$tullio/IS-1/2014/Progetto/C4.pdf};
	\item \textbf{Verbare esterno:} \\ 
		Verbale derivato dall'incontro con il Proponente in data 2014-12-18 (Verbale2014-12-18\_v1.0.pdf).
\end{itemize}

\subsubsection{Informativi}
\begin{itemize}
	\item Materiale didattico Ingegneria del Software: \\
	\href{http://www.math.unipd.it/~tullio/IS-1/2014}{http://www.math.unipd.it/$\sim$tullio/IS-1/2014};
	
	\item Prezi:
	\href{https://prezi.com}{https://prezi.com}.
\end{itemize}
