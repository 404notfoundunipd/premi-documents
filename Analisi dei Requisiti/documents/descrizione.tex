\section{Descrizione generale}

\subsection{Contesto d'uso del prodotto}
\subsubsection{Processi produttivi e modalità d'uso}
Il progetto \textbf{Premi} vuole fornire la possibilità di creare presentazioni tramite l'utilizzo di un browser$_G$ sfruttando le potenzialità delle tecnologie HTML5$_G$ e Javascript$_G$. \\
L'utente avrà quindi la possibilità di creare e visualizzare le proprie presentazioni tramite un browser$_G$, mentre, nell'ambito mobile, potrà solo visualizzare le proprie presentazioni.

\subsubsection{Ambiente di utilizzo}
Essendo un progetto sviluppato come una web app, l'ambiente di utilizzo è un qualsiasi sistema operativo dotato di browser$_G$ visuale. Non ci sono quindi vincoli legati all'architettura hardware utilizzata e al sistema operativo che utilizza.


\subsection{Funzioni del prodotto}
Il prodotto darà la possibilità di:
\begin{itemize}
	\item Creare e visualizzare presentazioni;
	\item condividere presentazioni;
	\item salvare le presentazioni all'interno del proprio account;
	\item stampare le presentazioni come poster o sequenza di frame;
	\item avviare e partecipare a presentazioni live.
\end{itemize}

\subsection{Caratteristiche degli utenti}
Il prodotto è rivolto a tutti gli utenti che utilizzano un browser$_G$ visuale e, come è stato messo in luce durante l' incontro con il proponente (vedi documento allegato \textit{Verbale2014-12-18.pdf} ), non è stata posta particolare attenzione alle categorie di utenti che fanno uso dello screenreader$_G$. In particolare il software verrà utilizzato da chiunque abbia la necessità di creare una presentazione, e cerchi uno strumento alternativo al noto e diffuso PowerPoint$_G$.

\subsection{Vincoli generali}
I vincoli generali individuati durante lo studio del capitolato sono:
\begin{itemize}
\item Browser$_G$ che supporti la tecnologia HTML5$_G$;
\item Connessione ad Internet;
\item Javascript abilitato sul proprio browser$_G	$.
\end{itemize}

