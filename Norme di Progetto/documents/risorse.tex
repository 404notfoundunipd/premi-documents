\section{Risorse} \label{risorse}
La realizzazione del progetto richiede l'utilizzo di alcune risorse suddivisibili in diverse categorie.
\subsection{Risorse Necessarie} 
\subsubsection{Risorse Umane}
\begin{itemize}
	\item[-] \textbf{\ruoloResponsabile}: E' responsabile nei confronti del committente della corretta realizzazione del prodotto;
	\item[-] \textbf{\ruoloAnalista}: E' colui che si occupa dell'individuazione dei requisiti, impliciti ed espliciti del progetto.
	\item[-] \textbf{\ruoloProgettista}: Ha il compito di trovare una soluzione possibile per lo sviluppo.
	\item[-] \textbf{\ruoloVerificatore}: Coordina e svolge le attività di verifica vere e proprie;
	\item[-] \textbf{\ruoloProgrammatore}: Codifica ed esegue attività di debugging sul codice.
	\item[-] \textbf{\ruoloAmministratore}: Ha il compito di gestire e garantire le risorse e l'ambiente di sviluppo e di lavoro all'interno del gruppo.
\end{itemize}
Per una descrizione dettagliata e più completa delle figure elencate si rimanda al documento allegato \PdP .

\paragraph{Rotazione dei ruoli\\}
Da regolamento\cite{docorganigramma} ogni componente del gruppo può occupare più di un ruolo nelle varie fasi del progetto, garantendo però l'assenza di conflitto di interessi tra i ruoli assunti. Questo richiede che le seguenti norme vengano attuate nell'assegnazione dei ruoli:
\begin{itemize}
\item rotazione del ruolo di \ruoloResponsabile{} ad ogni macro-fase per non lasciare in mano allo stesso componente il coordinamento delle attività per tutto l'arco temporale del progetto;
\item la verifica di un documento o di parte del codice dev'essere sempre eseguita da componenti che non hanno scritto quel documento o codificato quella parte di software.
\end{itemize}
Per facilitare lo sviluppo del software vengono inoltre suggerite le seguenti raccomandazioni:
\begin{itemize}
\item separazione del ruolo di \ruoloResponsabile{} nei suoi incarichi di redattore di documenti (che potranno essere scritti da più componenti) e di coordinatore di attività (insieme di incarichi che dovranno essere svolti dal Responsabile assegnato per la macro-fase corrente);
\item specializzazione di componenti del gruppo, quando ritenuto opportuno, nella stesura di determinati documenti per una maggiore efficienza nella loro elaborazione e verifiche incrociate su tali documenti per assicurare che ognuno li conosca in dettaglio.
\end{itemize}

\subsubsection{Risorse Software}
Durante la fase di realizzazione del progetto saranno necessari:
\begin{itemize}
	\item[-] Software per la gestione di documenti in \LaTeX;
	\item[-] Piattaforma di testing sui vari browser$_G$ dell'applicazione da sviluppare;
	\item[-] Piattaforma di versionamento per la creazione e la gestione di ticket$_G$;
	\item[-] Software per la creazione dei diagrammi in UML$_G$;
	\item[-] Ambiente per lo sviluppo del codice nel linguaggio di programmazione scelto;
	\item[-] Strumenti di validazione del codice prodotto.
\end{itemize}

\subsubsection{Risorse Hardware}
\begin{itemize}
	\item[-] Computer dotati di tutti gli strumenti software descritti nel \PdQ{} e nelle \NdP ;
	\item[-] Luogo fisico in cui incontrarsi per lo sviluppo del progetto, possibilmente con una connessione ad Internet.
\end{itemize}

\subsection{Risorse Disponibili}
\subsubsection{Risorse Software}
Vengono di seguito elencate le risorse software disponibili. Per una descrizione più dettagliata si rimanda alla sottosezione Strumenti.
\begin{itemize}
	\item[-] TeXMaker per l'editing dei documenti in \LaTeX;
	\item[-] BrowserStack per il testing sui vari browser$_G$;
	\item[-] GitHub per il versionamento e la gestione dei ticket$_G$;
	\item[-] Astah per i diagrammi UML$_G$;
	\item[-] WebStorm come ambiente di sviluppo;
	\item[-] Strumenti di validazione online del W3C$_G$.
\end{itemize}

\subsubsection{Risorse Hardware}
\begin{itemize}
	\item[-] Computer personali (portatili o fissi) dei membri del gruppo; 
	\item[-] Computer messi a disposizione nei laboratori informatici del Dipartimento di Matematica Pura ed Applicata dell'Università di Padova;
	\item[-] Aule studio del Dipartimento di Matematica Pura ed Applicata
	dell'Università di Padova. 
\end{itemize}