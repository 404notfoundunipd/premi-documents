\section{Collaborazione}
\subsection{Comunicazioni}
\subsubsection{Comunicazioni esterne}
Le comunicazioni esterne avvengono obbligatoriamente tramite l’utilizzo della mail associata al gruppo
\begin{center} \href{mailto:404notfound.unipd@gmail.com}{404notfound.unipd@gmail.com}
\end{center}
tale indirizzo sarà l’unico utilizzabile per tali comunicazioni; non sono ammesse comunicazioni individuali verso l’esterno.
La suddetta casella postale sarà utilizzabile solamente dal \ruoloResponsabile, il quale avrà quindi l’onere di comunicare con il committente del progetto.

\subsubsection{Comunicazioni interne}
Le comunicazioni interne formali avvengono attraverso l’utilizzo di una mailing list$_G$ creata all'interno della rubrica di Gmail$_G$. Al momento della composizione andrà indicato come destinatario della mail il nome della mailing list$_G$ creata.
Con comunicazioni formali si intende:
\begin{itemize}
	\item Comunicazione riunione interna;
	\item Comunicazione riunione esterna;
	\item Diffusione verbale ultima riunione.
\end{itemize}

Per altri tipi di comunicazione vengono utilizzati servizi di instant messaging quali Whatsapp$_G$ o Skype$_G$ e un gruppo su Facebook$_G$ appositamente creato per lo scambio di informazioni utili varie ed eventuali.

\subsubsection{Compisizione e-mail}
In questo paragrafo verranno illustrate le norme secondo le quali comporre le e-mail, in riferimento sia a comunicazioni esterne che interne.

\paragraph{Destinatario}
Nel caso di più destinatari è possibile utilizzare le e-mail in copia utilizzando il campo ``Cc''. Vietato l'uso del campo di copia nascosta ``Ccn''.
\begin{itemize}
	\item \textbf{Mail interne:} il nome della mailing list creata;
	\item \textbf{Mail esterne:} il destinatario potrà essere il committente, il \Vardanega\ e il \Cardin.
\end{itemize}

\paragraph{Mittente}
\begin{itemize}
	\item \textbf{Mail interne:} il mittente sarà l'indirizzo del gruppo \href{mailto:404notfound.unipd@gmail.com}{404notfound.unipd@gmail.com};
	\item \textbf{Mail esterne:} dovranno sempre avere come mittente l'indirizzo del gruppo di lavoro \href{mailto:404notfound.unipd@gmail.com}{404notfound.unipd@gmail.com}.
\end{itemize}

\paragraph{Oggetto}
L’oggetto deve essere chiaro e conciso e non modificato una volta avviata una comunicazione.
Nel caso di risposta va preceduto da ``Re:'', mentre per un inoltro va utilizzato ``I:'';

\paragraph{Corpo}
Il corpo della mail dovrà essere scritto chiaramente e dovrà contenere tutte le informazioni necessarie al fine di rendere comprensibile l'argomento trattato.
Nel caso di risposta va mantenuto il corpo delle mail precedenti in modo da creare uno storico della conversazione.

\paragraph{Firma}
In fondo al corpo della mail andrà indicata la firma ``404NotFound''.

\paragraph{Allegati}
Sono permessi allegati solo nel caso in cui fossero strettamente necessari, ad esempio per divulgare il verbale di una riunione. I formati permessi sono PDF$_G$, JPEG$_G$, PNG$_G$, SVG$_G$.






\subsection{Riunioni}
Le riunioni del gruppo di lavoro dovranno avere una frequenza almeno quindicinale.
\subsubsection{Riunioni interne}
Il \ruoloResponsabile\ ha il compito di convocare le riunioni generali, ovvero le riunioni in cui vengono convocati tutti i membri del gruppo, ed eventualmente di valutare se anticipare la naturale scadenza della riunione successiva. Il \ruoloResponsabile\ deve convocare l’assemblea con almeno tre giorni di preavviso attraverso il canale di comunicazione interna (vedi sezione 2.2). \\ \\
Ciascun componente del gruppo può avanzare una richiesta di riunione interna.
Tale richiesta deve pervenire al \ruoloResponsabile\ che provvederà a valutare le motivazioni espresse dal richiedente e ad approvare e quindi indire la riunione. \\ \\
È inoltre possibile e auspicabile che possano essere necessarie riunioni tra specifici membri del gruppo senza richiedere la presenza di persone non direttamente coinvolte nel lavoro in questione, che verranno comunque informate delle decisioni prese tramite verbale inviato mediante posta elettronica nella mail ufficiale del gruppo.

\subsubsection{Riunioni Esterne}
Con riunioni esterne si intende qualsiasi incontro fra Proponente/Committente
e un gruppo di rappresentanza, composto da almeno la maggioranza assoluta del
gruppo di progetto, altrimenti detto \textit{numero legale}.\\
\\
La richiesta di indire una riunione esterna può essere avanzata da qualsiasi componente del gruppo; è compito del \ruoloResponsabile\ approvare ed eventualmente organizzare l'evento. Una volta che è stata approvata una richiesta, il \ruoloResponsabile\ dovrà contattare, con le modalità scritte nella sezione 2.2, il Proponente/Committente.
\\\\
Ad ogni incontro verrà incaricato dal \ruoloResponsabile\ un membro del gruppo con il compito di stilare un verbale della riunione, spedito in copia poi a tutti i membri del gruppo per presa visione, verifica e conferma di quanto riportato.






\subsection{Metodi di Condivisione}
Qui vengono elencate le piattaforme adottate per facilitare lo svolgimento collettivo del progetto.

\subsubsection{Google Calendar}
Google Calendar$_G$ viene utilizzato come promemoria per le date degli incontri e per segnalare le fasce orarie in cui un membro non è disponibile in caso di impegni settimanali inderogabili.

\subsubsection{Google Drive}
Google Drive$_G$ permette una facile condivisione di documenti informali e di tutte quelle risorse utili al progetto che non sono soggette a versionamento. \\
Per sfruttare nel modo migliore le potenzialità della piattaforma si raccomanda di attenersi alla suddivisione intuitiva in cartelle dei file e di utilizzare, quando possibile, formati compatibili con gli editor$_G$ testuali del sito.

\subsubsection{Mega}
Per la condivisione delle esportazioni delle Virtual Machine$_G$ con Ubuntu Desktop$_G$ è stato scelto il servizio cloud$_G$ Mega, che dà a disposizione 50GB di spazio gratuito.

\subsubsection{GitHub}
GitHub$_G$ è un servizio di hosting di repository$_G$ Git$_G$. \\
Una repository$_G$ è uno spazio di archiviazione da cui è possibile recuperare software o codice sorgente; il sistema di controllo di versione$_G$ Git$_G$ in particolare sfrutta la tecnica dei branch$_G$, ossia di ramificazioni del codice sorgente su cui poter lavorare senza  entrare in conflitto con il lavoro di altri collaboratori. Il branch originale viene chiamato \textit{master}. \\
Sono state create due repository$_G$, accessibili dall'indirizzo:\\ 
\url{https://github.com/404notfoundunipd/}
\begin{itemize}
\item \texttt{premi} contiene i file di codifica e di sviluppo dell'applicazione eseguibile;
\item \texttt{premi-documents} contiene i file di tutti i documenti formali soggetti a versionamento compresi i template$_G$ per la loro stesura.
\end{itemize}
