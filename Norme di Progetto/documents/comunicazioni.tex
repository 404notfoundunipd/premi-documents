\section{Comunicazioni}
\subsection{Comunicazioni esterne}
Le comunicazioni esterne avvengono obbligatoriamente tramite l’utilizzo della mail associata al gruppo
\begin{center} \href{mailto:404notfound.unipd@gmail.com}{404notfound.unipd@gmail.com}
\end{center}
tale indirizzo sarà l’unico utilizzabile per tali comunicazioni; non sono ammesse comunicazioni individuali verso l’esterno.
La suddetta casella postale sarà utilizzabile solamente dal \ruoloResponsabile, il quale avrà quindi l’onere di comunicare con il committente del progetto.

\subsection{Comunicazioni interne}
Le comunicazioni interne formali avvengono attraverso l’utilizzo di una mailing list$_{G}$ creata all'interno della rubrica di Gmail$_{G}$. Al momento della composizione andrà indicato come destinatario della mail il nome della mailing list$_{G}$ creata.
Con comunicazioni formali si intende:
\begin{itemize}
	\item Comunicazione riunione interna
	\item Comunicazione riunione esterna
	\item Diffusione verbale ultima riunione
\end{itemize}

Per altri tipi di comunicazione vengono utilizzati servizi di instant messaging quali Whatsapp$_{G}$ o Skype$_{G}$ e un gruppo su Facebook$_{G}$ appositamente creato per lo scambio di informazioni utili varie ed eventuali.

\subsection{Compisizione e-mail}
In questo paragrafo verranno illustrate le norme secondo le quali comporre le e-mail, in riferimento sia a comunicazioni esterne che interne.

\subsubsection{Destinatario}
Nel caso di più destinatari è possibile utilizzare le e-mail in copia utilizzando il campo ``Cc''. Vietato l'uso del campo di copia nascosta ``Ccn''.
\begin{itemize}
	\item \textbf{Mail interne:} il nome della mailing list creata;
	\item \textbf{Mail esterne:} il destinatario potrà essere il committente, il \Vardanega\ e il \Cardin.
\end{itemize}

\subsubsection{Mittente}
\begin{itemize}
	\item \textbf{Mail interne:} il mittente sarà l'indirizzo del gruppo \href{mailto:404notfound.unipd@gmail.com}{404notfound.unipd@gmail.com};
	\item \textbf{Mail esterne:} dovranno sempre avere come mittente l'indirizzo del gruppo di lavoro \href{mailto:404notfound.unipd@gmail.com}{404notfound.unipd@gmail.com}.
\end{itemize}

\subsubsection{Oggetto}
L’oggetto deve essere chiaro e conciso e non modificato una volta avviata una comunicazione.
Nel caso di risposta va preceduto da ``Re:'', mentre per un inoltro va utilizzato ``I:'';

\subsubsection{Corpo}
Il corpo della mail dovrà essere scritto chiaramente e dovrà contenere tutte le informazioni necessarie al fine di rendere comprensibile l'argomento trattato.
Nel caso di risposta va mantenuto il corpo delle mail precedenti in modo da creare uno storico della conversazione.

\subsubsection{Firma}
In fondo al corpo della mail andrà indicata la firma ``404NotFound''.

\subsubsection{Allegati}
Sono permessi allegati solo nel caso in cui fossero strettamente necessari, ad esempio per divulgare il verbale di una riunione. I formati permessi sono PDF$_{G}$, JPEG$_{G}$, PNG$_{G}$, SVG$_{G}$.

\newpage