\section{Comunicazioni}
\subsection{Comunicazioni interne}
Le comunicazioni interne formali avvengo attraverso l’utilizzo di una mailing list creata appositamente: 
\\
\begin{center}
	\href{mailto:indirizzo@mailinglist\_interna.com}{indirizzo@mailinglist\_interna.com}
\end{center}
Con comunicazioni formali si intende:
\begin{itemize}
	\item Comunicazione riunione interna
	\item Comunicazione riunione esterna
	\item Diffusione verbale ultima riunione
\end{itemize}

Per altri tipi di comunicazione vengono utilizzati servizi di messaggistica quali Whatsapp o Skype e un gruppo facebook appositamente creato per lo scambio di inforamazioni utili varie ed eventuali.

\subsection{Comunicazioni esterne}
Le comunicazioni esterne avvengono obbligatoriamente tramite l’utilizzo della mail associata al gruppo \href{404notfound.unipd@gmail.com}{404notfound.unipd@gmail.com}, tale indirizzo sarà l’unico utilizzabile per tali comunicazioni; non sono ammesse comunicazioni individuali verso l’esterno.
La suddetta casella postale sarà utilizzabile solamente dal \ruoloResponsabile, il quale avrà quindi l’onere di comunicare con il committente del progetto.

\subsection{Compisizione e-mail}
In questo paragrafo verranno illustrate le norme seconde le quali verranno composte le e-mail, in riferimento sia a comunicazioni esterne che interne.

\subsubsection{Mail interne}
\begin{itemize}
	\item \textbf{Destinatario:} \href{mailto:indirizzo@mailinglist\_interna.com}{indirizzo@mailinglist\_interna.com}, unico indirizzo utilizzabile per e-mail interne;
	\item \textbf{Mittente:} indirizzo personale del \ruoloResponsabile
	\item \textbf{Oggetto:} 
\end{itemize}




