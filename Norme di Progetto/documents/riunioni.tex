\section{Riunioni}
\subsection{Frequenza}
Le riunioni del gruppo di lavoro dovranno avere una frequenza almeno quindicinale.
\subsection{Convocazione Riunioni}
\subsubsection{Riunioni interne}
Il \ruoloResponsabile\ ha il compito di convocare le riunioni generali, ovvero le riunioni in cui vengono convocati tutti i membri del gruppo, ed eventualmente di valutare se anticipare la naturale scadenza della riunione successiva. Il \ruoloResponsabile\ deve convocare l’assemblea con almeno tre giorni di preavviso attraverso il canale di comunicazione interna (vedi sezione 2.2). \\ \\
Ciascun componente del gruppo può avanzare una richiesta di riunione interna.
Tale richiesta deve pervenire al \ruoloResponsabile\ che provvederà a valutare le motivazioni espresse dal richiedente e ad approvare e quindi indire la riunione. \\ \\
È inoltre possibile e auspicabile che possano essere necessarie riunioni tra specifici membri del gruppo senza richiedere la presenza di persone non direttamente coinvolte nel lavoro in questione, che verranno comunque informate delle decisioni prese tramite verbale inviato mediante posta elettronica nella mail ufficiale del gruppo.

\subsubsection{Riunioni Esterne}
Con riunioni esterne si intende qualsiasi incontro fra Proponente/Committente
e un gruppo di rappresentanza, composto da almeno la maggioranza assoluta del
gruppo di progetto, altrimenti detto \textit{numero legale}.\\
\\
La richiesta di indire una riunione esterna può essere avanzata da qualsiasi componente del gruppo; è compito del \ruoloResponsabile approvare ed eventualmente organizzare l'evento. Una volta che è stata approvata una richiesta, il \ruoloResponsabile\ dovrà contattare, con le modalità scritte nella sezione 2.2, il Proponente/Committente.
\\\\
Ad ogni incontro verrà incaricato dal \ruoloResponsabile\ un membro del gruppo con il compito di stilare un verbale della riunione, spedito in copia poi a tutti i membri del gruppo per presa visione, verifica e conferma di quanto riportato.