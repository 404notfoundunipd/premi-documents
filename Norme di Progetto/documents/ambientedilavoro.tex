\section{Ambiente di Lavoro} \label{ambientedilavoro}
\subsection{Coordinamento}
\subsubsection{Software di versionamento}
E' stato scelto di utilizzare GitHub$_G$ come software di versionamento. I motivi principali della scelta sono:
\begin{itemize}
	\item \textbf{Flessibilità:} Git$_G$ è un repository$_G$ distribuito con la possibilità di commit$_G$ e revert locali;
	\item \textbf{Esperienza del team:} Git$_G$ è già stato usato da alcuni componenti del gruppo.
\end{itemize}
Inoltre Git$_G$ mette a disposizione un'interfaccia grafica (SmartGit$_G$) e un sistema di integrazione con WebStorm.
Nella sezione 5.7 del documento corrente sono specificate alcune procedure di utilizzo del repository$_G$.

\subsubsection{Condivisione dei file}

\paragraph{6.1.2.1 Mega \\}
Per la condivisione delle esportazioni delle Virtual Machine$_G$ con Ubuntu Desktop$_G$ è stato scelto il servizio cloud$_G$ Mega, che dà a disposizione 50GB di spazio gratuito.

\paragraph{6.1.2.2 Google Drive\\}
Google Drive$_G$ permette una facile condivisione di documenti informali e di tutte quelle risorse utili al progetto che non sono soggette a versionamento. \\
Per sfruttare nel modo migliore le potenzialità della piattaforma si raccomanda di attenersi alla suddivisione intuitiva in cartelle dei file e di utilizzare, quando possibile, formati compatibili con gli editor$_G$ testuali del sito.

\subsubsection{Google Calendar}
Google Calendar$_G$ viene utilizzato come promemoria per le date degli incontri e per segnalare le fasce orarie in cui un membro non è disponibile in caso di impegni settimanali inderogabili.

\subsubsection{Mozilla Thunderbird}
	E' stato scelto Mozilla Thunderbird$_G$ come client di posta, in quanto è uno strumento utilizzato quotidianamente da molti componenti del gruppo, e inoltre è già presente nell'installazione base di Ubuntu desktop. Questo client, permette l'installazione di un plugin a noi necessario: Lightning$_G$. Quest'ultimo verrà utilizzato per sincronizzare i calendari di Google Calendar$_G$.

\subsection{Pianificazione}
Per pianificare le attività legate allo sviluppo del progetto e la gestione delle risorse si è scelto di utilizzare \textbf{ProjectLibre}.
ProjectLibre è un programma open source per il project management basato su
Java$_G$ e quindi eseguibile su ogni sistema operativo. Le principali caratteristiche di tale software sono:
\begin{itemize}
	\item Portabilità, essendo basto su Java$_G$;
	\item Open-source;
	\item Genera diagrammi Gantt$_G$;
	\item Genera automaticamente diagrammi Program Evaluation and Review Technique(PERT$_G$), a partire dal Gantt$_G$;
	\item Salvataggio su file XML$_G$: essendo un file testuale è possibile effettuare il merge$_G$ di più file in caso di conflitti sul repository$_G$.

\end{itemize}

\subsubsection{Modalità di utilizzo}
Il \ruoloResponsabile{} ha il compito di creare un progetto per ogni macro-fase indicata nella sezione Pianificazione delle attività del \PdP{} e procedere nel modo seguente:

\begin{enumerate}
	\item Creare un calendario lavorativo per il progetto;
	\item Inserire le attività da svolgere e le corrispondenti sotto-attività;
	\item Inserire le dipendenze temporali tra le attività;
	\item Inserire i periodi di slack dove previsto;
	\item Inserire la milestoneG per indicare il termine previsto delle attività;
	\item Creare le risorse e suddividerle adeguatamente tra le attività definite.
\end{enumerate}

\subsection{Controllo e rendicontazione}
Il sistema di Milestone$_G$ e ticketing descritto nella sezione 5.7 del documento corrente, permette un'organizzazione del lavoro in macroattività, a loro volta suddivise in attività più circoscritte ed assegnate ai componenti del gruppo, a discrezione del \ruoloResponsabile.
Per mantenere sotto controllo lo stato delle suddette attività ed avere in ogni momento una visione generale della situazione il \ruoloResponsabile{} utilizzerà \textbf{ProjectLibre}, uno strumento software che permette una gestione facilitata della pianificazione e una rendicontazione dei periodi temporali spesi nelle varie attività;

\subsubsection{Controllo attività}
Il sistema di project management adottato, permette di visualizzare in modo dinamico l'andamento generale delle attività attraverso il diagramma di Gannt$_G$, che fornisce le seguenti informazioni:
\begin{itemize}
\item La percentuale di completamento delle attività aperte;
\item Le attività in ritardo;
\item Le attività concluse.
\end{itemize}

\subsubsection{Controllo date}
Per ottimizzare la pianificazione e tenerla in costante aggiornamento si utilizzano dei calendari a disposizione del gruppo:
\begin{itemize}
\item \textbf{Calendario attività:} Il sistema di project management adottato, genera automaticamente un calendario in cui vengono indicate inizio e fine delle varie attività e la risorsa (componente del gruppo) a cui sono state assegnate;
\item \textbf{Calendario risorse:} Il calendario a disposizione del gruppo, descritto nella sezione 6.1, sarà utilizzato dal \ruoloResponsabile{} per gestire il personale e distribuire le attività in base agli impegni dei vari componenti, agli impegni del gruppo nell'ambito del progetto e alle scadenze da rispettare.
\end{itemize}

\subsubsection{Rendicontazione}
Per rendicontare le ore dedicate alle varie attività si è deciso di utilizzare il foglio elettronico di Libreoffice. Tale strumento, tramite l'utilizzo di tabelle, permette di visualizzare le ore di lavoro in base all'attività e al ruolo svolti.

\subsection{Strumenti per la documentazione}
\subsubsection{Stesura documenti}
Per la stesura dei documenti si è scelto di usare il linguaggio di markup$_G$ \LaTeX.
Il motivo principale che ha portato a questa scelta è la facilità di separazione tra contenuto e formattazione: con \LaTeX è possibile definire l’aspetto delle pagine in un file template condiviso da tutti i documenti, cosa che non sarebbe stata possibile optando per altre soluzioni come Microsoft Office o LibreOffice. Inoltre permette la definizione di funzioni e variabili globali garantendo una formattazione del testo uniforme a tutti i documenti e la scrittura del contenuto più corretta da un punto di vista semantico.
Il team \gruppo\ si avvarrà del software \textit{TexMaker} versione 4.4.1 per la redazione, la compilazione e l'esportazione dei file in \LaTeX;

\subsubsection{Verifica}
Per il processo di verifica dei documenti si è scelto di utilizzare i seguenti strumenti:
\begin{itemize}

\item \textbf{Correttore automatico di TeXMaker$_G$:} l'ambiente grafi TeXMaker$_G$, già utilizzato per la stesura dei documenti, integra i dizionari di OpenOffice.org e segnala i potenziali errori ortografici presenti nel testo;

\item \textbf{404TrackerDB}: Strumento software realizzato dal gruppo \gruppo\ che contiene ed associa:
	\begin{itemize}
		\item Requisiti individuati durante l'analisi;
		\item Fonti di requisiti individuate, inclusi anche i casi d'uso.
	\end{itemize}
	Permette inoltre di esportare automaticamente:
	\begin{itemize}
		\item Codice \LaTeX\ per la descrizione dei casi d'uso;
		\item Tabella in \LaTeX\ per il tracciamento fonti-requisiti.
	\end{itemize}

\end{itemize}
\subsubsection{Diagrammi UML}
Per la creazione dei diagrammi UML$_G$ è stato scelto di usare il software \textit{Astah} versione community 6.9.0, disponibile per il sistema operativo Linux installato nella Virtual Machine$_G$.

\subsection{Strumenti di sviluppo}

\subsubsection{Ambiente}
\begin{itemize}
	\item \textbf{Virtual Machine$_G$}:
	\smallbreak
	Come ambiente di sviluppo si è scelto di creare una macchina virtuale$_G$ (VM) con Ubuntu Desktop 14.04, per avere a disposizione un ambiente uniforme per tutti i componenti del gruppo \gruppo. La VM sarà disponibile per il download ai componenti del team nell'account Mega$_G$
	\\ \\
	\item \textbf{Virtual Box versione 4.3.20}: è il software utilizzato per far girare la macchina virtuale sui personal computer de vari componenti del gruppo di lavoro;
	\item \textbf{Server dedicato}:
	\gruppo\ si avvale dell'uso di un server dedicato gestito da un applicazione creata dai membri del team di sviluppo denominata 404TrackerDB. Questo strumento servirà a contenere ed associare i requisiti individuati in fase di analisi;
    \item \textbf{Inkscape versione 0.48}: Software scelto su suggerimento del \textit{Proponente} per la creazione dei file SVG$_G$.
\end{itemize}

\subsubsection{Framework}
Come emerso dalla riunione con il proponente (vedi documento allegato \textit{Verbale18-12-2014\_v2.0.pdf}), è stato scelto di utilizzare \textit{MeteorJS} come framework open-source$_G$ per JavaScript$_G$. All'interno questo framework integra il database MongoDB$_G$.
	
\subsubsection{Linee Guida}
La linea guida grafica che si è scelto di utilizzare nella realizzazione del progetto Premi è quella ispirata da Google del Material Design$_G$. Questo in accordo con i desideri del Proponente \Zucchetti.


\subsection{Strmenti di codifica}
\subsubsection{Stesura}
    Per lo sviluppo del progetto si è scelto di utilizzare L’IDE JetBrains WebStorm 7.0.3. Questo strumento nasce come IDE per applicazioni web, quindi basate su HTML$_G$, CSS$_G$ e JavaScript$_G$. E' stato scelto principalmente perchè supporta nativamente lo sviluppo su MeteorJS$_G$ e integra git$_G$ per semplificare i commit$_G$ sul repository$_G$. Ogni membro del gruppo ha installato sul suo computer la versione per studenti dell'IDE che è disponibile in licenza gratuita per un anno associando il proprio account alla e-mail universitaria personale; 


\subsubsection{Verifica}
Per lo svolgimento del processo di verifica faremo uso dei seguenti strumenti:
\begin{itemize}
	\item Strumenti W3C$_G$ (\href{www.w3.org}{www.w3.org}) per la validazione:
	    \begin{itemize}
	    	\item \textbf{validatore HTML5$_G$} (\href{http://validator.w3.org}{http://validator.w3.org});
	    	\item \textbf{validatore CSS$_G$}
	    	(\href{http://jigsaw.w3.org/css-validator/}{http://jigsaw.w3.org/css-validator/}).
	    \end{itemize}
	
	\item Strumenti per debugging$_G$ HTML$_G$, CSS$_G$ e JavaScript$_G$ messi a disposizione dai vari browser$_G$:
	    \begin{itemize}
	    	\item \textbf{Chrome Developer Tools} (\href{https://developers.google.com/chrome-developer-tools}
	    	{https://developers.google.com/chrome-developer-tools});
	    	\item \textbf{Firebug}
	    	(\href{http://getfirebug.com/}{http://getfirebug.com/}).
	    \end{itemize}
	\item \textbf{JSLint} Ambiente di test (\href{http://www.junit.org}{http://www.junit.org}): tool per la validazione di codice JavaScript$_G$;
	\item \textbf{JUnit} (\href{http://www.junit.org}{http://www.junit.org}): semplice framework per eseguire test ripetibili;
	\item \textbf{BrowserStack} (\href{http://www.browserstack.com/}{http://www.browserstack.com/}):  per eseguire il test comparato sui vari browser$_G$;
	\item \textbf{WebStorm} (\href{https://www.jetbrains.com/webstorm/}{https://www.jetbrains.com/webstorm/}): IDE JavaScript$_G$ scelto come ambiente di sviluppo;
	\item \textbf{Jasmine} (\href{http://jasmine.github.io/}{http://jasmine.github.io/}): framework per testare codice Javascript;
	\item \textbf{Velocity} (\href{http://www.meteortesting.com/chapter/velocity}{http://www.meteortesting.com/chapter/velocity}): framework ufficiale per i test su MeteorJS.
\end{itemize}
Il gruppo dovrà inoltre avere a disposizione i seguenti sistemi operativi:

\subsubsection{Ambiente Desktop}
I seguenti sistemi operativi:
\begin{itemize}
\item Microsoft Windows XP
\item Microsoft Windows 8
\item Linux Ubuntu 14.04
\item Apple MacOs 
\end{itemize}
I seguenti browser:
\begin{itemize}
\item Microsoft Internet Explorer 8
\item Microsoft Internet Explorer 11
\item Apple Safari
\item Mozilla Firefox
\item Google Chrome
\end{itemize}
Le seguenti risoluzioni di monitor:
\begin{itemize}
\item 1024x764
\item 1440x960
\item 1920x1080
\end{itemize}

\subsubsection{Ambiente Mobile}
I seguenti sistemi operativi:
\begin{itemize}
\item Android
\item iOS
\item Windows Phone 8.1
\end{itemize}

\subsubsection{Ambiente Server}
I seguenti sistemi operativi:
\begin{itemize}
\item Ubuntu Server 14.04
\end{itemize}