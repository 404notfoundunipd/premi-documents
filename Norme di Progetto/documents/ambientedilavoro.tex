\section{Ambiente di Lavoro}
\subsection{Coordinamento}
Il coordinamento del gruppo avverrà tramite i seguenti strumenti:
\begin{itemize}
	\item \textbf{Account Google}:
	\begin{itemize}
		\item[-] Mailing List;
		\item[-] Google Calendar$_G$.
	\end{itemize}
	E' stato scelto Mozilla Thunderbird$_G$ come client di posta, in quanto è uno strumento utilizzato quotidianamente da molti componenti del gruppo, e inoltre è già presente nell'installazione base di Ubuntu desktop. Questo client, permette l'installazione di un plugin a noi necessario: Lightning$_G$. Quest'ultimo verrà utilizzato per sincronizzare i calendari di Google Calendar$_G$.
	\item \textbf{GitHub$_G$}:
	\begin{itemize}
		\item[-] Versionamento di GitHub$_G$;
		\item[-] Interfaccia Grafica SmartGit$_G$;
		\item[-] Integrazione di GitHub$_G$ in WebStorm;
		\item[-] Ticketing.
	\end{itemize}
\end{itemize}
\subsection{Ambiente Documentale}
\begin{itemize}
	\item \textbf{Documenti}:\\
	Per la stesura dei documenti si è scelto di usare il linguaggio di markup$_G$ \LaTeX. Il team \gruppo\ si avvarrà del software \textit{TexMaker} versione 4.4.1 per la redazione, la compilazione e l'esportazione dei file in \LaTeX;
	\item \textbf{Diagrammi UML}:
	Per la creazione dei diagrammi UML$_G$ è stato scelto di usare il software \textit{Astah} versione community 6.9.0, disponibile per il sistema operativo Linux installato nella Virtual Machine$_G$.
\end{itemize}
\subsection{Linee Guida}
La linea guida grafica che si è scelto di utilizzare nella realizzazione del progetto Premi è quella ispirata da Google del Material Design$_G$. Questo in accordo con i desideri del Proponente \Zucchetti.
\subsection{Framework}
Come emerso dalla riunione con il proponente (vedi documento allegato \textit{Verbale18-12-2014\_v2.0.pdf}), è stato scelto di utilizzare \textit{MeteorJS} come framework open-source$_G$ per JavaScript$_G$. All'interno questo framework integra il database MongoDB$_G$.

\subsection{Ambiente di Sviluppo}

\begin{itemize}
	\item \textbf{Virtual Machine$_G$}:
	\smallbreak
	Come ambiente di sviluppo si è scelto di creare una macchina virtuale$_G$ (VM) con Ubuntu Desktop 14.04, per avere a disposizione un ambiente uniforme per tutti i componenti del gruppo \gruppo. La VM sarà disponibile per il download ai componenti del team nell'account Mega$_G$
	\\ \\
	\textbf{Virtual Box versione 4.3.20}: è il software utilizzato per far girare la macchina virtuale sui personal computer de vari componenti del gruppo di lavoro;
	\item \textbf{Server dedicato}:
	\gruppo\ si avvale dell'uso di un server dedicato gestito da un applicazione creata dai membri del team di sviluppo denominata 404TrackerDB. Questo strumento servirà a contenere ed associare i requisiti individuati in fase di analisi;
    \item \textbf{WebStorm versione 9.0}: WebStrom è l'IDE che verrà utilizzato per lo sviluppo del progetto. Nasce come IDE per applicazioni web, quindi basate su HTML$_G$, CSS$_G$ e JavaScript$_G$. E' stato scelto principalmente perchè supporta nativamente lo sviluppo su MeteorJS$_G$ e integra git$_G$ per semplificare i commit$_G$ sul repository$_G$. Ogni membro del gruppo ha installato sul suo computer la versione per studenti dell'IDE che è disponibile in licenza gratuita per un anno associando il proprio account alla e-mail universitaria personale;
    
    \item \textbf{Inkscape versione 0.48}: Software scelto su suggerimento del \textit{Proponente} per la creazione dei file SVG$_G$.
\end{itemize}

\subsection{Ambiente di Verifica}
Gli strumenti di cui il team si avvarrà in fase di verifica sono descritti alla sezione 2.4.1 del documento \textit{PianoDiQualifica\_v2.0.pdf}.
Il gruppo dovrà inoltre avere a disposizione i seguenti sistemi operativi:

\subsubsection{Ambiente Desktop}
I seguenti sistemi operativi:
\begin{itemize}
\item Microsoft Windows XP
\item Microsoft Windows 8
\item Linux Ubuntu 14.04
\item Apple MacOs 
\end{itemize}
I seguenti browser:
\begin{itemize}
\item Microsoft Internet Explorer 8
\item Microsoft Internet Explorer 11
\item Apple Safari
\item Mozilla Firefox
\item Google Chrome
\end{itemize}
Le seguenti risoluzioni di monitor:
\begin{itemize}
\item 1024x764
\item 1440x960
\item 1920x1080
\end{itemize}

\subsubsection{Ambiente Mobile}
I seguenti sistemi operativi:
\begin{itemize}
\item Android
\item iOS
\item Windows Phone 8.1
\end{itemize}

\subsubsection{Ambiente Server}
I seguenti sistemi operativi:
\begin{itemize}
\item Ubuntu Server 14.04
\end{itemize}