\section{Procedure e supporto dei processi}
\subsection{Procedura per lo sviluppo dell'applicazione}
Vengono qui elencate le norme per l'utilizzo degli strumenti di sviluppo del sito web GitHub$_G$. \\
Per una guida completa sull'uso di GitHub$_G$ si consiglia il sito: \\
\url{https://guides.github.com/}\\
Per una guida completa sull'uso di git$_G$ da terminale si consiglia il sito: \\
\url{http://git-scm.com/doc/}

\subsubsection{Creare una milestone}
La creazione di una milestone$_G$ viene affidata esclusivamente al \ruoloResponsabile\ e il suo scopo è quello di suddividere il lavoro in elenchi di attività per un maggiore controllo sullo stato di avanzamento del progetto. Le attività possono essere a loro volta suddivise in task$_G$ e spartite tra i componenti del gruppo attraverso la creazione di ticket$_G$. \\cccx
    Dalla pagina della repository$_G$ accedere al menu \textit{issues}, andare sulla sezione \textit{Milestones} e quindi su \textit{New milestone$_G$}.\\
    Inserire un breve titolo che la distingua dalle milestone$_G$ precedenti, una descrizione approfondita che identifichi le parti del progetto su cui si sta lavorando e infine la data di chiusura.

\subsubsection{Creare un ticket}
Un ticket$_G$ corrisponde ad un compito da portare a termine all'interno di una milestone$_G$. Vengono creati dal \ruoloResponsabile\ e assegnati a sua discrezione ad un componente del gruppo. \\
Dalla pagina della repository$_G$ accedere al menu \textit{issues} e premere su \textit{New Issue}.\\
È richiesta la compilazione dei seguenti campi:
\begin{itemize}
    \item \textit{Title:} descrive sinteticamente il compito assegnato;
    \item \textit{Assigned to:} è il membro del gruppo che si occuperà del compito;
    \item \textit{Comment:} descrizione approfondita del compito. È possibile includere immagini esplicative se ritenuto necessario;
    \item \textit{Milestone$_G$:} per includere il ticket$_G$ in un contesto temporale;
    \item \textit{Labels:} aiutano a suddividere i compiti per tipo.
\end{itemize}
È possibile creare dei collegamenti ai ticket$_G$ all'interno dei commenti in GitHub$_G$ scrivendo l'id del ticket preceduto dal carattere \#.

\subsubsection{Eseguire il compito}
Per eseguire il compito assegnato, l'incaricato deve creare un branch$_G$ su cui lavorare. \\
Dalla pagina della repository$_G$ premere sul pulsante della lista dei branch$_G$ e digitare il nome del nuovo branch$_G$, che deve essere formato dal nome dell'incaricato seguito dall'id del ticket$_G$ nella forma: 
\begin{verbatim}
nome-id
\end{verbatim}
Se quello su cui si sta lavorando non dovesse essere collegato a nessun ticket$_G$ inserire, invece dell'id, un nome che identifichi il compito. \\
Viene lasciato a discrezione dell'incaricato il numero di salvataggi (commit$_G$) da effettuare. È però consigliabile salvare i files su su cui si sta lavorando nella repository$_G$ remota almeno una volta al giorno per non perdere il lavoro in caso di incidenti e in modo che il resto del gruppo possa vedere lo stato di avanzamento del compito. \\
Ogni commit$_G$ deve essere accompagnato da un breve titolo e una descrizione che elenchi quello che è stato aggiunto al branch$_G$. \\
Quando il lavoro è concluso inviare una \textit{pull request}$_G$, ossia la richiesta di riunire il branch$_G$ con l'originale. Dalla pagina della repository$_G$ accedere al menu \textit{Pull Requests}$_G$ e premere sul pulsante \textit{New pull request}; selezionare il branch$_G$ base e quello su cui si ha eseguito il compito e premere \textit{Create pull request}.\\ È richiesta la compilazione dei seguenti campi:
\begin{itemize}
    \item \textit{titolo:} deve essere il nome del branch$_G$ da riunire;
    \item \textit{commento:} deve contenere almeno l'id del ticket$_G$(\textit{issue}) del compito eseguito nella forma \texttt{\#id} e preceduto dalla parola chiave \texttt{Fixes }, per rendere automatica la chiusura del ticket una volta accettata la richiesta. 
\end{itemize}

\subsubsection{Fase di Verifica}
Il \ruoloVerificatore\ ha il compito di esaminare le \textit{pull request}$_G$ proposte dai membri del gruppo.\\
Dopo aver verificato il lavoro svolto, dal menu \textit{Pull Requests}$_G$ ha tre possibilità:
\begin{itemize}
\item \textit{Accettare la richiesta:} Il lavoro è corretto e soddisfa i requisiti del compito; il branch$_G$ viene unito al \textit{master} e il ticket$_G$ viene chiuso. Questa azione viene chiamata \textit{merge}$_G$;
\item \textit{Comunicare con il richiedente:} Se il lavoro presenta degli errori o non soddisfa i requisiti del compito, il \ruoloVerificatore\ può lasciare dei commenti all'interno della \textit{pull request}$_G$ in cui descrive i problemi riscontrati. L'incaricato è tenuto a risolvere i problemi e ad aggiornare il branch$_G$;
\item \textit{Rifiutare la richiesta:} La richiesta non è pertinente al lavoro in corso e viene chiusa senza effettuare il \textit{merge}$_G$. Il \ruoloVerificatore\ è tenuto comunque a lasciare un commento che giustifichi tale azione.
\end{itemize}

\subsubsection{Chiudere una milestone}
Quando tutti i ticket$_G$ sono stati chiusi il \ruoloResponsabile\ può chiudere la milestone$_G$ in corso. \\
Dalla pagina della repository$_G$ accedere al menu \textit{issues}, andare sulla sezione \textit{Milestones} e selezionare quella che si intende chiudere; infine premere su \textit{close}.




\subsection{Analisi dei Requisiti}
\subsubsection{Studio di Fattibilità}
In seguito alla scelta del capitolato da parte del gruppo è necessario che gli \textit{Analisti} redigano uno \textit{Studio di Fattibilità}. Questo documento dovrà focalizzarsi su:
\begin{itemize}
\item \textbf{Dominio applicativo e tecnologico:} la conoscenza, da parte dei componenti del gruppo, del dominio applicativo e delle tecnologie richieste nel  ; 
\item \textbf{Guadagno:} rapporto tra costi per la realizzazione e dei benefici ricavati dal prodotto completo;
\item \textbf{Analisi dei rischi:} individuazione dei rischi derivati da lacune di conoscenza del dominio e dalle tecnologie richieste.  
\end{itemize}
\subsubsection{Requisiti}
In seguito allo studio di fattibilità gli \ruoloAnalista dovranno identificare e classificare i Requisiti.
La classificazione verrà operata sui seguenti attributi:
\begin{itemize}
\item \textbf{Tipo del Requisito} i cui valori possono essere:
\begin{itemize}
\item[F] Funzionale;
\item[Q] di Qualità;
\item[V] di Vincolo;
\item[P] Prestazionale.
\end{itemize}
\item \textbf{Valore del Requisito} i cui valori possono essere:
\begin{itemize}
\item[Ob] Obbligatorio;
\item[De] Desiderabile;
\item[Op] Opzionale.
\end{itemize}
\item \textbf{Codice univoco} è il codice espresso in modo gerarchico come segue.
\begin{center}
[codice del padre].[numero unico rispetto ai fratelli]
\end{center}
\end{itemize} 
Oltre a questa classificazione è necessario aggiungere una breve descrizione ad ogni requisito
\subsubsection{Casi d'Uso}
In seguito alla classificazione dei requisiti gli \textit{Analisti} dovranno identificare i \textit{casi d'uso} o \textit{use case}.
Di ogni caso d'uso interessa:
\begin{itemize}
\item \textbf{Titolo} breve titolo che identifica lo UC;
\item \textbf{Pre e Post Condizione};
\item \textbf{Attori} principali e secondari;
\item \textbf{Scenari} principali e alternativi definendo per ognuno titolo, attori coinvolti, una breve descrizione e il caso d'uso a cui si riferiscono se presente;
\item \textbf{Descrizione} breve descrizione dello UC;
\item \textbf{Requisiti Dedotti}.
\end{itemize} 
\subsubsection{Tracciamento}
Sia i requisiti che le fonti dovranno essere inseriti, dagli \textit{Analisti}, in un database$_G$ appositamente creato. Per facilitare le operazioni necessarie è stata creata un'applicazione web il cui nome è 404TrackerDB.
Compito dell'applicazione è anche quella di creare il codice {Latex}, sia delle tabelle per il tracciamento sia delle generalità di fonti e requisiti, in modo automatico per velocizzare la stesura del documento \textit{Analisi dei Requisiti}. 