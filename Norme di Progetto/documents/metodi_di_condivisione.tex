\section{Metodi di Condivisione}
Qui vengono elencate le piattaforme adottate per facilitare lo svolgimento collettivo del progetto. \\
%Il gruppo di applicazioni web di Google viene utilizzato per la condivisione di documenti informali e la comunicazione degli incontri tra i membri, mentre il repository GitHub struttura il versionamento e la creazione della documentazione ufficiale e del software.\\ 
\subsection{Google Calendar}
Google Calendar viene utilizzato come promemoria per le date degli incontri e per segnalare le fasce orarie in cui un membro non e' disponibile in caso di impegni settimanali inderogabili. \\
\subsection{Google Drive}
Google Drive permette una facile condivisione di documenti informali e di tutte quelle risorse utili al progetto che non sono soggette a versionamento. \\
Per sfruttare nel modo migliore le potenzialita' della piattaforma si raccomanda di attenersi alla suddivisione intuitiva in cartelle dei file e di utilizzare, quando possibile, formati compatibili con gli editor testuali del sito.
\subsection{GitHub}
\href{http://www.github.com/}{GitHub} e' un servizio di hosting di repository \texttt{git}. \\
Una repository e' uno spazio di archiviazione da cui e' possibile recuperare software o codice sorgente; \texttt{git} in particolare sfrutta la tecnica dei branch, ossia di ramificazioni del codice sorgente su cui poter lavorare senza  entrare in conflitto con il lavoro di altri collaboratori.\\
Sono state create due repository, accessibili dall'indirizzo:\\ 
\url{https://github.com/404notfoundunipd/}
\begin{itemize}
\item \texttt{premi} contiene i file di codifica e di sviluppo dell'applicazione eseguibile;
\item \texttt{premi-documents} contiene i file di tutti i documenti formali soggetti a versionamento compresi i template per la loro stesura.
\end{itemize}
\section{Procedura per lo sviluppo dell'applicazione}
Vengono qui elencate le norme per l'utilizzo degli strumenti di sviluppo del sito web \href{http://www.github.com/}{GitHub}.
\subsection{Creare una milestone}
La creazione di una milestone viene affidata esclusivamente al Responsabile di Progetto e il suo scopo e' quello di suddividere il lavoro in piu' fasi per un maggiore controllo sullo stato di avanzamento del progetto. Questa parte di lavoro viene a sua volta spartita tra i componenti del gruppo attraverso i ticket. \\
    Dalla pagina della repository accedere al menu \texttt{issues}, andare sulla sezione \texttt{Milestones} e quindi su \texttt{New milestone}.\\
    Inserire un breve titolo che la distingua dalle milestone precedenti, una descrizione approfondita che identifichi le parti del progetto su cui si sta lavorando e infine la data di chiusura.

\subsection{Creare un ticket}
Un ticket corrisponde ad un compito da portare a termine all'interno di una milestone. Vengono creati dal Responsabile di Progetto e assegnati a sua discrezione ad un componente del gruppo. \\
Dalla pagina della repository accedere al menu \texttt{issues} e premere su \texttt{New Issue}.\\
E' richiesta la compilazione dei seguenti campi:
\begin{itemize}
    \item \texttt{Title:} descrive sinteticamente il compito assegnato;
    \item \texttt{Assigned to:} e' il membro del gruppo che si occupera' del compito;
    \item \texttt{Comment:} descrizione approfondita del compito. E' possibile includere immagini esplicative se ritenuto necessario;
    \item \texttt{Milestone:} per includere il ticket in un contesto temporale;
    \item \texttt{Labels:} aiutano a suddividere i compiti per tipo.
\end{itemize}

\subsection{Eseguire il compito}
Per portare a termine il compito assegnato, l'incaricato deve creare un branch su cui lavorare. \\
Dalla pagina della repository premere sul pulsante della lista dei branch e digitare il nome del nuovo branch, che deve essere formato da:
    \begin{itemize}
    \item Cognome e Nome dell'incaricato;
    \item \#id e nome del ticket, oppure un breve titolo che identifichi quello su cui si sta lavorando in caso non sia collegato a nessun ticket.
    \end{itemize}
    Viene lasciato a discrezione dell'incaricato il numero di salvataggi (commit) da effettuare. E' pero' consigliabile aggiornare il branch su cui si sta lavorando almeno una volta la settimana in modo che il resto del gruppo possa vedere il livello di avanzamento del compito. \\
    Ogni commit deve essere accompagnato da un breve titolo e una descrizione che elenchi quello che e' stato aggiunto al branch.
    Quando il lavoro e' terminato,

\subsection{Creare un ticket di verifica}
\subsection{Fase di Verifica(?)}
\subsection{Chiudere una milestone}