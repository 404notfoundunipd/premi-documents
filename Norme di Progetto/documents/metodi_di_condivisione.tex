\section{Metodi di Condivisione}
Qui vengono elencate le piattaforme adottate per facilitare lo svolgimento collettivo del progetto.

\subsection{Google Calendar}
Google Calendar$_G$ viene utilizzato come promemoria per le date degli incontri e per segnalare le fasce orarie in cui un membro non è disponibile in caso di impegni settimanali inderogabili.

\subsection{Google Drive}
Google Drive$_G$ permette una facile condivisione di documenti informali e di tutte quelle risorse utili al progetto che non sono soggette a versionamento. \\
Per sfruttare nel modo migliore le potenzialità della piattaforma si raccomanda di attenersi alla suddivisione intuitiva in cartelle dei file e di utilizzare, quando possibile, formati compatibili con gli editor$_G$ testuali del sito.

\subsection{GitHub}
GitHub$_G$ è un servizio di hosting di repository$_G$ Git$_G$. \\
Una repository è uno spazio di archiviazione da cui è possibile recuperare software o codice sorgente; il sistema di controllo di versione$_G$ Git in particolare sfrutta la tecnica dei branch$_G$, ossia di ramificazioni del codice sorgente su cui poter lavorare senza  entrare in conflitto con il lavoro di altri collaboratori. Il branch originale viene chiamato \textit{master}. \\
Sono state create due repository, accessibili dall'indirizzo:\\ 
\url{https://github.com/404notfoundunipd/}
\begin{itemize}
\item \texttt{premi} contiene i file di codifica e di sviluppo dell'applicazione eseguibile;
\item \texttt{premi-documents} contiene i file di tutti i documenti formali soggetti a versionamento compresi i template$_G$ per la loro stesura.
\end{itemize}

\subsection{Mega}
Mega$_G$ è un servizio di storage online e gratuito. È stato creato un account per il team \gruppo, che conterrà le esportazioni delle Virtual Machine$_G$ (VM) con Ubuntu Desktop. È stato scelto questo servizio cloud in quanto si è scelto di non tener traccia del versionamento delle VM. Sono inoltre a disposizione 50GB di spazio gratuito, necessario per poter memorizzare file di grosse di dimensioni come le VM.

\newpage
\section{Procedura per lo sviluppo dell'applicazione}
Vengono qui elencate le norme per l'utilizzo degli strumenti di sviluppo del sito web GitHub$_G$. \\
Per una guida completa sull'uso di GitHub si consiglia il sito: \\
\url{https://guides.github.com/}\\
Per una guida completa sull'uso di git$_G$ da terminale si consiglia il sito: \\
\url{http://git-scm.com/doc/}

\subsection{Creare una milestone}
La creazione di una milestone$_G$ viene affidata esclusivamente al \ruoloResponsabile{} e il suo scopo è quello di suddividere il lavoro in più fasi per un maggiore controllo sullo stato di avanzamento del progetto. Questa parte di lavoro viene a sua volta spartita tra i componenti del gruppo attraverso i ticket$_G$. \\
    Dalla pagina della repository$_G$ accedere al menu \textit{issues}, andare sulla sezione \textit{Milestones} e quindi su \textit{New milestone}.\\
    Inserire un breve titolo che la distingua dalle milestone precedenti, una descrizione approfondita che identifichi le parti del progetto su cui si sta lavorando e infine la data di chiusura.

\subsection{Creare un ticket}
Un ticket$_G$ corrisponde ad un compito da portare a termine all'interno di una milestone$_G$. Vengono creati dal \ruoloResponsabile{} e assegnati a sua discrezione ad un componente del gruppo. \\
Dalla pagina della repository$_G$ accedere al menu \textit{issues} e premere su \textit{New Issue}.\\
È richiesta la compilazione dei seguenti campi:
\begin{itemize}
    \item \textit{Title:} descrive sinteticamente il compito assegnato;
    \item \textit{Assigned to:} è il membro del gruppo che si occuperà del compito;
    \item \textit{Comment:} descrizione approfondita del compito. È possibile includere immagini esplicative se ritenuto necessario;
    \item \textit{Milestone$_G$:} per includere il ticket in un contesto temporale;
    \item \textit{Labels:} aiutano a suddividere i compiti per tipo.
\end{itemize}
È possibile creare dei collegamenti ai ticket all'interno dei commenti in GitHub$_G$ scrivendo l'id del ticket preceduto dal carattere \#.

\subsection{Eseguire il compito}
Per eseguire il compito assegnato, l'incaricato deve creare un branch$_G$ su cui lavorare. \\
Dalla pagina della repository premere sul pulsante della lista dei branch$_G$ e digitare il nome del nuovo branch$_G$, che deve essere formato dal nome dell'incaricato seguito dall'id del ticket nella forma: 
\begin{verbatim}
nome-id
\end{verbatim}
Se quello su cui si sta lavorando non dovesse essere collegato a nessun ticket inserire, invece dell'id, un nome che identifichi il compito. \\
Viene lasciato a discrezione dell'incaricato il numero di salvataggi (commit$_G$) da effettuare. È però consigliabile salvare i files su su cui si sta lavorando nella repository remota almeno una volta al giorno per non perdere il lavoro in caso di incidenti e in modo che il resto del gruppo possa vedere lo stato di avanzamento del compito. \\
Ogni commit deve essere accompagnato da un breve titolo e una descrizione che elenchi quello che è stato aggiunto al branch. \\
Quando il lavoro è concluso inviare una \textit{pull request}$_G$, ossia la richiesta di riunire il branch con l'originale. Dalla pagina della repository accedere al menu \textit{Pull Requests} e premere sul punsante \textit{New pull request}; selezionare il branch base e quello su cui si ha eseguito il compito e premere \textit{Create pull request}.\\ È richiesta la compilazione dei seguenti campi:
\begin{itemize}
    \item \textit{titolo:} deve essere il nome del branch da riunire;
    \item \textit{commento:} deve contenere almeno l'id del ticket(\textit{issue}) del compito eseguito nella forma \texttt{\#id} e preceduto dalla parola chiave \texttt{Fixes }, per rendere automatica la chiusura del ticket una volta accettata la richiesta. 
\end{itemize}

\subsection{Fase di Verifica}
Il Verificatore ha il compito di esaminare le \textit{pull request}$_G$ proposte dai membri del gruppo.\\
Dopo aver verificato il lavoro svolto, dal menu \textit{Pull Requests}$_G$ ha tre possibilità:
\begin{itemize}
\item \textit{Accettare la richiesta:} Il lavoro è corretto e soddisfa i requisiti del compito; il branch$_G$ viene unito al \textit{master} e il ticket$_G$ viene chiuso. Questa azione viene chiamata \textit{merge}$_G$;
\item \textit{Comunicare con il richiedente:} Se il lavoro presenta degli errori o non soddisfa i requisiti del compito, il Verificatore può lasciare dei commenti all'interno della \textit{pull request} in cui descrive i problemi riscontrati. L'incaricato è tenuto a risolvere i problemi e ad aggiornare il branch$_G$;
\item \textit{Rifiutare la richiesta:} La richiesta non è pertinente al lavoro in corso e viene chiusa senza effettuare il \textit{merge}$_G$. Il Verificatore è tenuto comunque a lasciare un commento che giustifichi tale azione.
\end{itemize}

\subsection{Chiudere una milestone}
Quando tutti i ticket sono stati chiusi il Responsabile può chiudere la milestone$_G$ in corso. \\
Dalla pagina della repository$_G$ accedere al menu \textit{issues}, andare sulla sezione \textit{Milestones} e selezionare quella che si intende chiudere; infine premere su \textit{close}.