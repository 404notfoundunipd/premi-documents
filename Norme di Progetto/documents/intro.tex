\section{Introduzione}

\subsection{Scopo del documento}
Lo scopo di questo documento è quello di delineare le norme e le convenzioni che verranno seguite dal gruppo \gruppo\ nel corso di tutte le attività di progetto quali comunicazioni, stesura della documentazione, riunioni e uso degli strumenti.
Questo documento è disponibile a tutti i componenti di \gruppo\ che dovranno seguirne le convenzioni al fine di ottimizzare il lavoro di gruppo.
Durante lo svolgimento delle attività sarà il responsabile a verificare l’applicazione delle norme di seguito definite e ad approvare nuove convenzioni proposte dai componenti del gruppo.

\subsection{Scopo del Prodotto}
Lo scopo del progetto è la realizzazione di un software di presentazione di slide non basato sul modello di PowerPoint$_{G}$, sviluppato in tecnologia HTML5$_{G}$ e che funzioni sia su desktop che su dispositivo mobile. Il software dovrà permettere la creazione da parte dell'autore e la successiva presentazione del lavoro, fornendo effetti grafici di supporto allo storytelling e alla creazione di mappe mentali. 

\subsection{Glossario}
Al fine di evitare ogni ambiguità relativa al linguaggio e ai termini utilizzati nei documenti formali tutti i termini e gli acronimi presenti nel seguente documento che necessitano di definizione saranno seguiti da una ”G” in pedice e saranno riportati in un documento esterno denominato Glossario.pdf. Tale documento accompagna e completa il presente e consiste in un listato ordinato di termini e acronimi con le rispettive definizioni e spiegazioni.

\subsection{Riferimenti}
\subsubsection{Informativi}
\begin{itemize}
	\item \textbf{Specifiche UTF-8$-G$:} \href{http://www.unicode.org/versions/Unicode6.1.0/ch03.pdf}{http://www.unicode.org/versions/Unicode6.1.0/ch03.pdf}
	\item \textbf{Specifiche HTML5-8$-G$:} \href{http://www.w3.org/html/wg/drafts/html/master/}{http://www.w3.org/html/wg/drafts/html/master/}
\end{itemize}