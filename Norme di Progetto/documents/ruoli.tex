\section{Ruoli di Progetto}
Per la lista dei compiti previsti per i sei ruoli del progetto si rimanda al Regolamento dell'Organigramma\cite{docorganigramma}.

\subsection{Rotazione dei ruoli}
Da regolamento\cite{docorganigramma} ogni componente del gruppo può occupare più di un ruolo nelle varie fasi del progetto, garantendo però l'assenza di conflitto di interessi tra i ruoli assunti. Questo richiede che le seguenti norme vengano attuate nell'assegnazione dei ruoli:
\begin{itemize}
\item rotazione del ruolo di \ruoloResponsabile{} ad ogni macro-fase per non lasciare in mano allo stesso componente il coordinamento delle attività per tutto l'arco temporale del progetto;
\item la verifica di un documento o di parte del codice dev'essere sempre eseguita da componenti che non hanno scritto quel documento o codificato quella parte di software.
\end{itemize}
Per facilitare lo sviluppo del software vengono inoltre suggerite le seguenti raccomandazioni:
\begin{itemize}
\item separazione del ruolo di \ruoloResponsabile{} nei suoi incarichi di redattore di documenti (che potranno essere scritti da più componenti) e di coordinatore di attività (insieme di incarichi che dovranno essere svolti dal Responsabile assegnato per la macro-fase corrente);
\item specializzazione di componenti del gruppo, quando ritenuto opportuno, nella stesura di determinati documenti per una maggiore efficienza nella loro elaborazione e verifiche incrociate su tali documenti per assicurare che ognuno li conosca in dettaglio.
\end{itemize}