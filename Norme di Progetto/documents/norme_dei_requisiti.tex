
\section{Analisi dei Requisiti}
\subsection{Studio di Fattibilità}
In seguito alla scelta del capitolato da parte del gruppo è necessario che gli \textit{Analisti} redigano uno \textit{Studio di Fattibilità}. Questo documento dovrà focalizzarsi su:
\begin{itemize}
\item \textbf{Dominio applicativo e tecnologico:} la conoscenza, da parte dei componenti del gruppo, del dominio applicativo e delle tecnologie richieste nel  ; 
\item \textbf{Guadagno:} rapporto tra costi per la realizzazione e dei benefici ricavati dal prodotto completo;
\item \textbf{Analisi dei rischi:} individuazione dei rischi derivati da lacune di conoscenza del dominio e dalle tecnologie richieste.  
\end{itemize}
\subsection{Requisiti}
In seguito allo studio di fattibilità gli \ruoloAnalista dovranno identificare e classificare i Requisiti.
La classificazione verrà operata sui seguenti attributi:
\begin{itemize}
\item \textbf{Tipo del Requisito} i cui valori possono essere:
\begin{itemize}
\item[F] Funzionale;
\item[Q] di Qualità;
\item[V] di Vincolo;
\item[P] Prestazionale.
\end{itemize}
\item \textbf{Valore del Requisito} i cui valori possono essere:
\begin{itemize}
\item[Ob] Obbligatorio;
\item[De] Desiderabile;
\item[Op] Opzionale.
\end{itemize}
\item \textbf{Codice univoco} è il codice espresso in modo gerarchico come segue.
\begin{center}
[codice del padre].[numero unico rispetto ai fratelli]
\end{center}
\end{itemize} 
Oltre a questa classificazione è necessario aggiungere una breve descrizione ad ogni requisito
\subsection{Casi d'Uso}
In seguito alla classificazione dei requisiti gli \textit{Analisti} dovranno identificare i \textit{casi d'uso} o \textit{use case}.
Di ogni caso d'uso interessa:
\begin{itemize}
\item \textbf{Titolo} breve titolo che identifica lo UC;
\item \textbf{Pre e Post Condizione};
\item \textbf{Attori} principali e secondari;
\item \textbf{Scenari} principali e alternativi definendo per ognuno titolo, attori coinvolti, una breve descrizione e il caso d'uso a cui si riferiscono se presente;
\item \textbf{Descrizione} breve descrizione dello UC;
\item \textbf{Requisiti Dedotti}.
\end{itemize} 
\subsection{Tracciamento}
Sia i requisiti che le fonti dovranno essere inseriti, dagli \textit{Analisti}, in un database$_G$ appositamente creato. Per facilitare le operazioni necessarie è stata creata un'applicazione web il cui nome è 404TrackerDB.
Compito dell'applicazione è anche quella di creare il codice {Latex}, sia delle tabelle per il tracciamento sia delle generalità di fonti e requisiti, in modo automatico per velocizzare la stesura del documento \textit{Analisi dei Requisiti}. 