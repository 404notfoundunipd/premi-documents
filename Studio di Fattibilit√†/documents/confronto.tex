\section{Confronto con gli altri capitolati}
\subsection{Capitolato C1 - Applicazione Cloud per il Monitoraggio dei
	BigData nei Social Network}
  \subsubsection{Dominio delle Applicazioni}
  Proposto dalla societa Zing S.r.l. (www.zing-store.com/it), questo capitolato chiede la realizzazione di una infrastruttura che permetta di interrogare big data dai socialnetwork Facebook, Twitter e Instagram.
  \begin{itemize}
  	\item \textbf{La Google Cloud Platform} (https://cloud.google.com/) è uno stack tecnologico composto da una serie di prodotti pensati per il supporto allo sviluppo nel cloud.
  	\begin{itemize}
  		\item Google App Engine
  		\item Google Compute Engine
        \item Google Cloud Storage
  		\item Google Cloud Datastore
  		\item Google Cloud SQL
  		\item Google BigQuery
  		\item Google Cloud Endpoints
  	\end{itemize}
  \end{itemize}
  \subsubsection{Conoscenze interne}
  Buona conoscenza da parte del team della maggiorparte dei linguaggi consigliati dal capitolato Java, Php, HTML, CSS3, JQuery.
  \subsubsection{Analisi dei rischi}
  Il rischio più grande emerso durante lo studio del capitolato in oggetto è dovuto alla presenza di altri due gruppi di lavoro in gara per l'appalto. La scelta da parte del team di concorrere per questo capitolato rappresenterebbe pertanto un alto rischio di essere respinti dall'accesso al progetto d'esame.
  \subsubsection{Punti a favore}
  \begin{itemize}
     \item Zing S.r.l. si propone di illustrare l’utilizzo degli strumenti sopra elencati per progettare e realizzare una piattaforma web che sfrutti al meglio le potenzialità del cloud.
     \item Tali prodotti hanno tutti una versione di utilizzo gratuita e includono una interfaccia web.
     \item Grande interesse da parte del gruppo a svolgere il progetto, sia per l'ambito di sviluppo dei socialnetwork, sia per la modernità e la spendibilità delle conoscenze tecniche aquisibili dal progetto.
  \end{itemize}
\subsection{Capitolato C2 - GUS: Glass (Uni) Scanner}
  \subsubsection{Dominio delle Applicazioni}
  Il progetto richiede la costruzione di una applicazione/sistema software, applicabile specificamente al settore del vetro, il  sistema  informatico deve essere in grado di analizzare un'intera immagine prodotta nella fase finale di produzione dalla scansione della lastra di vetro evidenziando e classificandone i difetti. E' Richiesto espressamente l'uso delle seguenti tecnologie:
  \begin{itemize}
  	\item \textbf{Linguaggio  di  programmazione:} C++/Php/JavaScript
  	\item \textbf{IDE di sviluppo:} QtEditor
  	\item \textbf{DatabaseRelazionale:} MySQL
  	\item \textbf{Licenza d'uso:} Licenza MIT (http://opensource.org/licenses/MIT)
  \end{itemize}
  \subsubsection{Conoscenze interne}
  
  \subsubsection{Analisi dei rischi}
  %il sistema in oggetto e sicuramente complesso e richiede particolare attenzione in tutti i suoi componenti
  
  %primaria efficacia e facilità d'uso
  
  
  \subsubsection{Punti a favore}
  
  %conoscenze interne 
  
  
  
\subsection{Capitolato C3}
  \subsubsection{Dominio delle Applicazioni}
  \subsubsection{Conoscenze interne}
  \subsubsection{Analisi dei rischi}
  \subsubsection{Punti a favore}
\subsection{Capitolato C5}
  \subsubsection{Dominio delle Applicazioni}
  \subsubsection{Conoscenze interne}
  \subsubsection{Analisi dei rischi}
  \subsubsection{Punti a favore}