\section{Confronto con gli altri capitolati}
\subsection{Capitolato C1 - BDSMApp: Big Data Social Monitoring App}
  \subsubsection{Dominio delle Applicazioni}
  Proposto dalla società Zing S.r.l. (www.zing-store.com/it), questo capitolato chiede la realizzazione di una infrastruttura che permetta di interrogare big data dai socialnetwork Facebook, Twitter e Instagram.
  \begin{itemize}
  	\item \textbf{La Google Cloud Platform} (https://cloud.google.com/) è uno stack tecnologico composto da una serie di prodotti pensati per il supporto allo sviluppo nel cloud:
  	\begin{itemize}
  		\item Google App Engine;
  		\item Google Compute Engine;
        \item Google Cloud Storage;
  		\item Google Cloud Datastore;
  		\item Google Cloud SQL;
  		\item Google BigQuery;
  		\item Google Cloud Endpoints.
  	\end{itemize}
  \end{itemize}
  \subsubsection{Conoscenze interne}
  Buona conoscenza da parte del team della maggiorparte dei linguaggi consigliati dal capitolato Java, Php, HTML, CSS3, JQuery.
  \subsubsection{Analisi dei rischi}
  Il rischio più grande emerso durante lo studio del capitolato in oggetto è dovuto alla presenza di altri due gruppi di lavoro in gara per l'appalto. La scelta da parte del team di concorrere per questo capitolato rappresenterebbe pertanto un alto rischio di essere respinti dall'accesso al progetto d'esame.
  \subsubsection{Punti a favore}
  \begin{itemize}
     \item Zing S.r.l. si propone di illustrare l’utilizzo degli strumenti sopra elencati per progettare e realizzare una piattaforma web che sfrutti al meglio le potenzialità del cloud;
     \item Tali prodotti hanno tutti una versione di utilizzo gratuita e includono una interfaccia web;
     \item Grande interesse da parte del gruppo a svolgere il progetto, sia per l'ambito di sviluppo dei socialnetwork, sia per la modernità e la spendibilità delle conoscenze tecniche acquisibili dal progetto.
  \end{itemize}
\subsection{Capitolato C2 - GUS: Glass (Uni) Scanner}
  \subsubsection{Dominio delle Applicazioni}
  Il progetto richiede la costruzione di una applicazione/sistema software, applicabile specificamente al settore del vetro, il  sistema  informatico deve essere in grado di analizzare un'intera immagine prodotta nella fase finale di produzione dalla scansione della lastra di vetro evidenziando e classificandone i difetti. E' Richiesto espressamente l'uso delle seguenti tecnologie:
  \begin{itemize}
  	\item \textbf{Linguaggio  di  programmazione:} C++/Php/JavaScript;
  	\item \textbf{IDE di sviluppo:} QtEditor;
  	\item \textbf{DatabaseRelazionale:} MySQL;
  	\item \textbf{Licenza d'uso:} Licenza MIT (http://opensource.org/licenses/MIT).
  \end{itemize}
  \subsubsection{Conoscenze interne}
  All'interno del team tutti i membri hanno una approfondita conoscenza del linguaggio C++ derivante dagli studi universitari e hanno già utilizzato per lo svolgimento di progetti didattici le librerie Qt e l'IDE QtCreator/QtEditor. Php, Java e i database relazionali nonchè il linguaggio MySQL fanno altresì parte del baglio culturale dei tutti i membri del gruppo.
  \subsubsection{Analisi dei rischi}
  Il sistema in oggetto è sicuramente complesso e richiede particolare attenzione. La difficoltà maggiore è stata individuata nel creare un algoritmo che analizzi in modo efficiente ed efficace immagini molto grandi in termini di spazio (circa 500Mb). Inoltre un altro punto critico è la categorizzazione delle imperfezioni rilevate dal software che richiede uno studio e una conoscenza più approfondita del processo produttivo del vetro.
  \subsubsection{Punti a favore}
  \begin{itemize}
  	\item Le tecnologie necessarie allo svolgimento del progetto sono bene conosciute e già state utilizzate dal gruppo;
  	\item Il prodotto da realizzare sembra essere facilmente suddivisibile in moduli;
  	\item il capitolato si presenta molto dettagliato ed è esposto in tutte le sue parti in maniera chiara e senza ambiguità.
  \end{itemize}
\newpage
\subsection{Capitolato C3 - Norris: Node Real-time Intelligence}
  \subsubsection{Dominio delle Applicazioni}
  Proposto dalla società CoffeStrap, questo capitolato richiede la produzione di un framework$_{G}$ che permette di raccogliere dati provenienti da sorgenti arbitrarie e visualizzarli come grafici in modo semplice e veloce. Stack tecnologico richiesto:
  \begin{itemize}
  	\item \textbf{Node.js} (http://nodejs.org/);
  	\item \textbf{Express.js} (http://expressjs.com/);
  	\item \textbf{Socket.io} (http://socket.io);
  	\item \textbf{Licenza d'uso:} Licenza MIT (http://opensource.org/licenses/MIT).
  \end{itemize}
  \subsubsection{Conoscenze interne}
  Dall'analisi non sono emerse competenze interne al team di sviluppo per quanto riguarda le tecnologie Express.js e Socket.io. Solo alcuni possiedono invece una conoscenza superficiale di Node.js
  \subsubsection{Analisi dei rischi}
  Il gruppo \gruppo\ ha manifestato scarso interesse e conoscenze del dominio applicativo e si ritiene che le tecnologie sconosciute 
  potrebbero creare diffcoltà e/o rallentamenti durante lo sviluppo.
  Inoltre quando è stato valutato il capitolato altri tre gruppi di lavoro erano già in gara per l'appalto ed era già stato raggiunto il numero massimo di team gestibili dal proponente. La scelta da parte del team di concorrere per questo capitolato rappresenterebbe pertanto un alto rischio di essere respinti dall'accesso al progetto d'esame.
  \subsubsection{Punti a favore}
  \begin{itemize}
  	\item Numerosissime fonti informative citate nel capitolato;
  	\item Grande disponibilità rilevata da parte del proponente.
  \end{itemize}
\newpage
\subsection{Capitolato C5 - sHike: A smart cloud and mobile platform appliance for the safety and health in mountain hiking}
  \subsubsection{Dominio delle Applicazioni}
  Proposto dalla società SI14 SpA, il progetto sHike consiste nello sviluppo di una applicazione per lo Smartwatch WearIT, che si appoggi su una piattaforma cloud$_{G}$, e che fornisca all'utilizzatore del dispositivo informazioni e suggerimenti utili sull'ambiente in cui si trova, l'obiettivo è dare un supporto ai neofiti delle attività fisiche sportive e/o ricreative in montagna. Si chiede di realizzare l'applicazione per il dispositivo basato sul sistema operativo Android 4.4.2, opzionale è invece l'applicazione Java per il portale cloud WearIT. Lo stack tecnologico da utilizzare è il seguente:
  \begin{itemize}
  \item \textbf{sHike SmartHiking App:}
     \begin{itemize}
	     \item Android 4.4.2;
	     \item Extension WearIT API;
		 \item JSON Schema (http://json-schema.org/).
     \end{itemize}
  \item \textbf{sHike SmartHiking Cloud Application:}
     \begin{itemize}
       	\item JSON Schema (http://json-schema.org/);
       	\item The Spring Framework;
	    \item WearIT-Cloud API.
     \end{itemize}
  \end{itemize}
  \subsubsection{Conoscenze interne}
  Non è emersa alcuna conoscenza pregressa interna al team di sviluppo da parte di nessun membro, inoltre è stato rilevato scarso gradimento per il contesto di utilizzo della tecnologia sHike nella maggiorparte dei componenti del gruppo.
  \subsubsection{Analisi dei rischi}
  Al momento dell'analisi del capitolato altri tre gruppi di lavoro erano già in gara per l'appalto ed era già stato raggiunto il numero massimo di team gestibili dal proponente. La scelta da parte del team di concorrere per questo capitolato rappresenterebbe pertanto un alto rischio di essere respinti dall'accesso al progetto d'esame. E' inoltre molto probabile che le tecnologie sconosciute creino difficoltà e/o rallentamenti durante lo sviluppo di un progetto che nel suo complesso risulta senza dubbio complesso.
  \subsubsection{Punti a favore}
  \begin{itemize}
     \item Forte interesse del gruppo a lavorare con la nuova tecnologia Smartwatch WearIT non ancora lanciata nel mercato;
     \item La grande integrazione dell'azienda IS14 SpA con l'Università degli Studi di Padova lascia immaginare un maggior supporto agli studenti che dovessero scegliere questo capitolato.
  \end{itemize}