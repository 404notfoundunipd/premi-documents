\section{Capiotlato C4 - Premi: "better than Prezi"}
\subsection{Descrizione}
Lo scopo del progetto è la realizzazione di un software di presentazione di \textit{slide} non basato sul modello di PowerPoint, sviluppato in tecnologia HTML5 e che funzioni sia su desktop che su dispositivo mobile. Si richiede agli sviluppatori di realizzare effetti grafici a supporto dello
“storytelling” che siano comparabili al conosciuto sistema Prezi. Il software dovrà fornire funzionalità per la creazione da parte dell' autore e la presentazione al pubblico, sia in presenza diretta che via web. Le specifiche dell'appalto richiedono di utilizzare tecnologie Web. Il programma è inteso come un software che per l'occasione utilizzi il linguaggio Javascript e le librerie contenute
nel browser.

\subsection{Studio del dominio}
Per lo sviluppo del capitolato scelto sono necessarie alcune competenze di natura tecnologica e la conoscenza del contesto nel quale si inserisce l’applicazione. Il capitolato in esame affronta il problema della realizzazione ed esposizione di presentazioni sul browser. Di seguito vengono illustrati i domini tecnologici e applicativi ai quali si riferisce il capitolato.

\subsubsection{Dominio applicativo}
Per l’approfondimento del dominio applicativo, il gruppo si è documentato su alcuni sistemi software suggeriti dal Proponente:\\
\\
\textbf{Software:}
\begin{itemize}
	\item \textbf{Microsoft Power Point} - Il più noto e diffuso sistema per la realizzazione di presentazioni.
	\item \textbf{Prezi} - Alternativa a PowerPoint ad oggi molto diffusa.
	\item \textbf{Presentazioni in Google Documents} - Offerta di Google per la creazione e visualizzazione di slides.
\end{itemize}
\textbf{Aplicazioni Web:}
\begin{itemize}
	\item \textbf{Visme.co} - http://www.visme.co/ - Strumento per la creazione online di presentazioni, animazioni, banner animati e infografiche sul browser.
	\item \textbf{RealTime board} - https://realtimeboard.com/ - Software online per la collaborazione di gruppo e tool per condividere brainstorming.
	\item \textbf{Canva.com} - https://www.canva.com/ - Sito per la creazione di design sul web o per la stampa: presentazioni, cover di Facebook, poster, inviti, ecc...
	\item \textbf{Easel.ly} - http://www.easel.ly/ - Semplice Web tool che permette la creazione e la condivisione di infografiche e poster.
	\item \textbf{PicktoChart} - http://piktochart.com/ - Facile app per il design di infografiche e per la realizzazione di grafici di alta qualità.
\end{itemize}

\subsubsection{Dominio tecnologico}

Da una prima analisi del capitolato è emerso la necessità di conoscenza delle seguenti tecnologie:

\begin{itemize}
	\item \textbf{HTML5:} per la realizzazione di pagine web (requisito obbligatrio).
	\item \textbf{JavaScript:} per la realizzazione di pagine web (requisito obbligatrio).
	\item \textbf{AngularJS e Node.js/Meteor:} per la comunicazione client-server.
	\item \textbf{Tecnologia SVG:} per la visualizzazione di oggetti di grafica vettoriale.
\end{itemize}
Ulteriori tecnologie prese in considerazione:
\begin{itemize}
	\item \textbf{Librerie HTML5}
	\begin{itemize}
		\item Impress.js
		\item Reveal.js
		\item Deck.js
		\item Google slides template - https://code.google.com/p/io-2012-slides/
		\item Slides - https://github.com/briancavalier/slides
	\end{itemize}
\end{itemize}

\subsection{Fattibilità}
Il capitolato presenta i seguenti punti che il gruppo ha valutato positivi:
\begin{itemize}
	\item[-] Forte interesse del gruppo nei confronti del capitolato, della progrmmazione sul web e delle tecnologie necessarie alla realizzazione del prodotto.
	\item[-] Esperienza e conoscenze tecniche acquisite a fine progetto giudicate spendibili nel mondo del lavoro.
	\item[-] Ampia scelta sulle tecnologie da usare.
	\item[-] Buona conoscenza da parte di tutto il gruppo di HTML e Javascript derivata dagli studi universitari intrapresi.
	\item[-] Conoscenze pregresse di alcuni membri del team di sviluppo del linguaggio AngularJS.
	\item[-] Consapevolezza della forte differenza ce c'è tra PowerPoint e Prezi e chiara comprensione delle aspettative del Proponente.
\end{itemize}

\subsection{Potenziali criticità}
Nessun membro del gruppo possiede conoscenze approfondite riguardanti HTML5 e solo pochi membri hanno usato precedentemente AngularJs e Node.js/Meteor. Sebbene Il capitolato presenti in maniera chiara le caratteristiche e i requisiti minimi richiesti, il gruppo ritiene che ci siano alcune caratteristiche/condizioni non esplicite e sono inoltre presenti molti requisiti opzionali. La presenza di altri software simili molto affermati nel mercato rappresenta poi un potenziale ostacolo da superare.