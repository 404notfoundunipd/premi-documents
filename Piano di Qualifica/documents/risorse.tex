\section{Risorse}
La gestione della qualità prevede l'utilizzo di alcune risorse suddivisibili in categorie.
\subsection{Risorse Necessarie} 
\subsubsection{Risorse Umane}
I ruoli necessari per garantire un'adeguata qualità sono i seguenti:
\begin{itemize}
	\item[-] \textbf{\ruoloResponsabile}: E' responsabile nei confronti del committente della corretta realizzazione del prodotto;
	\item[-] \textbf{\ruoloVerificatore}: Coordina e svolge le attività di verifica vere e proprie;
	\item[-] \textbf{\ruoloProgrammatore}: Esegue attività di debugging sul codice.
\end{itemize}
Per una descrizione dettagliata delle figure elencate e di tutti gli altri ruoli specifici si rimanda al documento allegato \textit{PianoDiPrgetto\_v1.0.pdf}.

\subsubsection{Risorse Software}
Durante la fase di realizzazione del progetto saranno necessari:
\begin{itemize}
	\item[-] Software per la gestione di documenti in \LaTeX;
	\item[-] Piattaforma di testing sui vari browser$_G$ dell'applicazione da sviluppare;
	\item[-] Piattaforma di versionamento per la creazione e la gestione di ticket$_G$;
	\item[-] Software per la creazione dei diagrammi in UML$_G$;
	\item[-] Ambiente per lo sviluppo del codice nel linguaggio di programmazione scelto;
	\item[-] Strumenti di validazione del codice prodotto.
\end{itemize}

\subsubsection{Risorse Hardware}
\begin{itemize}
	\item[-] Computer dotati di tutti gli strumenti software descritti nel Piano di Qualifica e nelle Norme di Progetto;
	\item[-] Luogo fisico in cui incontrarsi per lo sviluppo del progetto, possibilmente con una connessione ad Internet.
\end{itemize}

\subsection{Risorse Disponibili}
\subsubsection{Risorse Software}
Vengono di seguito elencate le risorse software disponibili. Per una descrizione più dettagliata si rimanda alla sottosezione Strumenti 2.4.1 del presente documento.
\begin{itemize}
	\item[-] TeXMaker per l'editing dei documenti in \LaTeX;
	\item[-] BrowserStack per il testing sui vari browser$_G$;
	\item[-] GitHub per il versionamento e la gestione dei ticket$_G$;
	\item[-] Astah per i diagrammi UML$_G$;
	\item[-] WebStorm come ambiente di sviluppo;
	\item[-] Strumenti di validazione online del W3C$_G$.
\end{itemize}

\subsubsection{Risorse Hardware}
\begin{itemize}
	\item[-] Computer personali (portatili o fissi) dei membri del gruppo; 
	\item[-] Computer messi a disposizione nei laboratori informatici del Dipartimento di Matematica Pura ed Applicata dell'Università di Padova;
	\item[-] Aule studio del Dipartimento di Matematica Pura ed Applicata
	dell'Università di Padova. 
\end{itemize}