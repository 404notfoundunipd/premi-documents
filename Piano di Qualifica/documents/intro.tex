\section{Introduzione}

\subsection{Scopo del documento}
Questo documento ha lo scopo di illustrare le strategie adottate per implementare i processi di verifica e validazione del lavoro svolto da \gruppo\ per assicurare la qualità del progetto Premi e dei processi coinvolti nel suo sviluppo. Per raggiungere gli obiettivi di qualità è necessario un processo di verifica continua sulle attività svolte, per questo motivo il presente documento potrà essere aggiornato in seguito a scelte progettuali del gruppo e/o variazione dei requisiti da parte del Proponente.

\subsection{Scopo del Prodotto}
Lo scopo del progetto è la realizzazione di un software di presentazione di slide non basato sul modello di PowerPoint$_{G}$, sviluppato in tecnologia HTML5$_{G}$ e che funzioni sia su desktop che su dispositivo mobile. Il software dovrà permettere la creazione da parte dell'autore e la successiva presentazione del lavoro, fornendo effetti grafici di supporto allo storytelling e alla creazione di mappe mentali.

\subsection{Glossario}
Al fine di evitare ogni ambiguità relativa al linguaggio e ai termini utilizzati nei documenti formali tutti i termini e gli acronimi presenti nel seguente documento che necessitano di definizione saranno seguiti da una ``G'' in pedice e saranno riportati in un documento esterno denominato Glossario\_v1.0.pdf. Tale documento accompagna e completa il presente e consiste in un listato ordinato di termini e acronimi con le rispettive definizioni e spiegazioni.

\subsection{Riferimenti}
\subsubsection{Normativi}
\begin{itemize}
	\item \textbf{Norme di Progetto:} \textit{NormeDiProgetto\_v4.0.pdf};
	\item \textbf{Capitolato d'appalto C4:} Premi: Software di presentazione ``better than Prezi'' - \href{http://www.math.unipd.it/~tullio/IS-1/2014/Progetto/C4.pdf}{http://www.math.unipd.it/$\sim$tullio/IS-1/2014/Progetto/C4.pdf}.
\end{itemize}
\subsubsection{Informativi}
\begin{itemize}
	\item \textbf{Piano di Progetto}: \textit{PianoDiProgetto\_v.4.0.pdf};
	\item \textbf{Slide dell'insegnamento Ingegneria del Software modulo A}:\\ \href{http://www.math.unipd.it/~tullio/IS-1/2014/}{http://www.math.unipd.it/$\sim$tullio/IS-1/2014/};
	\item \textbf{Ingegneria del software - Ian Sommerville - 8a Edizione (2007)}:
	\begin{itemize}
		\item[-] Capitolo 27 - Gestione della qualità;
		\item[-] Capitolo 28 - Miglioramento dei processi.
	\end{itemize}
	\item \textbf{Complessità ciclomatica}: \href {http://it.wikipedia.org/wiki/Complessità\_ciclomatica}{http://it.wikipedia.org/wiki/Complessità\_ciclomatica};
	\item \textbf{ISO/IEC$_G$ 9126:2001} (inglobato da ISO/IEC$_G$ 25010:2011):\\
	\href{http://www2.cnipa.gov.it/site/_contentfiles/01379900/1379951_ISO \%209126.pdf}{http://www2.cnipa.gov.it/site/\_contentfiles/01379900/1379951\_ISO  209126.pdf};	\begin{itemize}
		\item[-] Systems and software engineering;
		\item[-] Systems and software Quality Requirements and Evaluation (SQuaRE);
		\item[-] System and software quality models.
	\end{itemize}
	\item \textbf{ISO/IEC$_G$ 15504:1998}: Information Tecnology - Process Assessment, conosciuto come SPICE (Software Process Improvement and Capability Determination): \href{http://www2.cnipa.gov.it/site/_contentfiles/00310300/310320_15504.pdf}{http://www2.cnipa.gov.it/site/\_contentfiles/00310300/310320\_15504.pdf}.
\end{itemize}