\section{Resoconto dell' Attività di Verifica}
In questa sezione vengono descritte le procedure adottate durante il processo di
verifica e i risultati ottenuti.
\subsection{Revisione della Documentazione}
Riguardo all'attività di verifica della documentazione, la checklist stilata dai verificatori durante i controlli sui documenti tramite Inspection è la seguente:\\ \\
\textbf{Per i documenti in \LaTeX}:
\begin{itemize}
	\item[-] assenza di doppi spazi;
	\item[-] uso corretto delle lettere maiuscole e minuscole negli elenchi puntati e all'inizio di ogni frase;
	\item[-] assenza di errori ortografici di battitura;
	\item[-] presenza dello spazio dopo il segno di punteggiatura; 
	\item[-] assenza di parti mancanti nei documenti;
	\item[-] mancanza nel glossario della spiegazione di termini segnati $_G$ nei documenti;
	\item[-] evitare di scrivere frasi troppo lunghe;
	\item[-] evitare l'inserimento di spazi nei tag \LaTeX{};
	\item[-] assenza di spazi all'apertura e alla chiusura delle parentesi tonde o quadre;
	\item[-] presenza dello spazio dopo i segni di punteggiatura; 
	\item[-] verifica del funzionamento dei link dei documenti.
\end{itemize}
\textbf{Per i diagrammi UML$_G$}:
\begin{itemize}
	\item[-] il sistema non deve essere un attore;
	\item[-] direzione delle freccie scorretta;
	\item[-] controllo ortografico.
\end{itemize}
\subsection{Tracciamento requisiti}
Il tracciamento dei requisiti viene eseguito tramite il software 404TrackerDB descritto nella sezione 2.4.1. 
Grazie anche allo strumento TexMaker$_G$ descritto nella sezione 2.4.1 si sono potuti individuare errori ortografici mentre la parte di controllo grammaticale è avvenuta mediante la rilettura da parte dei verificatori dei documenti. I verificatori nel segnalare gli errori hanno emesso i ticket ai redattori che sono stati poi risolti dagli stessi.
Ciò che si è controllato viene descritto dalla lista seguente:
\begin{itemize}
\item[-] ad ogni use case$_G$ deve corrispondere un requisito;
\item[-] ad ogni requisito deve corrispondere la sua fonte;
\item[-] i requisiti devono coprire l'intero capitolato;
\item[-] Ogni requisito deve avere un codice univoco;
\item[-] i codici dei casi d'uso nei diagrammi devono corrispondere;
\item[-] la numerazione dei casi d'uso non deve contenere salti.
\end{itemize}
I requisiti presenti nel documento \textit{AnalisiDeiRequisiti\_v1.0.pdf} sono:
\begin{itemize}
\item \textbf{Totali}: 45;
\item \textbf{Funzionali utente}: 42;   
\item \textbf{di Vincolo}: 3.
\end{itemize}
Gli use case$_G$ individuati sono 93 tutti lato utente.
\subsection{Dettaglio delle verifiche tramite analisi}
La verifiche tramite analisi statica avvengono con le modalità walkthrough e inspect spiegate sezione 2.4.2 e permettono di controllare l'andamento e la qualità del lavoro svolto.
\subsection{Revisione dei Requisiti}
Nel periodo antecedente la consegna di tale revisione sono stati verificati i documenti ed i processi.
L'analisi statica è stata applicata secondo i criteri e le modalità indicate nella sezione 2.5.2. Effettuando walkthrough sono stati riscontrati degli errori. Sono state quindi avviate le procedure per la segnalazione e la correzione, descritte nell'apposita sezione delle \textit{Norme di Progetto v2.0}.
Noti gli errori, si è provveduto a:
\begin{itemize}
	\item Correggere le imperfezioni rilevate;
	\item Segnalare gli errori più frequenti.Si è quindi applicato il ciclo PDCA per rendere più efficiente ed efficace il processo di verifica.
\end{itemize}
È stata in seguito applicata l'inspection utilizzando la lista di controllo stilata durante la verifica dei documenti precedentemente verificati, ponendo particolare attenzione ai grafici dei casi d'uso.
Il tracciamento (requisiti - fonti, use-case - requisiti) è stato effettuato tramite l'applicativo 404TrackerDB.

\subsection{Revisione di Progettazione}

Nel periodo antecedente la consegna di tale revisione sono stati verificati i documenti ed i processi.
