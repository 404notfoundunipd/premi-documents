\section{Resoconto dell' Attività di Verifica}

\subsection{Tracciamento componenti - requisiti}
Il tracciamento dei componenti e dei requisiti viene eseguita tramite il repository GitHub che mantiene uno storico delle attività svolte e delle segnalazioni effettuate dal Verificatore e dagli altri componenti del gruppo.
Grazie anche allo strumento TexMaker descritto nella sezione 2.4.1 si è potuto individuare errori ortografici mentre la parte di controllo grammaticale è avvenuta mediante la rilettura da parte dei verificatori dei documenti. I verificatori nel segnalare gli errori hanno emesso i ticket ai redattori che sono stati poi risolti dagli stessi.
Ciò che si è controllato viene descritto dalla lista seguente:
- ad ogni use case$_G$ deve corrispondere un requisito;
- ad ogni requisito deve corrispondere la sua fonte;
- i requisiti devono coprire l'intero capitolato
- Ogni requisito deve avere un codice univoco;
- i codici dei casi d'uso nei diagrammi devono corrispondere;
- la numerazione dei casi d'uso non deve contenere salti;
I requisiti presenti nel documento AnalisiDeiRequisiti.pdf sono:
\begin{itemize}
\item Totali: 
\item Funzionali utente:
\item Funzionali amministratore:
\item di Vincolo:
\item di Qualità:
\end{itemize}
Gli use case$_G$ individuati sono di cui lato utente e lato  amministratore.
\subsection{Dettaglio delle verifiche tramite analisi}
La verifiche tramite analisi statica avvengono con le modalità walkthrough e inspect spiegate sezione 2.4.2 e permettono di controllare l'andamento e la qualità del lavoro svolto.
