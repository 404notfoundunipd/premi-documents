\newenvironment{margini}[2]
{
	\begin{list}{} {
			\setlength{\leftmargin}{#1}
			\setlength{\rightmargin}{#2}
		} \item }
	{\end{list}}

\section{Domande e risposte}
Di seguito vengono riportate in grassetto le domande effettuate dal gruppo \gruppo\ nel corso della riunione e in corsivo le risposte date dal Proponente Zucchetti Spa.\\
\\\textbf{Cosa si intende precisamente dicendo che il software dovrà essere sviluppato in tecnologia HTML5, si tratta a tutti gli effetti di una Web Application?}\\
\begin{margini}{0.7cm}{0.7cm}
\textit{Si.  Aggiungo per  precisione che l'approccio che dovrete avere nella realizzazione di questo progetto sarà quello di usare normali tecnologie web come un vero e proprio ambiente di sviluppo software. Linguaggi come Javascript e ad HTML5 saranno pertanto utilizzati come qualsiasi alto linguaggio astraendo dal contesto web. Non vi viene quindi chiesto di porre l'accento su problemi di accessibilità né di fare assunzioni su categorie di utenze che si avvalgono dello screenreader per la navigazione sul web.}\\
\end{margini}
\hrule
\bigskip
\textbf{Qual'è la differenza tra presentazione e presentazione sul browser?}\\
\begin{margini}{0.7cm}{0.7cm}
	\textit{L'idea è quella di farvi esplorare nuove tecnologie e possibilmente nel browser. Al giorno d'oggi la maggior parte dei sistemi informatici soffre di una terribile intersezione con Office, noi miriamo ad abbattere quella barriera offrendo all'utenza qualcosa di alternativo. Esistono già da anni valide alternative a PowerPoint, noi vorremmo svilupparne una personalizzata che si distacchi il più possibile dal prodotto Microsoft. Fondamentale è il distacco dall'idea di una presentazione intesa come sequenza predefinita di slides, in questo senso ad esempio si può pensare di offrire più percorsi percorribili per ogni presentazione a seconda del pubblico o del presentatore. Quest'ultimo esempio si ricollega allo svolgimento non lineare delle presentazioni descritto nel capitolato.}\\
\end{margini}
\hrule
\bigskip
\textbf{Qual'è l'importanza dell'infografica menzionata nel capitolato?}\\
\begin{margini}{0.7cm}{0.7cm}
	\textit{La cosa importante è che l'applicazione renda facile la creazione di infografiche, quindi la comunicazione tramite un'immagine di informazioni che il pubblico possa cogliere in modo più immediato rispetto al testo sullo schermo. Vi sono vari modi di creare infografica. A noi sembra molto importante fornire come funzionalità del programma la possibilità di ridurre la  presentazione a un poster o mappa finale. Questa, ad esempio, potrebbe mostrare in grande le informazioni fondamentali con possibile perdita di dettaglio ma rimane comunque utile come infografica collettiva finale della presentazione nel suo insieme.}\\
\end{margini}
\newpage
\noindent \textbf{Possiamo usare librerie open source per template, icone e grafici?}\\
\begin{margini}{0.7cm}{0.7cm}
	\textit{Assolutamente si. Anzi vi esortiamo a sperimentare tutto quello che volete in questo senso!}\\
\end{margini}
\hrule
\bigskip
\textbf{Avete consigli sulle librerie esterne da utilizzare?}\\
\begin{margini}{0.7cm}{0.7cm}
	\textit{Vi consigliamo di dare un occhiata alle librerie D3js, perchè sono molto potenti e offrono una distinzione tra lista dei dati e lista di elementi a video che risulta essere molto comoda.}\\
\end{margini}
\hrule
\bigskip
\textbf{Quali funzionalità devono essere rese disponibili su dispositivo mobile?}\\
\begin{margini}{0.7cm}{0.7cm}
	\textit{Senz'altro la visualizzazione delle presentazioni. Per quanto riguarda la creazione di presentazioni è invece un'operazione che si presta poco ad uno schermo così piccolo che rende difficile all'utente questa operazione, pertanto direi che non è necessario.}\\
\end{margini}
\hrule
\bigskip
\textbf{É necessario rendere disponibile la presentazione in remoto?}\\
\begin{margini}{0.7cm}{0.7cm}
	\textit{Questo può essere un valido requisito opzionale, ma deve essere sensato. Potrebbe essere interessante utilizzare la cache applicativa del browser che è molto leggera e interroga meno il server, così da permettere all'utente di continuare a lavorare anche se la connessione a metà del lavoro venisse a mancare, ma per completare e salvare il lavoro è richiesta la connessione al server. Per quanto riguarda l'esportazione su file è anch'essa opzionale, non ci sono vincoli sul formato o sul numero di file in cui salvare la presentazione.}\\
\end{margini}
\hrule
\bigskip
\textbf{Cosa ne pensate dell'utilizzo combinato di Angularjs lato client e Nodejs o Meteor lato server?}\\
\begin{margini}{0.7cm}{0.7cm}
	\textit{Conosciamo molto bene Angularjs lo utilizziamo tutt'ora combinato a Nodejs e sono entrambi validi, se deciderete di utilizzare Meteor siamo molto interessati al vostro lavoro per conoscere la tecnologia e ci interesserà un vostro riscontro in proposito.}\\
\end{margini}
