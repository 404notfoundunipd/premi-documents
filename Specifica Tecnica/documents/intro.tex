\section{Introduzione}
\subsection{Scopo del documento}
Questo documento definisce la progettazione ad alto livello di Premi. Viene prima descritta la struttura generale del sistema e successivamente vengono analizzate le varie componenti software in relazione alle loro attività principali. Segue poi la descrizione delle tecnologie e dei Design Pattern$_{G}$ utilizzati, e un mockup$_{G}$ dell'interfaccia grafica lato utente.

\subsection{Scopo del prodotto}
Lo scopo del progetto è la realizzazione di un software di presentazione di slide non basato sul modello di PowerPoint$_{G}$, sviluppato in tecnologia HTML5$_{G}$ e che funzioni sia su desktop che su dispositivo mobile. Il software dovrà permettere la creazione da parte dell'autore e la successiva presentazione del lavoro, fornendo effetti grafici di supporto allo storytelling e alla creazione di mappe mentali. 

\subsection{Glossario}
Al fine di evitare ogni ambiguità relativa al linguaggio e ai termini utilizzati nei documenti formali tutti i termini e gli acronimi presenti nel seguente documento che necessitano di definizione saranno seguiti da una ”G” in pedice e saranno riportati in un documento esterno denominato Glossario.pdf. Tale documento accompagna e completa il presente e consiste in un listato ordinato di termini e acronimi con le rispettive definizioni e spiegazioni.


\textbf{Mockup} è un modello in scala o a dimensione reale di un progetto o un dispositivo, il cui scopo è quello di promuovere, descrivere o valutare il prodotto finale. Se un Mockup possiede almeno una parte delle funzionalità del prodotto finale e consente il loro collaudo allora viene definito \textit{prototipo}.
	
\textbf{Design Pattern}, nell'ambito dell'ingegneria del software, è un concetto che può essere definito come ''una soluzione progettuale generale ad un problema ricorrente''. \\
Ogni pattern descrive un problema riscontrabile più volte nell'ambiente di sviluppo di un progetto, e suggerisce le linee di base per risolvere questo problema in modo tale che la soluzione possa essere sempre adattabile al contesto specifico del progetto.
	
	
