\section{Descrizione dei singoli componenti}

\subsection{premi/server}
\begin{figure}[H]
\begin{center}
\includegraphics[scale=0.45]{img/diapkg/server.png}
\caption{Diagramma del package premi/server}
\end{center}
\end{figure}

\subsubsection{premi/server/publish}
\begin{itemize}
  \item[] \textbf{Nome:} publish
  \item[] \textbf{Tipo:} class
  \item[] \textbf{Package:} premi/server
  \item[] \textbf{Descrizione:} questa classe pubblica all'utente solo ed esclusivamente le informazioni a cui lui ha accesso e di cui ha bisogno nel contesto in cui si trova. \\ Utilizza funzionalità fornite dai framework MeteorJS$_G$ e AngularJS$_G$ che sfruttano il design pattern Publish-Subscribe$_G$ (vedere la sezione \ref{DesignPattern} per ulteriori informazioni)
\end{itemize}


\subsubsection{premi/server/methods}
\begin{itemize}
  \item[] \textbf{Nome:} server
  \item[] \textbf{Tipo:} class
  \item[] \textbf{Package:} premi/server
  \item[] \textbf{Descrizione:} questa classe fornisce al lato client dell'applicazione dei metodi per l'inserimento, l'aggiornamento e la rimozione dei dati del database, che verranno resi disponibili tramite speciali oggetti del framework$_G$ MeteorJS$_G$.
\end{itemize}





\subsection{premi/client}
\begin{figure}[H]
\begin{center}
\includegraphics[scale=0.45]{img/diapkg/client.png}
\caption{Diagramma dei package views e controllers di premi}
\end{center}
\end{figure}

\subsubsection{premi/client/views/header.ng}
\begin{itemize}
  \item[] \textbf{Nome:} header.ng
  \item[] \textbf{Tipo:} template
  \item[] \textbf{Package:} premi/client/views
  \item[] \textbf{Descrizione:} template dell'header della pagina principale dell'applicazione
\end{itemize}

\subsubsection{premi/client/controllers/premi}
\begin{itemize}
  \item[] \textbf{Nome:} premi
  \item[] \textbf{Tipo:} controller
  \item[] \textbf{Package:} premi/client/controllers
  \item[] \textbf{Descrizione:} controller generale della pagina principale dell'applicazione
\end{itemize}

\subsubsection{premi/client/lib/toastMessageFactory}
\begin{itemize}
  \item[] \textbf{Nome:} toastMessageFactory
  \item[] \textbf{Tipo:} classe
  \item[] \textbf{Package:} premi/client/lib
  \item[] \textbf{Descrizione:} fornisce una semplice funzione per l'invio di notifiche o messaggi di errore all'utente
\end{itemize}

\subsection{premi/client/userManager}
\begin{figure}[H]
\begin{center}
\includegraphics[scale=0.45]{img/diapkg/userManager.png}
\caption{Diagramma del package premi/client/userManager}
\end{center}
\end{figure}

\subsubsection{premi/client/userManager/views/signin.ng}
\begin{itemize}
  \item[] \textbf{Nome:} signin.ng
  \item[] \textbf{Tipo:} template
  \item[] \textbf{Package:} premi/client/userManager/views
  \item[] \textbf{Descrizione:} template dell'userManager per effettuare il login utente.
  \item[] \textbf{Relazioni con altri componenti:} la view generata da questo template è collegata allo \textit{\$scope} di premi/client/userManager/controllers/signinCtrl per effettuare il login utente.
\end{itemize}

\subsubsection{premi/client/userManager/views/signup.ng}
\begin{itemize}
  \item[] \textbf{Nome:} signup.ng
  \item[] \textbf{Tipo:} template
  \item[] \textbf{Package:} premi/client/userManager/views
  \item[] \textbf{Descrizione:} template dell'userManager per effettuare la registrazione dell' utente.
  \item[] \textbf{Relazioni con altri componenti:} la view generata da questo template è collegata allo \textit{\$scope} di premi/client/userManager/controllers/signupCtrl per effettuare la registrazione utente.
\end{itemize}

\subsubsection{premi/client/userManager/views/changePassword.ng}
\begin{itemize}
  \item[] \textbf{Nome:} changePassword.ng
  \item[] \textbf{Tipo:} template
  \item[] \textbf{Package:} premi/client/userManager/views
  \item[] \textbf{Descrizione:} template dell'userManager per effettuare il cambio password.
  \item[] \textbf{Relazioni con altri componenti:} la view generata da questo template è collegata allo \textit{\$scope} di premi/client/userManager/controllers/changePasswordCtrl per effettuare il cambio password.
\end{itemize}

\subsubsection{premi/client/userManager/views/userManager.ng}
\begin{itemize}
  \item[] \textbf{Nome:} user.ng
  \item[] \textbf{Tipo:} template
  \item[] \textbf{Package:} premi/client/userManager/views
  \item[] \textbf{Descrizione:} template principale dell'userManager che serve a contenere altre view.
  \item[] \textbf{Relazioni con altri componenti:} la view generata da questo template è collegata allo \textit{\$scope} di premi/client/userManager/controllers/userCtrl per eseguire gli altri controller.
\end{itemize}

\subsubsection{premi/client/userManager/controllers/signinCtrl}
\begin{itemize}
  \item[] \textbf{Nome:} signinCtrl
  \item[] \textbf{Tipo:} controller
  \item[] \textbf{Package:} premi/client/userManager/controllers
  \item[] \textbf{Descrizione:} controller di premi/client/userManager/views/signin.ng
  \item[] \textbf{Relazioni con altri componenti:} modella lo \textit{\$scope} per interagire con la view generata da premi/client/userManager/views/signin.ng
\end{itemize}

\subsubsection{premi/client/userManager/controllers/signupCtrl}
\begin{itemize}
  \item[] \textbf{Nome:} signupCtrl
  \item[] \textbf{Tipo:} controller
  \item[] \textbf{Package:} premi/client/userManager/controllers
  \item[] \textbf{Descrizione:} controller di premi/client/userManager/views/signup.ng
  \item[] \textbf{Relazioni con altri componenti:} modella lo \textit{\$scope} per interagire con la view generata da premi/client/userManager/views/signup.ng e dipende da:
  \begin{itemize}
   \item \textit{premi/client/presentation/lib/databaseAPI} per interagire con il database e salvare l'utente registrato.
  \end{itemize}
\end{itemize}

\subsubsection{premi/client/userManager/controllers/signoutCtrl}
\begin{itemize}
  \item[] \textbf{Nome:} signoutCtrl
  \item[] \textbf{Tipo:} controller
  \item[] \textbf{Package:} premi/client/userManager/controllers
  \item[] \textbf{Descrizione:} permette ad un utente loggato di effettuare il logout.
\end{itemize}

\subsubsection{premi/client/userManager/controllers/changePasswordCtrl}
\begin{itemize}
  \item[] \textbf{Nome:} userManagerCtrl
  \item[] \textbf{Tipo:} controller
  \item[] \textbf{Package:} premi/client/userManager/controllers
  \item[] \textbf{Descrizione:} controller di premi/client/userManager/views/changePassword.ng
  \item[] \textbf{Relazioni con altri componenti:} modella lo \textit{\$scope} per interagire con la view generata da premi/client/userManager/views/changePassword.ng e dipende da:
  \begin{itemize}
   \item \textit{premi/client/lib/toastMessageFactory} per la gestione delle notifiche all'utente.
  \end{itemize}
\end{itemize}

\subsection{premi/client/presentation}
\begin{figure}[!h]
\begin{center}
\includegraphics[scale=0.45]{img/diapkg/presentation.png}
\caption{Diagramma del package premi/client/presentation}
\end{center}
\end{figure}


\subsubsection{premi/client/presentation/lib/databaseAPI}
\begin{itemize}
  \item[] \textbf{Nome:} databaseAPI
  \item[] \textbf{Tipo:} classe
  \item[] \textbf{Package:} premi/client/presentation/lib/
  \item[] \textbf{Descrizione:} estende i metodi di premi/server/publish e li specializza per i bisogni del client
   \item[] \textbf{Relazioni con altri componenti:} dipende da:
 \begin{itemize}
 \item \textit{premi/server/publish} per derivare i metodi di gestione dei dati lato client
 \end{itemize}
\end{itemize}


\subsubsection{premi/client/presentation/lib/OrderedGoList}
\begin{itemize}
  \item[] \textbf{Nome:} OrderedGoList
  \item[] \textbf{Tipo:} classe
  \item[] \textbf{Package:} premi/client/presentation/lib/
  \item[] \textbf{Descrizione:} classe che modella una lista ordinata di oggetti grafici. Verrà inserita in un servizio factory$_G$ di AngularJS$_G$ che genererà istanze della classe quando esse saranno richieste attraverso il pattern Dependency Injection (vedere la sezione \ref{DesignPattern} per ulteriori informazioni)
\end{itemize}
  
\subsubsection{premi/client/presentation/lib/Trail}
\begin{itemize}
  \item[] \textbf{Nome:} Trail
  \item[] \textbf{Tipo:} classe
  \item[] \textbf{Package:} premi/client/presentation/lib/
  \item[] \textbf{Descrizione:} classe che modella un Trail, ossia un percorso di presentazione. Deve poter fornire i metodi per scorrere la presentazione e inserire o rimuovere Frame e checkpoint. Verrà inserito in un servizio factory$_G$ di AngularJS$_G$ che genererà istanze della classe quando esse saranno richieste attraverso il pattern Dependency Injection (vedere la sezione \ref{DesignPattern} per ulteriori informazioni)
\end{itemize}

\subsubsection{premi/client/presentation/lib/signalCtrl}
\begin{itemize}
  \item[] \textbf{Nome:} signalCtrl
  \item[] \textbf{Tipo:} classe
  \item[] \textbf{Package:} premi/client/presentation/lib/
  \item[] \textbf{Descrizione:} classe ha lo scopo di registrare i signal quando un certo slot viene creato in modo tale che facendo un controllo sul relativo controller si eviti di aggiungerne in più.
\end{itemize}

\subsubsection{premi/client/presentation/lib/filter}
\begin{itemize}
  \item[] \textbf{Nome:} filter
  \item[] \textbf{Tipo:} classe
  \item[] \textbf{Package:} premi/client/presentation/lib/
  \item[] \textbf{Descrizione:} filter di angular che permette di ordinare gli elementi all'interno di un oggetto JSON. 
\end{itemize}



\subsection{premi/client/presentationManager}
\begin{figure}[!h]
\begin{center}
\includegraphics[scale=0.45]{img/diapkg/presentationManager.png}
\caption{Diagramma del package premi/client/presentation}
\end{center}
\end{figure}


\subsubsection{premi/client/presentationManager/views/editPresentation.ng}
\begin{itemize}
  \item[] \textbf{Nome:} editPresentation.ng
  \item[] \textbf{Tipo:} template
  \item[] \textbf{Package:}  premi/client/presentationManager/views/
  \item[] \textbf{Descrizione:} template della parte di pagina che offre all'utente la possibilità di modificare una presentazione
  \item[] \textbf{Relazioni con altri componenti:}  la view generata da questo template è collegata allo \textit{\$scope} di premi/client/presentationManager/controllers/editPresentationCtrl per la modifica di una presentazione
\end{itemize}

\subsubsection{premi/client/presentationManager/views/newPresentation.ng}
\begin{itemize}
  \item[] \textbf{Nome:} newPresentation.ng
  \item[] \textbf{Tipo:} template
  \item[] \textbf{Package:} premi/client/presentationManager/views/
  \item[] \textbf{Descrizione:} template della parte di pagina che offre all'utente la possibilità di creare una nuova presentazione
  \item[] \textbf{Relazioni con altri componenti:}  la view generata da questo template è collegata allo \textit{\$scope} di premi/client/presentationManager/controllers/newPresentationCtrl per l'aggiunta di una presentazione vuota nel database di proprietà dell'utente
\end{itemize}

\subsubsection{premi/client/presentationManager/views/presentationManager.ng}
\begin{itemize}
  \item[] \textbf{Nome:} presentationManager.ng
  \item[] \textbf{Tipo:} template
  \item[] \textbf{Package:} premi/client/presentationManager/views/
  \item[] \textbf{Descrizione:} template dello scheletro della pagina di gestione delle presentazioni dell'utente
  \item[] \textbf{Relazioni con altri componenti:}  la view generata da questo template è collegata allo \textit{\$scope} di premi/client/presentationManager/controllers/newPresentationCtrl
\end{itemize}

\subsubsection{premi/client/presentationManager/views/presentations.ng}
\begin{itemize}
  \item[] \textbf{Nome:} presentations.ng
  \item[] \textbf{Tipo:} template
  \item[] \textbf{Package:} premi/client/presentationManager/views/
  \item[] \textbf{Descrizione:} template della parte di pagina che mostra all'utente la lista delle sue presentazioni
  \item[] \textbf{Relazioni con altri componenti:} la view generata da questo template è collegata allo \textit{\$scope} di premi/client/presentationManager/controllers/presentationsCtrl per accedere alla lista delle presentazioni
\end{itemize}

\subsubsection{premi/client/presentationManager/views/removePresentation.ng}
\begin{itemize}
  \item[] \textbf{Nome:} removePresentation.ng
  \item[] \textbf{Tipo:} template 
  \item[] \textbf{Package:} premi/client/presentationManager/views/
  \item[] \textbf{Descrizione:} template della parte di pagina che offre all'utente la possibilità di eliminare una presentazione 
  \item[] \textbf{Relazioni con altri componenti:} la view generata da questo template è collegata allo \textit{\$scope} di premi/client/presentationManager/controllers/removePresentationCtrl per rimuovere una presentazione dal database
\end{itemize}

\subsubsection{premi/client/presentationManager/controllers/editPresentationCtrl}
\begin{itemize}
  \item[] \textbf{Nome:} editPresentationCtrl
  \item[] \textbf{Tipo:} controller
  \item[] \textbf{Package:} premi/client/presentationManager/controllers
  \item[] \textbf{Descrizione:} controller di premi/client/presentationManager/views/editPresentation.ng
  \item[] \textbf{Relazioni con altri componenti:} modella lo \textit{\$scope} per interagire con la view generata da premi/client/presentationManager/views/editPresentation.ng e dipende anche da:
 \begin{itemize}
 \item \textit{premi/client/presentation/lib/databaseAPI} per l'accesso al database per la modifica dei campi dati della presentazione
 \end{itemize}
\end{itemize}

\subsubsection{premi/client/presentationManager/controllers/newPresentationCtrl}
\begin{itemize}
  \item[] \textbf{Nome:} newPresentationCtrl
  \item[] \textbf{Tipo:} controller
  \item[] \textbf{Package:} premi/client/presentationManager/controllers/
  \item[] \textbf{Descrizione:} controller di premi/client/presentationManager/views/newPresentation.ng
  \item[] \textbf{Relazioni con altri componenti:} modella lo \textit{\$scope} per interagire con la view generata da premi/client/presentationManager/views/newPresentation.ng e dipende anche da:
 \begin{itemize}
 \item \textit{premi/client/presentation/lib/databaseAPI} per l'accesso al database per l'aggiunta di una nuova presentazione
 \end{itemize}
\end{itemize}

\subsubsection{premi/client/presentationManager/controllers/PresentationManagerCtrl}
\begin{itemize}
  \item[] \textbf{Nome:} presentationManagerCtrl
  \item[] \textbf{Tipo:} controller
  \item[] \textbf{Package:} premi/client/presentationManager/controllers
  \item[] \textbf{Descrizione:}  controller di premi/client/presentationManager/views/presentationManager.ng
  \item[] \textbf{Relazioni con altri componenti:} modella lo \textit{\$scope} per interagire con la view generata da premi/client/presentationManager/views/presentationManager.ng
\end{itemize}

\subsubsection{premi/client/presentationManager/controllers/presentationsCtrl}
\begin{itemize}
  \item[] \textbf{Nome:} presentationsCtrl
  \item[] \textbf{Tipo:} controller
  \item[] \textbf{Package:} premi/client/presentationManager/controllers
  \item[] \textbf{Descrizione:} controller di premi/client/presentationManager/views/presentationManager.ng, fornisce alla vista la lista delle presentazioni dell'utente pubblicate dal server
  \item[] \textbf{Relazioni con altri componenti:} modella lo \textit{\$scope} per interagire con la view generata da premi/client/presentationManager/views/presentations.ng
\end{itemize}

\subsubsection{premi/client/presentationManager/controllers/removePresentationCtrl}
\begin{itemize}
  \item[] \textbf{Nome:} removePresentationCtrl
  \item[] \textbf{Tipo:} controller
  \item[] \textbf{Package:} premi/client/presentationManager/controllers
  \item[] \textbf{Descrizione:} controller di premi/client/presentationManager/views/removePresentation.ng, fornisce alla vista dei metodi per la rimozione della presentazione selezionata
  \item[] \textbf{Relazioni con altri componenti:} modella lo \textit{\$scope} per interagire con la view generata da premi/client/presentationManager/views/removePresentation.ng e dipende anche da:
 \begin{itemize}
 \item \textit{premi/client/presentation/lib/databaseAPI} per l'accesso al database per la rimozione della presentazione selezionata
\end{itemize}
\end{itemize}



\clearpage
\subsection{premi/client/editor}
\begin{figure}[!h]
\begin{center}
\includegraphics[scale=0.50]{img/diapkg/editor.png}
\caption{Diagramma del package premi/client/editor}
\end{center}
\end{figure}
\subsubsection{premi/client/editor/lib/GObject}
\begin{itemize}
  \item[] \textbf{Nome:} GObject
  \item[] \textbf{Tipo:} \textit{abstract class}
  \item[] \textbf{Package:} premi/client/editor/lib
  \item[] \textbf{Descrizione:} rappresenta gli oggetti grafici nella presentazione. Verrà inserito in un servizio factory$_G$ di AngularJS$_G$ che genererà istanze della classe quando esse saranno richieste attraverso il pattern Dependency Injection (vedere la sezione \ref{DesignPattern} per ulteriori informazioni)
\end{itemize}
\subsubsection{premi/client/editor/lib/Observer}
\begin{itemize}
  \item[] \textbf{Nome:} Observer
  \item[] \textbf{Tipo:} classe
  \item[] \textbf{Package:} premi/client/editor/lib
  \item[] \textbf{Descrizione:} si occupa di osservare degli oggetti grafici impostando e inviando dei segnali.
\end{itemize}
\subsubsection{premi/client/editor/lib/InteractInit}
\begin{itemize}
  \item[] \textbf{Nome:} InteractInit
  \item[] \textbf{Tipo:} classe
  \item[] \textbf{Package:} premi/client/editor/lib
  \item[] \textbf{Descrizione:} classe che inizializza la libreria esterna Interact e la prepara per il ridimensionamento e lo spostamento di oggetti grafici.
  \item[] \textbf{Relazioni con altri componenti:} InteractInit dipende da:
  \begin{itemize}
  	\item \textit{premi/client/editor/lib/Observer} per osservare gli oggetti grafici;
  	\item \textit{Interact} libreria esterna che facilita il ridimensionamento e lo spostamento di oggetti grafici
  \end{itemize}
\end{itemize}
\subsubsection{premi/client/editor/lib/GOProvider}
\begin{itemize}
  \item[] \textbf{Nome:} GOProvider
  \item[] \textbf{Tipo:} classe
  \item[] \textbf{Package:} premi/client/editor/lib
  \item[] \textbf{Descrizione:} permette di interfacciarsi con gli oggetti image, text, shape accedendo ai loro metodi pubblici. Verrà inserito in un servizio factory$_G$ di AngularJS$_G$ che genererà istanze della classe quando esse saranno richieste attraverso il pattern Dependency Injection (vedere la sezione \ref{DesignPattern} per ulteriori informazioni)
  \item[] \textbf{Relazioni con altri componenti:} GOProvider dipende da:
  \begin{itemize}
  	\item \textit{premi/client/editor/lib/Text} per accedere ai metodi di text;
  	\item \textit{premi/client/editor/lib/Image} per accedere ai metodi di image;
  	\item \textit{premi/client/editor/lib/Shape} per accedere ai metodi di shape.
  \end{itemize}
\end{itemize}
\subsubsection{premi/client/editor/lib/GOContainer}
\begin{itemize}
  \item[] \textbf{Nome:} GOContainer
  \item[] \textbf{Tipo:} \textit{abstract class}
  \item[] \textbf{Package:} premi/client/editor/lib
  \item[] \textbf{Descrizione:} rappresenta gli oggetti grafici che possono essere contenuti in un frame. Verrà inserito in un servizio factory$_G$ di AngularJS$_G$ che genererà istanze della classe quando esse saranno richieste attraverso il pattern Dependency Injection (vedere la sezione \ref{DesignPattern} per ulteriori informazioni)
  \item[] \textbf{Relazioni con altri componenti:} estende \textit{premi/client/editor/GObject} e dipende anche da:
  \begin{itemize} 
	\item \textit{premi/client/editor/lib/GOProvider} per accedere ai metodi pubblici degli oggetti image, text e shape.
\end{itemize}  
\end{itemize}
\subsubsection{premi/client/editor/lib/Text}
\begin{itemize}
  \item[] \textbf{Nome:} Text
  \item[] \textbf{Tipo:} classe
  \item[] \textbf{Package:} premi/client/editor/lib
  \item[] \textbf{Descrizione:} rappresenta un'area di testo nella presentazione. Verrà inserito in un servizio factory$_G$ di AngularJS$_G$ che genererà istanze della classe quando esse saranno richieste attraverso il pattern Dependency Injection (vedere la sezione \ref{DesignPattern} per ulteriori informazioni)
  \item[] \textbf{Relazioni con altri componenti:} estende \textit{premi/client/editor/GObject}
\end{itemize}
\subsubsection{premi/client/editor/lib/Image}
\begin{itemize}
  \item[] \textbf{Nome:} Image
  \item[] \textbf{Tipo:} classe
  \item[] \textbf{Package:} premi/client/editor/lib
  \item[] \textbf{Descrizione:} rappresenta un'immagine nella presentazione. Verrà inserito in un servizio factory$_G$ di AngularJS$_G$ che genererà istanze della classe quando esse saranno richieste attraverso il pattern Dependency Injection (vedere la sezione \ref{DesignPattern} per ulteriori informazioni)
  \item[] \textbf{Relazioni con altri componenti:} estende \textit{premi/client/editor/GObject}
\end{itemize}
\subsubsection{premi/client/editor/lib/Shape}
\begin{itemize}
  \item[] \textbf{Nome:} Shape
  \item[] \textbf{Tipo:} classe
  \item[] \textbf{Package:} premi/client/editor/lib
  \item[] \textbf{Descrizione:} rappresenta una figura nella presentazione. Uno shape può avere forme diverse come un quadrato, un cerchio, una freccia. Può diventare un elemento di abbellimento o di aumento dell' informazione che si vuole rappresentare. . Verrà inserito in un servizio factory$_G$ di AngularJS$_G$ che genererà istanze della classe quando esse saranno richieste attraverso il pattern Dependency Injection (vedere la sezione \ref{DesignPattern} per ulteriori informazioni)
  \item[] \textbf{Relazioni con altri componenti:} estende \textit{premi/client/editor/GObject}
\end{itemize}
\subsubsection{premi/client/editor/lib/Frame}
\begin{itemize}
  \item[] \textbf{Nome:} Frame
  \item[] \textbf{Tipo:} classe
  \item[] \textbf{Package:} premi/client/editor/lib
  \item[] \textbf{Descrizione:} rappresenta un frame$_G$ nella presentazione. Verrà inserito in un servizio factory$_G$ di AngularJS$_G$ che genererà istanze della classe quando esse saranno richieste attraverso il pattern Dependency Injection (vedere la sezione \ref{DesignPattern} per ulteriori informazioni)
  \item[] \textbf{Relazioni con altri componenti:} estende \code{premi/client/editor/GObject} e contiene un insieme di oggetti \code{premi/client/editor/GObject} che rappresentano il testo, le immagini e gli shape di una slide. Nonostante la classe frame$_G$ estenda GObject, un frame$_G$ non può contenere altri frame$_G$. Tuttavia si è scelto di lasciare che un frame$_G$ possa contenere ogni tipo derivato da GObject per l'eventualità futura di dare nuove funzionalità alla classe. Dipende anche da: 
\begin{itemize} 
	\item \textit{premi/client/editor/lib/Image} per interagire con gli oggetti di tipo image;
	\item \textit{premi/client/editor/lib/Text} per interagire con gli oggetti di tipo testo;
	\item \textit{premi/client/editor/lib/Shape} per interagire con gli oggetti di tipo shape;
	\item \textit{premi/client/editor/lib/Saver} per interagire con il database e salvare e modificare gli oggetti di un frame;
\end{itemize}  
\end{itemize}
\subsubsection{premi/client/editor/lib/Infographic}
\begin{itemize}
  \item[] \textbf{Nome:} Infographic
  \item[] \textbf{Tipo:} classe
  \item[] \textbf{Package:} premi/client/editor/lib
  \item[] \textbf{Descrizione:} rappresenta l'infografica di una presentazione. Verrà inserito in un servizio factory$_G$ di AngularJS$_G$ che genererà istanze della classe quando esse saranno richieste attraverso il pattern Dependency Injection (vedere la sezione \ref{DesignPattern} per ulteriori informazioni)
  \item[] \textbf{Relazioni con altri componenti:} estende \code{premi/client/editor/gOContainer} e dipende da:
\begin{itemize} 
	\item \textit{premi/client/editor/lib/frame} per interagire con gli oggetti di tipo frame;
	\item \textit{premi/client/editor/lib/saver} per interagire con il database, salvare e modificare gli oggetti dell'infografica;
\end{itemize}  
\end{itemize}
\subsubsection{premi/client/editor/lib/Saver}
\begin{itemize}
  \item[] \textbf{Nome:} Saver
  \item[] \textbf{Tipo:} classe
  \item[] \textbf{Package:} premi/client/editor/lib
  \item[] \textbf{Descrizione:} classe che permette di effettuare le modifiche degli oggetti sul database
  \item[] \textbf{Relazioni con altri componenti:} Dipende da:
\begin{itemize} 
	\item \textit{premi/client/presentation/lib/databaseAPI} per interagire con il database.
\end{itemize}  
\end{itemize}
\subsubsection{premi/client/editor/views/editor.ng}
\begin{itemize}
  \item[] \textbf{Nome:} editor.ng
  \item[] \textbf{Tipo:} template
  \item[] \textbf{Package:} premi/client/editor/views
  \item[] \textbf{Descrizione:} Template che fornisce uno scheletro per le altre viste dedicate alla gestione dell'editor.
\end{itemize}
\subsubsection{premi/client/editor/views/basicToolbar.ng}
\begin{itemize}
  \item[] \textbf{Nome:} basicToolbar.ng
  \item[] \textbf{Tipo:} template
  \item[] \textbf{Package:} premi/client/editor/views
  \item[] \textbf{Descrizione:} template della parte di pagina che mostra la toolbar che contiene il menù per spostarsi da un editor all'altro.
\end{itemize}
\subsubsection{premi/client/editor/controllers/editorCtrl}
\begin{itemize}
  \item[] \textbf{Nome:} editorCtrl
  \item[] \textbf{Tipo:} controller
  \item[] \textbf{Package:} premi/client/editor/controllers
  \item[] \textbf{Descrizione:} controller di premi/client/editor/views/editor.ng.
\end{itemize}
\subsubsection{premi/client/editor/controllers/basicToolbarCtrl}
\begin{itemize}
  \item[] \textbf{Nome:} basicToolbarCtrl
  \item[] \textbf{Tipo:} controller
  \item[] \textbf{Package:} premi/client/editor/controllers
  \item[] \textbf{Descrizione:} controller di premi/client/editor/views/basicToolbar.ng.
  \item[] \textbf{Relazioni con altri componenti:} modella lo \textit{\$scope} per interagire con la view generata da premi/client/editor/views/basicToolbarCtrl.ng
\end{itemize}


\subsection{premi/client/frameEditor}
\begin{figure}[!h]
\begin{center}
\includegraphics[scale=0.45]{img/diapkg/frameEditor.png}
\caption{Diagramma del package premi/client/frameEditor}
\end{center}
\end{figure}
\subsubsection{premi/client/frameEditor/views/frame.ng}
\begin{itemize}
  \item[] \textbf{Nome:} frame.ng
  \item[] \textbf{Tipo:} template
  \item[] \textbf{Package:} premi/client/frameEditor/views
  \item[] \textbf{Descrizione:} template della parte di pagina che permette la modifica dei Frame$_G$ e degli oggetti in esso contenuti.
\end{itemize}
\subsubsection{premi/client/frameEditor/controllers/frameEditorCtrl}
\begin{itemize}
  \item[] \textbf{Nome:} FrameEditorCtrl
  \item[] \textbf{Tipo:} controller
  \item[] \textbf{Package:} premi/client/frameEditor/controllers
  \item[] \textbf{Descrizione:} controller di premi/client/frameEditor/views/toolbar.ng e di premi/client/frameEditor/views/frame.ng
  \item[] \textbf{Relazioni con altri componenti:} modella lo \textit{\$scope} per interagire con le views generate da premi/client/frameEditor/views/toolbar.ng e premi/client/frameEditor/views/frame.ng dipende anche da:
 \begin{itemize} 
	\item \textit{premi/client/presentation/lib/databaseAPI} per interagire con il database;  
	\item \textit{premi/client/editor/lib/interactInit} per interagire con la libreria Interactjs;
	\item \textit{premi/client/editor/lib/frame} per interagire con la libreria frame e usare i metodi per la modifica, aggiunta, cancellazione di un frame;
	\item \textit{premi/client/editor/lib/Observer} per interagire con la libreria observer; 
	\item \textit{premi/presentation/lib/orderedGOList} per interagire con la libreria orderedGOList per ordinare la lista dei frame. 
  \end{itemize} 
\end{itemize}

\clearpage
\subsection{premi/client/infographicEditor}
\begin{figure}[!h]
\begin{center}
\includegraphics[scale=0.45]{img/diapkg/infographicEditor.png}
\caption{Diagramma del package premi/client/infographicEditor}
\end{center}
\end{figure}
\subsubsection{premi/client/infographicEditor/views/infographic.ng}
\begin{itemize}
  \item[] \textbf{Nome:} infographic.ng
  \item[] \textbf{Tipo:} template
  \item[] \textbf{Package:} premi/client/infographicEditor/views
  \item[] \textbf{Descrizione:}  template della parte di pagina per la creazione o modifica dell'infografica$_G$
\end{itemize}

\subsubsection{premi/client/infographicEditor/controllers/infographicEditorCtrl}
\begin{itemize}
  \item[] \textbf{Nome:} infographicEditorCtrl
  \item[] \textbf{Tipo:} controller
  \item[] \textbf{Package:} premi/client/infographicEditor/controllers
  \item[] \textbf{Descrizione:} controller di premi/client/infographicEditor/views/frameList.ng e premi/client/infographicEditor/views/infographic.ng
  \item[] \textbf{Relazioni con altri componenti:} modella lo \textit{\$scope} per interagire con le views generate da premi.client.InfographicEditor.views.frame.ng e di premi.client.InfographicEditor.views.toolbar.ng e dipende da: 
  \begin{itemize}  
  \item[] \textit{premi/client/presentation/lib/databaseAPI} per interagire con il database;
  \item[] \textit{premi/client/editor/lib/interactInit} per interagire con la libreria interactjs;
  \item[] \textit{premi/client/editor/lib/infographic} per interagire con la libreria infografica e usare i metodi per la gestione dell'infografica;
  \item[] \textit{premi/client/editor/lib/observer} per interagire con la libreria observer;
  \item[] \textit{premi/presentation/lib/orderedGOList} per interagire con la libreria orderedGOList per ordinare la lista dei frame. 
  \end{itemize}
\end{itemize}

\subsection{premi/client/trailsEditor}
\begin{figure}[!h]
\begin{center}
\includegraphics[scale=0.45]{img/diapkg/trailsEditor.png}
\caption{Diagramma del package premi/client/trailsEditor}
\end{center}
\end{figure}
\subsubsection{premi/client/trailsEditor/views/editTrail.ng}
\begin{itemize}
  \item[] \textbf{Nome:} editTrail.ng
  \item[] \textbf{Tipo:} template
  \item[] \textbf{Package:} premi/client/trailsEditor/views
  \item[] \textbf{Descrizione:}  template della parte di pagina per la modifica del titolo di un trail$_G$
\end{itemize}
\subsubsection{premi/client/trailsEditor/views/listTrail.ng}
\begin{itemize}
  \item[] \textbf{Nome:} listTrail.ng
  \item[] \textbf{Tipo:} template
  \item[] \textbf{Package:} premi/client/trailsEditor/views
  \item[] \textbf{Descrizione:}  template della parte di pagina per la visualizzazione della lista dei trail$_G$ esistenti
\end{itemize}
\subsubsection{premi/client/trailsEditor/views/modTrail.ng}
\begin{itemize}
  \item[] \textbf{Nome:} modTrail.ng
  \item[] \textbf{Tipo:} template
  \item[] \textbf{Package:} premi/client/trailsEditor/views
  \item[] \textbf{Descrizione:}  template della parte di pagina per la modifica di un trail$_G$
\end{itemize}
\subsubsection{premi/client/trailsEditor/views/newTrail.ng}
\begin{itemize}
  \item[] \textbf{Nome:} newTrail.ng
  \item[] \textbf{Tipo:} template
  \item[] \textbf{Package:} premi/client/trailsEditor/views
  \item[] \textbf{Descrizione:}  template della parte di pagina per l'aggiunta un nuovo trail$_G$
\end{itemize}
\subsubsection{premi/client/trailsEditor/views/removeTrail.ng}
\begin{itemize}
  \item[] \textbf{Nome:} removeTrail.ng
  \item[] \textbf{Tipo:} template
  \item[] \textbf{Package:} premi/client/trailsEditor/views
  \item[] \textbf{Descrizione:}  template della parte di pagina per la rimozione di un trail$_G$
\end{itemize}
\subsubsection{premi/client/trailsEditor/views/removeChkPnt.ng}
\begin{itemize}
  \item[] \textbf{Nome:} removeChkPnt.ng
  \item[] \textbf{Tipo:} template
  \item[] \textbf{Package:} premi/client/trailsEditor/views
  \item[] \textbf{Descrizione:}  template della parte di pagina per la rimozione di un checkpoint in un trail$_G$
\end{itemize}
\subsubsection{premi/client/trailsEditor/controllers/editTrailCtrl}
\begin{itemize}
  \item[] \textbf{Nome:} editTrailCtrl
  \item[] \textbf{Tipo:} controller
  \item[] \textbf{Package:} premi/client/trailsEditor/controllers
  \item[] \textbf{Descrizione:} controller di premi/client/trailsEditor/views/editTrail.ng
  \item[] \textbf{Relazioni con altri componenti:} modella lo \textit{\$scope} per interagire con la view generata da premi/client/trailsEditor/views/editTrail.ng e dipende da:   
  \begin{itemize}
  \item[] \textit{premi/client/presentation/lib/databaseAPI} per interagire con il database;    
  \end{itemize}
\end{itemize}
\subsubsection{premi/client/trailsEditor/controllers/listTrailCtrl}
\begin{itemize}
  \item[] \textbf{Nome:} listTrailCtrl
  \item[] \textbf{Tipo:} controller
  \item[] \textbf{Package:} premi/client/trailsEditor/controllers
  \item[] \textbf{Descrizione:} controller di premi/client/trailsEditor/views/listTrail.ng
  \item[] \textbf{Relazioni con altri componenti:} modella lo \textit{\$scope} per interagire con la view generata da premi/client/trailsEditor/views/listTrail.ng
\end{itemize}
\subsubsection{premi/client/trailsEditor/controllers/modTrailCtrl}
\begin{itemize}
  \item[] \textbf{Nome:} modTrailCtrl
  \item[] \textbf{Tipo:} controller
  \item[] \textbf{Package:} premi/client/trailsEditor/controllers
  \item[] \textbf{Descrizione:} controller di premi/client/trailsEditor/views/modTrail.ng
  \item[] \textbf{Relazioni con altri componenti:} modella lo \textit{\$scope} per interagire con la view generata da premi/client/trailsEditor/views/modTrail.ng e dipende da:
  \begin{itemize}
  	\item \textit{premi/client/presentation/lib/trail} per interagire con i metodi per gestire un trail.
  \end{itemize}
\end{itemize}
\subsubsection{premi/client/trailsEditor/controllers/newTrailCtrl}
\begin{itemize}
  \item[] \textbf{Nome:} newTrailCtrl
  \item[] \textbf{Tipo:} controller
  \item[] \textbf{Package:} premi/client/trailsEditor/controllers
  \item[] \textbf{Descrizione:} controller di premi/client/trailsEditor/views/newTrail.ng
  \item[] \textbf{Relazioni con altri componenti:} modella lo \textit{\$scope} per interagire con la view generata da premi/client/trailsEditor/views/newTrail.ng e dipende da:
  \begin{itemize}
  	\item \textit{premi/client/presentation/lib/databaseAPI} per interagire con il database.
  \end{itemize}
\end{itemize}
\subsubsection{premi/client/trailsEditor/controllers/removeTrailCtrl}
\begin{itemize}
  \item[] \textbf{Nome:} removeTrailCtrl
  \item[] \textbf{Tipo:} controller
  \item[] \textbf{Package:} premi/client/trailsEditor/controllers
  \item[] \textbf{Descrizione:} controller di premi/client/trailsEditor/views/removeTrail.ng
  \item[] \textbf{Relazioni con altri componenti:} modella lo \textit{\$scope} per interagire con la view generata da premi/client/trailsEditor/views/removeTrail.ng e dipende da:
  \begin{itemize}
  	\item \textit{premi/client/presentation/lib/databaseAPI} per interagire con il database.
  \end{itemize}
\end{itemize}
\subsubsection{premi/client/trailsEditor/controllers/trailsEditorCtrl}
\begin{itemize}
  \item[] \textbf{Nome:} trailsEditorCtrl
  \item[] \textbf{Tipo:} controller
  \item[] \textbf{Package:} premi/client/trailsEditor/controllers
  \item[] \textbf{Descrizione:} controller generale che gestisce le view di premi/client/trailsEditor/views
\end{itemize}
\subsubsection{premi/client/trailsEditor/controllers/removeChkPnt}
\begin{itemize}
  \item[] \textbf{Nome:} removeChkPnt
  \item[] \textbf{Tipo:} controller
  \item[] \textbf{Package:} premi/client/trailsEditor/controllers
  \item[] \textbf{Descrizione:} controller di premi/client/trailsEditor/views/removeChkPnt.ng
  \item[] \textbf{Relazioni con altri componenti:} modella lo \textit{\$scope} per interagire con la view generata da premi/client/trailsEditor/views/removeChkPnt.ng
\end{itemize}

\subsection{premi/client/viewer}
\begin{figure}[!h]
\begin{center}
\includegraphics[scale=0.45]{img/diapkg/viewer.png}
\caption{Diagramma del package premi/client/viewer}
\end{center}
\end{figure}
\subsubsection{premi/client/viewer/views/trails.ng}
\begin{itemize}
  \item[] \textbf{Nome:} trails.ng
  \item[] \textbf{Tipo:} template
  \item[] \textbf{Package:} premi/client/viewer/views
  \item[] \textbf{Descrizione:} template del viewer che permette di visualizzare i trail disponibili per la scelta del percorso da visualizzare
  \item[] \textbf{Relazioni con altri componenti:} è la vista di premi/client/viewer/views/trailsCtrl
\end{itemize}
\subsubsection{premi/client/viewer/views/viewer.ng}
\begin{itemize}
  \item[] \textbf{Nome:} viewer.ng
  \item[] \textbf{Tipo:} template
  \item[] \textbf{Package:} premi/client/viewer/views
  \item[] \textbf{Descrizione:} template del viewer che permette di visualizzare la presentazione.
  \item[] \textbf{Relazioni con altri componenti:} è la vista dedicata di premi/client/viewer/views/viewerCtrl e dipende da:
  \begin{itemize}
  	\item \textit{Impress} libreria esterna che interagisce con l'HTML della vista per creare una presentazione
  \end{itemize}
\end{itemize}
\subsubsection{premi/client/viewer/controllers/trailsCtrl}
\begin{itemize}
  \item[] \textbf{Nome:} trailsCtrl
  \item[] \textbf{Tipo:} controller
  \item[] \textbf{Package:} premi/client/viewer/controllers
  \item[] \textbf{Descrizione:} controller di premi/client/viewer/views/trails.ng per fornire le funzionalità per la visualizzazione dei trails
  \item[] \textbf{Relazioni con altri componenti:} modella lo \textit{\$scope} per interagire con la view generata da premi/client/viewer/views/trails.ng
\end{itemize}
\subsubsection{premi/client/viewer/controllers/viewerCtrl}
\begin{itemize}
  \item[] \textbf{Nome:} viewerCtrl
  \item[] \textbf{Tipo:} controller
  \item[] \textbf{Package:} premi/client/viewer/controllers
  \item[] \textbf{Descrizione:} controller di premi/client/userManager/views/viewer.ng fornisce le funzionalità per la visualizzazione della presentazione
  \item[] \textbf{Relazioni con altri componenti:} modella lo \textit{\$scope} per interagire con la view generata da premi/client/viewer/views/viewer.ng per visualizzare la presentazione
\end{itemize}

\subsection{Premi/client/trailMap}
\begin{figure}[!h]
\begin{center}
\includegraphics[scale=0.45]{img/diapkg/trailMap.png}
\caption{Diagramma del package premi/client/trailMap}
\end{center}
\end{figure}
\subsubsection{Premi/client/trailMap/views/trailMap.ng}
\begin{itemize}
  \item[] \textbf{Nome:} trailMap.ng
  \item[] \textbf{Tipo:} template
  \item[] \textbf{Package:} Premi/client/trailMap/views
  \item[] \textbf{Descrizione:}  template che permette di visualizzare e modificare il percorso di un trail.
\end{itemize}
\subsubsection{Premi/client/trailMap/controllers/trailMapCtrl}
\begin{itemize}
  \item[] \textbf{Nome:} trailMapCtrl
  \item[] \textbf{Tipo:} controller
  \item[] \textbf{Package:} Premi/client/trailMap/controllers
  \item[] \textbf{Descrizione:} controller di premi/client/trailMapCtrl/views/trailMap.ng
  \item[] \textbf{Relazioni con altri componenti:} modella lo \textit{\$scope} per interagire con la view generata da Premi/client/trailMap/views/trailMap.ng e dipende da:
  \begin{itemize}
  	\item \textit{Premi/client/presentation/lib/trail} per interagire con gli oggetti trail;
  	\item \textit{Premi/client/presentation/lib/orderedGOList} per ordinare gli oggetti grafici;
  	\item \textit{Premi/client/presentation/lib/signalCtrl} per gestire i vari stati;
  \end{itemize}
\end{itemize}















