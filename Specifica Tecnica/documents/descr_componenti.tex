\section{Descrizione dei singoli componenti}
	Tipo, obiettivo e funzione del componente \\
	Relazioni d'uso di altre componenti \\
	Interfacce con le relazioni di uso da altre componenti \\
\subsection{Premi}
-----------------------------premi-class.jpg

----template------
\subsubsection{Premi.}
\begin{itemize}
  \item \textbf{Nome:}
  \item \textbf{Tipo:} class
  \item \textbf{Package:} Premi.
  \item \textbf{Descrizione:}
  \item \textbf{Relazioni con altri componenti:} 
\end{itemize}
---/template-----


\subsubsection{Premi.Utility}
\begin{itemize}
  \item[] \textbf{Nome:} Utility
  \item[] \textbf{Tipo:} class
  \item[] \textbf{Package:} Premi
  \item[] \textbf{Descrizione:} classe statica di utilità che fornisce strumenti per interagire con la base di dati
\end{itemize}

\subsubsection{Premi.View}
\begin{itemize}
  \item \textbf{Nome:} View
  \item \textbf{Tipo:} class
  \item \textbf{Package:} Premi
  \item \textbf{Descrizione:} view generale che funge da sfondo per le altre view
\end{itemize}

\subsubsection{Premi.PremiCtrl}
\begin{itemize}
  \item \textbf{Nome:} PremiCtrl
  \item \textbf{Tipo:} class
  \item \textbf{Package:} Premi
  \item \textbf{Descrizione:} controller generale che modifica la view principale in base alle scelte dell'utente
  \item \textbf{Relazioni con altri componenti:} la classe altera l'aspetto di \code{Premi.View} caricando le view di cui l'utente ha bisogno 
\end{itemize}

\subsection{Premi.UserManager}
-----------------------------premi-class.jpg
\subsubsection{Premi.UserManager.User}
-----------------------------UserManager-class.jpg (mostra le relazioni che possiede ESTERNE al package)
\begin{itemize}
  \item \textbf{Nome:} User
  \item \textbf{Tipo:} class
  \item \textbf{Package:} Premi.UserManger
  \item \textbf{Descrizione:} classe che definisce un utente e fornisce le funzionalità necessarie alla sua creazione, al suo login e ad un eventuale cambio di password
  \item \textbf{Relazioni con altri componenti:} si collega a \code{Premi.Utility} per le interazioni con la base di dati
\end{itemize}

\subsubsection{Premi.UserManager.View}
\begin{itemize}
  \item \textbf{Nome:} View
  \item \textbf{Tipo:} class
  \item \textbf{Package:} Premi.UserManger
  \item \textbf{Descrizione:} vista generale del package \code{UserManager}. Puo' contenere le view abbinate alle funzionalità di \code{User}
\end{itemize}
\subsubsection{Premi.UserManager.UserManagerCtrl}
\begin{itemize}
  \item \textbf{Nome:} UserManagerCtrl
  \item \textbf{Tipo:} class
  \item \textbf{Package:} Premi.UserManager
  \item \textbf{Descrizione:} controller generale del package \code{UserManager} dedicato a fornire alla view strumenti per l'interazione con l'utente
  \item \textbf{Relazioni con altri componenti:} si collega a \code{Premi.UserManager.View} per mostrare i dati dell'utente che va a prelevare tramite {Premi.UserManager.User}
\end{itemize}
\subsubsection{Premi.UserManager.LoginView}
\begin{itemize}
  \item \textbf{Nome:} LoginView
  \item \textbf{Tipo:} class
  \item \textbf{Package:} Premi.UserManager
  \item \textbf{Descrizione:} view che permette all'utente di effettuare la login
\end{itemize}
\subsubsection{Premi.UserManager.LoginController}
\begin{itemize}
  \item \textbf{Nome:} LoginController
  \item \textbf{Tipo:} class
  \item \textbf{Package:} Premi.UserManager
  \item \textbf{Descrizione:} fornisce gli strumenti necessari alla view per effettuare la login
  \item \textbf{Relazioni con altri componenti:} con \code{Premi.UserManager.LoginView} e con \code{Premi.UserManager.User} per trasmettere le informazioni relative al login
\end{itemize}
\subsubsection{Premi.UserManager.RegistrationView}
\begin{itemize}
  \item \textbf{Nome:} RegistrationView
  \item \textbf{Tipo:} class
  \item \textbf{Package:} Premi.UserManager
  \item \textbf{Descrizione:} view che permette all'utente di registrarsi nel sistema
\end{itemize}
\subsubsection{Premi.UserManager.RegistrationController}
\begin{itemize}
  \item \textbf{Nome:} RegistrationController
  \item \textbf{Tipo:} class
  \item \textbf{Package:} Premi.UserManager
  \item \textbf{Descrizione:} fornisce gli strumenti necessari alla view per registrare l'utente
  \item \textbf{Relazioni con altri componenti:} con \code{Premi.UserManager.RegistrationView} e con \code{Premi.UserManager.User} per trasmettere le informazioni relative alla registrazione
\end{itemize}
\subsubsection{Premi.UserManager.ChangePasswordView}
\begin{itemize}
  \item \textbf{Nome:} ChangePasswordView
  \item \textbf{Tipo:} class
  \item \textbf{Package:} Premi.UserManager
  \item \textbf{Descrizione:} view che permette all'utente di cambiare la propria password
\end{itemize}
\subsubsection{Premi.UserManager.ChangePasswordController}
\begin{itemize}
  \item \textbf{Nome:} ChangePassword
  \item \textbf{Tipo:} class
  \item \textbf{Package:} Premi.UserManager
  \item \textbf{Descrizione:} fornisce gli strumenti necessari alla view per cambiare la password dell'utente
  \item \textbf{Relazioni con altri componenti:} con \code{Premi.UserManager.ChangePasswordView} e con \code{Premi.UserManager.User} per trasmettere le informazioni relative al cambio password
\end{itemize}

\subsection{Premi.Viewer}
---------------------------viewer-class.jpg
\subsubsection{Premi.Viewer.View}
\begin{itemize}
  \item \textbf{Nome:} View
  \item \textbf{Tipo:} class
  \item \textbf{Package:} Premi.Viewer
  \item \textbf{Descrizione:} view che mostra la presentazione all'utente, offrendo funzionalità dipendenti dalla visibilità(pubblica o privata) o dal contesto in cui si trova(presentazione live) 
\end{itemize}
\subsubsection{Premi.ViewerCtrl}
\begin{itemize}
  \item \textbf{Nome:} ViewerCtrl
  \item \textbf{Tipo:} class
  \item \textbf{Package:} Premi.Viewer
  \item \textbf{Descrizione:} abilita funzionalità nella view in base all'utente
  \item \textbf{Relazioni con altri componenti:} con \code{Premi.Viewer.View}, con \code{Premi.UserManager.User} per verificare se l'utente è il proprietario della presentazione, e con la libreria esterna \code{ImpressJs} per aggiungere animazioni alla presentazione
\end{itemize}
\subsection{Premi.Presentation}
-------------------presentation-class.jpg
\subsubsection{Premi.Presentation.GraphicObject}
\begin{itemize}
  \item \textbf{Nome:} GraphicObject
  \item \textbf{Tipo:} \textit{abstract class}
  \item \textbf{Package:} Premi.Presentation
  \item \textbf{Descrizione:} rappresenta gli oggetti grafici nella presentazione
\end{itemize}
\subsubsection{Premi.Presentation.GoContent}
\begin{itemize}
  \item \textbf{Nome:} GoContent
  \item \textbf{Tipo:} \textit{abstract class}
  \item \textbf{Package:} Premi.Presentation
  \item \textbf{Descrizione:} rappresenta gli oggetti grafici con contenuto di una presentazione
  \item \textbf{Relazioni con altri componenti:} estende \code{Premi.Presentation.GraphicObject}
\end{itemize}
\subsubsection{Premi.Presentation.Text}
\begin{itemize}
  \item \textbf{Nome:} Text
  \item \textbf{Tipo:} class
  \item \textbf{Package:} Premi.Presentation
  \item \textbf{Descrizione:} rappresenta un'area di testo nella presentazione
    \item \textbf{Relazioni con altri componenti:} estende \code{Premi.Presentation.GoContent}
\end{itemize}
\subsubsection{Premi.Presentation.Image}
\begin{itemize}
  \item \textbf{Nome:} Image
  \item \textbf{Tipo:} class
  \item \textbf{Package:} Premi.Presentation
  \item \textbf{Descrizione:} rappresenta un'immagine nella presentazione
      \item \textbf{Relazioni con altri componenti:} estende \code{Premi.Presentation.GoContent}
\end{itemize}
\subsubsection{Premi.Presentation.Shape}
\begin{itemize}
  \item \textbf{Nome:} Shape
  \item \textbf{Tipo:} class
  \item \textbf{Package:} Premi.Presentation
  \item \textbf{Descrizione:} rappresenta una figura nella presentazione
      \item \textbf{Relazioni con altri componenti:} estende \code{Premi.Presentation.GoContent}
\end{itemize}
\subsubsection{Premi.Presentation.Frame}
\begin{itemize}
  \item \textbf{Nome:} Frame
  \item \textbf{Tipo:} class
  \item \textbf{Package:} Premi.Presentation
  \item \textbf{Descrizione:} rappresenta un frame nella presentazine
  \item \textbf{Relazioni con altri componenti:} estende \code{Premi.Presentation.GraphicObject} e possiede un insieme di oggetti  \code{Premi.Presentation.GoContent}
\end{itemize}
\subsubsection{Premi.Presentation.Trail}
\begin{itemize}
  \item \textbf{Nome:} Trail
  \item \textbf{Tipo:} class
  \item \textbf{Package:} Premi.Presentation
  \item \textbf{Descrizione:} rappresenta un percorso nella presentazione
  \item \textbf{Relazioni con altri componenti:} possiede una lista di riferimenti ad oggetti di tipo \code{Premi.Presentation.Frame}, ma non è in possesso degli oggetti stessi
\end{itemize}
\subsubsection{Premi.Presentation.Presentation}
\begin{itemize}
  \item \textbf{Nome:} Presentation
  \item \textbf{Tipo:} class
  \item \textbf{Package:} Premi.Presentation
  \item \textbf{Descrizione:} rappresenta una presentazione
  \item \textbf{Relazioni con altri componenti:} possiede una lista di oggetti \code{premi.Presentation.Frame}
\end{itemize}









