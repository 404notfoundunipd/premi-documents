\section{Descrizione dei singoli componenti}
	Tipo, obiettivo e funzione del componente \\
	Relazioni d'uso di altre componenti \\
	Interfacce con le relazioni di uso da altre componenti \\
\subsection{Premi}
-----------------------------premi-class.jpg


\subsubsection{Premi.Utility}
\begin{itemize}
  \item[] \textbf{Nome:} Utility
  \item[] \textbf{Tipo:} class
  \item[] \textbf{Package:} Premi
  \item[] \textbf{Descrizione:} classe statica di utilità che fornisce strumenti per interagire con la base di dati
\end{itemize}

\subsubsection{Premi.View}
\begin{itemize}
  \item \textbf{Nome:} View
  \item \textbf{Tipo:} class
  \item \textbf{Package:} Premi
  \item \textbf{Descrizione:} view generale che funge da sfondo per le altre view
\end{itemize}

\subsubsection{Premi.PremiCtrl}
\begin{itemize}
  \item \textbf{Nome:} PremiCtrl
  \item \textbf{Tipo:} class
  \item \textbf{Package:} Premi
  \item \textbf{Descrizione:} controller generale che modifica la view principale in base alle scelte dell'utente
  \item \textbf{Relazioni con altri componenti:} la classe altera l'aspetto di \code{Premi.View} caricando le view di cui l'utente ha bisogno 
\end{itemize}

\subsection{Premi.UserManager}
-----------------------------premi-class.jpg
\subsubsection{Premi.UserManager.User}
-----------------------------UserManager-class.jpg (mostra le relazioni che possiede ESTERNE al package)
\begin{itemize}
  \item \textbf{Nome:} User
  \item \textbf{Tipo:} class
  \item \textbf{Package:} Premi.UserManger
  \item \textbf{Descrizione:} classe che definisce un utente e fornisce le funzionalità necessarie alla sua creazione, al suo login e ad un eventuale cambio di password
  \item \textbf{Relazioni con altri componenti:} si collega a \code{Premi.Utility} per le interazioni con la base di dati
\end{itemize}

\subsubsection{Premi.UserManager.View}
\begin{itemize}
  \item \textbf{Nome:} View
  \item \textbf{Tipo:} class
  \item \textbf{Package:} Premi.UserManger
  \item \textbf{Descrizione:} vista generale del package \code{UserManager}. Puo' contenere le view abbinate alle funzionalità di \code{User}
\end{itemize}
\subsubsection{Premi.UserManager.UserManagerCtrl}
\begin{itemize}
  \item \textbf{Nome:} UserManagerCtrl
  \item \textbf{Tipo:} class
  \item \textbf{Package:} Premi.UserManager
  \item \textbf{Descrizione:} controller generale del package \code{UserManager} dedicato a fornire alla view strumenti per l'interazione con l'utente
  \item \textbf{Relazioni con altri componenti:} si collega a \code{Premi.UserManager.View} per mostrare i dati dell'utente che va a prelevare tramite {Premi.UserManager.User}
\end{itemize}
\subsubsection{Premi.UserManager.LoginView}
\begin{itemize}
  \item \textbf{Nome:} LoginView
  \item \textbf{Tipo:} class
  \item \textbf{Package:} Premi.UserManager
  \item \textbf{Descrizione:} view che permette all'utente di effettuare la login
\end{itemize}
\subsubsection{Premi.UserManager.LoginController}
\begin{itemize}
  \item \textbf{Nome:} LoginController
  \item \textbf{Tipo:} class
  \item \textbf{Package:} Premi.UserManager
  \item \textbf{Descrizione:} fornisce gli strumenti necessari alla view per effettuare la login
  \item \textbf{Relazioni con altri componenti:} con \code{Premi.UserManager.LoginView} e con \code{Premi.UserManager.User} per trasmettere le informazioni relative al login
\end{itemize}
\subsubsection{Premi.UserManager.RegistrationView}
\begin{itemize}
  \item \textbf{Nome:} RegistrationView
  \item \textbf{Tipo:} class
  \item \textbf{Package:} Premi.UserManager
  \item \textbf{Descrizione:} view che permette all'utente di registrarsi nel sistema
\end{itemize}
\subsubsection{Premi.UserManager.RegistrationController}
\begin{itemize}
  \item \textbf{Nome:} RegistrationController
  \item \textbf{Tipo:} class
  \item \textbf{Package:} Premi.UserManager
  \item \textbf{Descrizione:} fornisce gli strumenti necessari alla view per registrare l'utente
  \item \textbf{Relazioni con altri componenti:} con \code{Premi.UserManager.RegistrationView} e con \code{Premi.UserManager.User} per trasmettere le informazioni relative alla registrazione
\end{itemize}
\subsubsection{Premi.UserManager.ChangePasswordView}
\begin{itemize}
  \item \textbf{Nome:} ChangePasswordView
  \item \textbf{Tipo:} class
  \item \textbf{Package:} Premi.UserManager
  \item \textbf{Descrizione:} view che permette all'utente di cambiare la propria password
\end{itemize}
\subsubsection{Premi.UserManager.ChangePasswordController}
\begin{itemize}
  \item \textbf{Nome:} ChangePassword
  \item \textbf{Tipo:} class
  \item \textbf{Package:} Premi.UserManager
  \item \textbf{Descrizione:} fornisce gli strumenti necessari alla view per cambiare la password dell'utente
  \item \textbf{Relazioni con altri componenti:} con \code{Premi.UserManager.ChangePasswordView} e con \code{Premi.UserManager.User} per trasmettere le informazioni relative al cambio password
\end{itemize}

\subsection{Premi.Viewer}
---------------------------viewer-class.jpg
\subsubsection{Premi.Viewer.View}
\begin{itemize}
  \item \textbf{Nome:} View
  \item \textbf{Tipo:} class
  \item \textbf{Package:} Premi.Viewer
  \item \textbf{Descrizione:} view che mostra la presentazione all'utente, offrendo funzionalità dipendenti dalla visibilità(pubblica o privata) o dal contesto in cui si trova(presentazione live) 
\end{itemize}
\subsubsection{Premi.ViewerCtrl}
\begin{itemize}
  \item \textbf{Nome:} ViewerCtrl
  \item \textbf{Tipo:} class
  \item \textbf{Package:} Premi.Viewer
  \item \textbf{Descrizione:} abilita funzionalità nella view in base all'utente
  \item \textbf{Relazioni con altri componenti:} con \code{Premi.Viewer.View}, con \code{Premi.UserManager.User} per verificare se l'utente è il proprietario della presentazione, e con la libreria esterna \code{ImpressJs} per aggiungere animazioni alla presentazione
\end{itemize}
\subsection{Premi.Presentation}
-------------------presentation-class.jpg
\subsubsection{Premi.Presentation.GraphicObject}
\begin{itemize}
  \item \textbf{Nome:} GraphicObject
  \item \textbf{Tipo:} \textit{abstract class}
  \item \textbf{Package:} Premi.Presentation
  \item \textbf{Descrizione:} rappresenta gli oggetti grafici nella presentazione
\end{itemize}
\subsubsection{Premi.Presentation.GoContent}
\begin{itemize}
  \item \textbf{Nome:} GoContent
  \item \textbf{Tipo:} \textit{abstract class}
  \item \textbf{Package:} Premi.Presentation
  \item \textbf{Descrizione:} rappresenta gli oggetti grafici con contenuto di una presentazione
  \item \textbf{Relazioni con altri componenti:} estende \code{Premi.Presentation.GraphicObject}
\end{itemize}
\subsubsection{Premi.Presentation.Text}
\begin{itemize}
  \item \textbf{Nome:} Text
  \item \textbf{Tipo:} class
  \item \textbf{Package:} Premi.Presentation
  \item \textbf{Descrizione:} rappresenta un'area di testo nella presentazione
    \item \textbf{Relazioni con altri componenti:} estende \code{Premi.Presentation.GoContent}
\end{itemize}
\subsubsection{Premi.Presentation.Image}
\begin{itemize}
  \item \textbf{Nome:} Image
  \item \textbf{Tipo:} class
  \item \textbf{Package:} Premi.Presentation
  \item \textbf{Descrizione:} rappresenta un'immagine nella presentazione
      \item \textbf{Relazioni con altri componenti:} estende \code{Premi.Presentation.GoContent}
\end{itemize}
\subsubsection{Premi.Presentation.Shape}
\begin{itemize}
  \item \textbf{Nome:} Shape
  \item \textbf{Tipo:} class
  \item \textbf{Package:} Premi.Presentation
  \item \textbf{Descrizione:} rappresenta una figura nella presentazione
      \item \textbf{Relazioni con altri componenti:} estende \code{Premi.Presentation.GoContent}
\end{itemize}
\subsubsection{Premi.Presentation.Frame}
\begin{itemize}
  \item \textbf{Nome:} Frame
  \item \textbf{Tipo:} class
  \item \textbf{Package:} Premi.Presentation
  \item \textbf{Descrizione:} rappresenta un frame nella presentazine
  \item \textbf{Relazioni con altri componenti:} estende \code{Premi.Presentation.GraphicObject} e possiede un insieme di oggetti  \code{Premi.Presentation.GoContent}
\end{itemize}
\subsubsection{Premi.Presentation.Trail}
\begin{itemize}
  \item \textbf{Nome:} Trail
  \item \textbf{Tipo:} class
  \item \textbf{Package:} Premi.Presentation
  \item \textbf{Descrizione:} rappresenta un percorso nella presentazione
  \item \textbf{Relazioni con altri componenti:} possiede una lista di riferimenti ad oggetti di tipo \code{Premi.Presentation.Frame}, ma non è in possesso degli oggetti stessi
\end{itemize}
-------------------------------dipendenze di presentation.jpg CITARE L'IMMAGINE DELLE DIPENDENZE CON DI PRESENTATION
Completo-Dettagliato.jpg
\subsubsection{Premi.Presentation.Presentation}
\begin{itemize}
  \item \textbf{Nome:} Presentation
  \item \textbf{Tipo:} class
  \item \textbf{Package:} Premi.Presentation
  \item \textbf{Descrizione:} rappresenta una presentazione
  \item \textbf{Relazioni con altri componenti:} possiede una lista di oggetti \code{premi.Presentation.GraphicObject} che possono essere sia Frame che oggetti grafici generici e assieme compongono l'infografica della presentazione, possiede inoltre una lista di percorsi e quindi di oggetti \code{Premi.Presentation.Trail}, identifica l'appartenenza di una presentazione ad un determinato \code{Premi.UserManager.User} e si collega alla base di dati tramite la classe statica \code{Premi.Utility}
\end{itemize}



\subsection{Premi.PresentationManager}
-------usermanager-classe.jpg
\subsubsection{Premi.PresentationManager.Export}
\begin{itemize}
  \item \textbf{Nome:} Export 
  \item \textbf{Tipo:} class
  \item \textbf{Package:} Premi.PresentationManager
  \item \textbf{Descrizione:} permette di esportare la presentazione in formato poster
  \item \textbf{Relazioni con altri componenti:} preleva le informazioni da \code{Premi.Presentation.Presentation} per costruire il poster
\end{itemize}
\subsubsection{Premi.PresentationManager.Portable}
\begin{itemize}
  \item \textbf{Nome:} Portable
  \item \textbf{Tipo:} class
  \item \textbf{Package:} Premi.PresentationManager
  \item \textbf{Descrizione:} permette di rendere portabile una presentazione, e quindi di essere indipendente dal software Premi e di essere caricata come pagina HTML da un browser
  \item \textbf{Relazioni con altri componenti:} 
\end{itemize}
\subsubsection{Premi.PresentationManager.View}
\begin{itemize}
  \item \textbf{Nome:} View
  \item \textbf{Tipo:} class
  \item \textbf{Package:} Premi.PresentationController
  \item \textbf{Descrizione:} è la view generale delle operazioni che l'utente puo' effettuare sul
\end{itemize}
\subsubsection{Premi.PresentationManager.PresentationManagerCtrl}
\begin{itemize}
  \item \textbf{Nome:} PresentationManagerCtrl
  \item \textbf{Tipo:} class
  \item \textbf{Package:} Premi.PresentationManager
  \item \textbf{Descrizione:} fornisce alla view gli strumenti necessari alla gestione delle presentazioni
  \item \textbf{Relazioni con altri componenti:} è il controller dedicato di  \code{Premi.PresentationManager.View} 
\end{itemize}
\subsubsection{Premi.PresentationManager.RemoveView}
\begin{itemize}
  \item \textbf{Nome:} RemoveView
  \item \textbf{Tipo:} class
  \item \textbf{Package:} Premi.PresentationManager
  \item \textbf{Descrizione:} mostra una finestra di conferma eliminazione della presentazione selezionata
\end{itemize}
\subsubsection{Premi.PresentationManager.RemoveController}
\begin{itemize}
  \item \textbf{Nome:} RemoveController
  \item \textbf{Tipo:} class
  \item \textbf{Package:} Premi.PresentationManager
  \item \textbf{Descrizione:} fornisce alla view gli strumenti per rimuovere la presentazione o annullare l'azione
  \item \textbf{Relazioni con altri componenti:} è il controller dedicato di  \code{Premi.PresentationManager.RemoveView} e sfrutta le funzionalità di \code{Premi.Presentation.Presentation} per l'eliminazione della presentazione 
\end{itemize}
\subsubsection{Premi.PresentationManager.PublicView}
\begin{itemize}
  \item \textbf{Nome:} PublicView
  \item \textbf{Tipo:} class
  \item \textbf{Package:} Premi.PresentationManager
  \item \textbf{Descrizione:} mostra una finestra di conferma per rendere pubblica o privata una presentazione
\end{itemize}
\subsubsection{Premi.PresentationManager.PublicController}
\begin{itemize}
  \item \textbf{Nome:} PublicController
  \item \textbf{Tipo:} class
  \item \textbf{Package:} Premi.PresentationManager
  \item \textbf{Descrizione:} fornisce alla view gli strumenti per rendere pubblica o privata una presentazione
  \item \textbf{Relazioni con altri componenti:} è il controller dedicato di  \code{Premi.PresentationManager.PublicView} e sfrutta le funzionalità di \code{Premi.Presentation.Presentation} per accedere alla base di dati
\end{itemize}
\subsubsection{Premi.PresentationManager.EditTitleDescrView}
\begin{itemize}
  \item \textbf{Nome:} EditTitleDescrView
  \item \textbf{Tipo:} class
  \item \textbf{Package:} Premi.PresentationManager
  \item \textbf{Descrizione:} mostra una finestra per la modifica del titolo e della descrizone della presentazione
\end{itemize}
\subsubsection{Premi.PresentationManager.EditTitleDescrController}
\begin{itemize}
  \item \textbf{Nome:} EditTitleDescrController
  \item \textbf{Tipo:} class
  \item \textbf{Package:} Premi.PresentationManager
  \item \textbf{Descrizione:} fornisce alla view la possibilità di modificare titolo e descrizione della presentazione
  \item \textbf{Relazioni con altri componenti:} è il controller dedicato di  \code{Premi.PresentationManager.EditTitleDescrView} e sfrutta le funzionalità di \code{Premi.Presentation.Presentation} per accedere alla base di dati
\end{itemize}
\subsubsection{Premi.PresentationManager.NewView}
\begin{itemize}
  \item \textbf{Nome:} NewView
  \item \textbf{Tipo:} class
  \item \textbf{Package:} Premi.PresentationManager
  \item \textbf{Descrizione:} mostra una finestra per la creazione di una nuova presentazione
\end{itemize}
\subsubsection{Premi.PresentationManager.NewController}
\begin{itemize}
  \item \textbf{Nome:} NewController
  \item \textbf{Tipo:} class
  \item \textbf{Package:} Premi.PresentationManager
  \item \textbf{Descrizione:} fornisce alla view gli strumenti per creare una nuova presentazione
  \item \textbf{Relazioni con altri componenti:} è il controller dedicato di  \code{Premi.PresentationManager.NewView} e sfrutta le funzionalità di \code{Premi.Presentation.Presentation} per accedere alla base di dati
\end{itemize}
\subsubsection{Premi.PresentationManager.ListPresView}
\begin{itemize}
  \item \textbf{Nome:} ListPresView
  \item \textbf{Tipo:} class
  \item \textbf{Package:} Premi.PresentationManager
  \item \textbf{Descrizione:} mostra all'utente la lista delle sue presentazioni, e bottoni aggiuntivi per accedere alle altre funzionalità del package
\end{itemize}
\subsubsection{Premi.PresentationManager.ListPresController}
\begin{itemize}
  \item \textbf{Nome:} ListPresController
  \item \textbf{Tipo:} class
  \item \textbf{Package:} Premi.PresentationManager
  \item \textbf{Descrizione:} fornisce alla view una lista preformata delle presentazioni dell'utente
  \item \textbf{Relazioni con altri componenti:} è il controller dedicato di  \code{Premi.PresentationManager.ListPresView} e sfrutta le funzionalità di \code{Premi.Presentation.Presentation} per recuperare la lista delle presentazioni
\end{itemize}
\subsection{Premi.Editor}
\subsubsection{Premi.Editor.View}
\begin{itemize}
  \item \textbf{Nome:} View
  \item \textbf{Tipo:} class
  \item \textbf{Package:} Premi.Editor
  \item \textbf{Descrizione:} view generica dell'editor che funge da contenitore
\end{itemize}
\subsubsection{Premi.Editor.EditorCtrl}
\begin{itemize}
  \item \textbf{Nome:} EditorCtrl
  \item \textbf{Tipo:} class
  \item \textbf{Package:} Premi.Editor
  \item \textbf{Descrizione:} fornisce alla view le funzionzalità per la modifica di una presentazione
  \item \textbf{Relazioni con altri componenti:} è il controller dedicato di  \code{Premi.Editor.EditorView} e sfrutta \code{Premi.Presentation.Presentation} per recuperare la presentazione dalla base di dati
\end{itemize}
\subsubsection{Premi.Editor.TextView}
\begin{itemize}
  \item \textbf{Nome:} TextView
  \item \textbf{Tipo:} class
  \item \textbf{Package:} Premi.Editor
  \item \textbf{Descrizione:} mostra le opzioni per l'aggiunta di un'area di testo
\end{itemize}
\subsubsection{Premi.Editor.TextController}
\begin{itemize}
  \item \textbf{Nome:} TextController
  \item \textbf{Tipo:} class
  \item \textbf{Package:} Premi.Editor
  \item \textbf{Descrizione:} fornisce alla view le funzionalità per creare e modificare un'area di testo
  \item \textbf{Relazioni con altri componenti:} è il controller dedicato di  \code{Premi.Editor.TextView} e utilizza la classe \code{Premi.Presentation.Text} per rappresentare le aree di testo
\end{itemize}
\subsubsection{Premi.Editor.ShapeView}
\begin{itemize}
  \item \textbf{Nome:} ShapeView
  \item \textbf{Tipo:} class
  \item \textbf{Package:} Premi.Editor
  \item \textbf{Descrizione:} mostra le opzioni per l'aggiunta di uno shape
\end{itemize}
\subsubsection{Premi.Editor.ShapeController}
\begin{itemize}
  \item \textbf{Nome:} ShapeController
  \item \textbf{Tipo:} class
  \item \textbf{Package:} Premi.Editor
  \item \textbf{Descrizione:} fornisce alla view le funzionalità per creare e modificare uno shape
  \item \textbf{Relazioni con altri componenti:} è il controller dedicato di  \code{Premi.Editor.ShapeView}  e utilizza la classe \code{Premi.Presentation.Shape} per rappresentare le Shape
\end{itemize}
\subsubsection{Premi.Editor.ImageView}
\begin{itemize}
  \item \textbf{Nome:} ImageView
  \item \textbf{Tipo:} class
  \item \textbf{Package:} Premi.Editor
  \item \textbf{Descrizione:} mostra le opzioni per l'aggiunta di un'immagine
\end{itemize}
\subsubsection{Premi.Editor.ImageController}
\begin{itemize}
  \item \textbf{Nome:} ImageController
  \item \textbf{Tipo:} class
  \item \textbf{Package:} Premi.Editor
  \item \textbf{Descrizione:} fornisce alla view le funzionalità per aggiungere un'immagine
  \item \textbf{Relazioni con altri componenti:} è il controller dedicato di  \code{Premi.Editor.ImageView}  e utilizza la classe \code{Premi.Presentation.Image} per rappresentare le immagini
\end{itemize}
\subsubsection{Premi.Editor.StyleView}
\begin{itemize}
  \item \textbf{Nome:} StyleView
  \item \textbf{Tipo:} class
  \item \textbf{Package:} Premi.Editor
  \item \textbf{Descrizione:} permette la modifica dello stile di un Frame oppure dello sfondo dell'infografica
\end{itemize}
\subsubsection{Premi.Editor.StyleController}
\begin{itemize}
  \item \textbf{Nome:} StyleController
  \item \textbf{Tipo:} class
  \item \textbf{Package:} Premi.Editor
  \item \textbf{Descrizione:} fornisce alla view le funzionalità per la modifica dello stile
  \item \textbf{Relazioni con altri componenti:} è il controller dedicato di  \code{Premi.Editor.StyleView},  e utilizza la classe \code{Premi.Presentation.Frame} per rappresentare lo stile dei Frame e dell'infografica
\end{itemize}
\subsubsection{Premi.Editor.FrameListView}
\begin{itemize}
  \item \textbf{Nome:} FrameListView
  \item \textbf{Tipo:} class
  \item \textbf{Package:} Premi.Editor
  \item \textbf{Descrizione:} view di base per la rappresentazione di una lista dei Frame contenuti all'interno della presentazione
\end{itemize}



\subsection{Premi.Editor.FrameEditor}
\subsubsection{Premi.Editor.FrameEditor.View}
\begin{itemize}
  \item \textbf{Nome:} View
  \item \textbf{Tipo:} class
  \item \textbf{Package:} Premi.Editor.FrameEditor.View
  \item \textbf{Descrizione:} è la view generale della fase di modifica dei Frame
\end{itemize}
\subsubsection{Premi.Editor.FrameEditor.FrameEditorCtrl}
\begin{itemize}
  \item \textbf{Nome:} FrameEditorCtrl
  \item \textbf{Tipo:} class
  \item \textbf{Package:} Premi.Editor.FrameEditor.FrameEditorCtrl
  \item \textbf{Descrizione:} fornisce alla view le funzionalità per la gestione delle altre view contenute al suo interno
  \item \textbf{Relazioni con altri componenti:} è il controller dedicato di  \code{Premi.Editor.FrameEditor.View} 
\end{itemize}
\subsubsection{Premi.Editor.FrameEditor.FrameListController}
\begin{itemize}
  \item \textbf{Nome:} FrameListController
  \item \textbf{Tipo:} class
  \item \textbf{Package:} Premi.Editor.FrameEditor
  \item \textbf{Descrizione:} fornisce alla view \code{FrameListView} le funzionalità per rappresentazione della lista dei Frame
  \item \textbf{Relazioni con altri componenti:} è uno dei tre controller dedicati di  \code{Premi.Editor.FrameListView}  e utilizza la classe \code{Premi.Presentation.Frame} per rappresentare un'anteprima dei Frame
\end{itemize}

\subsection{Premi.Editor.InfographicEditor}
\subsubsection{Premi.Editor.InfographicEditor.View}
\begin{itemize}
  \item \textbf{Nome:} View
  \item \textbf{Tipo:} class
  \item \textbf{Package:} Premi.Editor.InfographicEditor
  \item \textbf{Descrizione:}  è la view generale della fase di creazione o modifica dell'infografica
\end{itemize}
\subsubsection{Premi.Editor.InfographicEditor.InfographicEditorCtrl}
\begin{itemize}
  \item \textbf{Nome:} InfographicEditorCtrl
  \item \textbf{Tipo:} class
  \item \textbf{Package:} Premi.Editor.InfographicEditor
  \item \textbf{Descrizione:} fornisce alla view le funzionalità per la gestione delle altre view contenute al suo interno, che permetteranno all'utente di produrre un'infografica attraverso l'utilizzo di oggetti grafici, tra cui i Frame creati nella fase precedente
  \item \textbf{Relazioni con altri componenti:} è il controller dedicato di  \code{Premi.Editor.InfoGraphicEditor.View} 
\end{itemize}
\subsubsection{Premi.Editor.InfographicEditor.FrameListController}
\begin{itemize}
  \item \textbf{Nome:} FrameListController
  \item \textbf{Tipo:} class
  \item \textbf{Package:} Premi.Editor.InfographicEditor
  \item \textbf{Descrizione:} fornisce alla view \code{FrameListView} le funzionalità per rappresentazione della lista dei Frame
  \item \textbf{Relazioni con altri componenti:} è uno dei tre controller dedicati di  \code{Premi.Editor.FrameListView}  e utilizza la classe \code{Premi.Presentation.Frame} per rappresentare un'anteprima dei Frame
\end{itemize}

\subsection{Premi.Editor.TrailsEditor}
\subsubsection{Premi.Editor.TrailsEditor.View}
\begin{itemize}
  \item \textbf{Nome:} View
  \item \textbf{Tipo:} class
  \item \textbf{Package:} Premi.Editor.TrailsEditor
  \item \textbf{Descrizione:} view generale della della fase di creazione o modifica dei percorsi di presentazione
\end{itemize}
\subsubsection{Premi.Editor.TrailsEditor.TrailsEditorCtrl}
\begin{itemize}
  \item \textbf{Nome:} TrailsEditorCtrl
  \item \textbf{Tipo:} class
  \item \textbf{Package:} Premi.Editor.TrailsEditor
  \item \textbf{Descrizione:} fornisce alla view le funzionalità per la gestione delle altre view contenute al suo interno, che permetteranno di gestire i percorsi di presentazione
  \item \textbf{Relazioni con altri componenti:}  è il controller dedicato di  \code{Premi.Editor.TrailsEditor.View} , si collega a \code{Premi.Presentation.Presentation} per recuperare oggetti di tipo \code{Premi.Presentation.Trail}
\end{itemize}
\subsubsection{Premi.Editor.TrailsEditor.FrameListController}
\begin{itemize}
  \item \textbf{Nome:} FrameListController
  \item \textbf{Tipo:} class
  \item \textbf{Package:} Premi.Editor.TrailsEditor
  \item \textbf{Descrizione:} fornisce alla view \code{FrameListView} le funzionalità per rappresentazione della lista dei Frame
  \item \textbf{Relazioni con altri componenti:} è uno dei tre controller dedicati di  \code{Premi.Editor.FrameListView}  e utilizza la classe \code{Premi.Presentation.Frame} per rappresentare un'anteprima dei Frame
\end{itemize}
\subsubsection{Premi.Editor.TrailsEditor.EditTrailView}
\begin{itemize}
  \item \textbf{Nome:} EditTrailView
  \item \textbf{Tipo:} class
  \item \textbf{Package:} Premi.Editor.TrailsEditor
  \item \textbf{Descrizione:} view dedicata alla modifica di un percorso di presentazione
\end{itemize}
\subsubsection{Premi.Editor.TrailsEditor.EditTrailController}
\begin{itemize}
  \item \textbf{Nome:} EditTrailController
  \item \textbf{Tipo:} class
  \item \textbf{Package:} Premi.Editor.TrailsEditor
  \item \textbf{Descrizione:} fornisce alla view le funzionalità per la gestione delle altre view contenute al suo interno, che permetteranno all'utente di produrre un percorso di presentazione con i Frame da lui creati
  \item \textbf{Relazioni con altri componenti:} è il controller dedicato di  \code{Premi.Editor.TrailsEditor.EditTrailView} 
\end{itemize}
\subsubsection{Premi.Editor.TrailsEditor.NewView}
\begin{itemize}
  \item \textbf{Nome:} NewView
  \item \textbf{Tipo:} class
  \item \textbf{Package:} Premi.Editor.TrailsEditor
  \item \textbf{Descrizione:} mostra la finestra di creazione di un percorso di presentazione
\end{itemize}
\subsubsection{Premi.Editor.TrailsEditor.NewController}
\begin{itemize}
  \item \textbf{Nome:} NewController
  \item \textbf{Tipo:} class
  \item \textbf{Package:} Premi.Editor.TrailsEditor
  \item \textbf{Descrizione:} fornisce alla view le funzionalità per la creazione di un nuovo percorso di presentazione
  \item \textbf{Relazioni con altri componenti:} è il controller dedicato di  \code{Premi.Editor.TrailsEditor.NewView} 
\end{itemize}
\subsubsection{Premi.Editor.TrailsEditor.EditTitleView}
\begin{itemize}
  \item \textbf{Nome:} EditTitleView
  \item \textbf{Tipo:} class
  \item \textbf{Package:} Premi.Editor.TrailsEditor
  \item \textbf{Descrizione:} mostra la finestra di modifica del titolo di un percorso di presentazione
\end{itemize}
\subsubsection{Premi.Editor.TrailsEditor.EditTitleController}
\begin{itemize}
  \item \textbf{Nome:} EditTileController
  \item \textbf{Tipo:} class
  \item \textbf{Package:} Premi.Editor.TrailsEditor
  \item \textbf{Descrizione:} fornisce alla view le funzionalità per la modifica del titolo di un percorso di presentazione
  \item \textbf{Relazioni con altri componenti:} è il controller dedicato di  \code{Premi.Editor.TrailsEditor.EditTitleView} 
\end{itemize}
\subsubsection{Premi.Editor.TrailsEditor.RemoveView}
\begin{itemize}
  \item \textbf{Nome:} RemoveView
  \item \textbf{Tipo:} class
  \item \textbf{Package:} Premi.Editor.TrailsEditor
  \item \textbf{Descrizione:} mostra la finestra di conferma rimozione di un percorso di presentazione
\end{itemize}
\subsubsection{Premi.Editor.TrailsEditor.RemoveController}
\begin{itemize}
  \item \textbf{Nome:} RemoveController
  \item \textbf{Tipo:} class
  \item \textbf{Package:} Premi.Editor.TrailsEditor
  \item \textbf{Descrizione:} fornisce alla view le funzionalità per la rimozione di un percorso di presentazione
  \item \textbf{Relazioni con altri componenti:} è il controller dedicato di  \code{Premi.Editor.TrailsEditor.RemoveView} 
\end{itemize}












