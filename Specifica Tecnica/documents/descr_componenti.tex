\section{Descrizione dei singoli componenti}
	Tipo, obiettivo e funzione del componente \\
	Relazioni d'uso di altre componenti \\
	Interfacce con le relazioni di uso da altre componenti \\
\subsection{Premi}
-----------------------------premi-class.jpg
\subsubsection{Premi.Utility}
\begin{itemize}
  \item[] \textbf{Nome:} Utility
  \item[] \textbf{Tipo:} class
  \item[] \textbf{Package:} Premi
  \item[] \textbf{Descrizione:} classe statica di utilità che fornisce strumenti per interagire con la base di dati
\end{itemize}

\subsubsection{Premi.View}
\begin{itemize}
  \item \textbf{Nome:} View
  \item \textbf{Tipo:} class
  \item \textbf{Package:} Premi
  \item \textbf{Descrizione:} view generale che funge da sfondo per le altre view
\end{itemize}

\subsubsection{Premi.PremiCtrl}
\begin{itemize}
  \item \textbf{Nome:} PremiCtrl
  \item \textbf{Tipo:} class
  \item \textbf{Package:} Premi
  \item \textbf{Descrizione:} controller generale che modifica la view principale in base alle scelte dell'utente
  \item \textbf{Relazioni con altri componenti:} la classe altera l'aspetto di \code{Premi.View} caricando le view di cui l'utente ha bisogno 
\end{itemize}

\subsection{Premi.UserManager}
-----------------------------premi-class.jpg
\subsubsection{Premi.UserManager.User}
-----------------------------UserManager-class.jpg (mostra le relazioni che possiede ESTERNE al package)
\begin{itemize}
  \item \textbf{Nome:} User
  \item \textbf{Tipo:} class
  \item \textbf{Package:} Premi.UserManger
  \item \textbf{Descrizione:} classe che definisce un utente e fornisce le funzionalità necessarie alla sua creazione, al suo login e ad un eventuale cambio di password
  \item \textbf{Relazioni con altri componenti:} si collega a \code{Premi.Utility} per le interazioni con la base di dati
\end{itemize}

\subsubsection{Premi.UserManager.View}
\begin{itemize}
  \item \textbf{Nome:} View
  \item \textbf{Tipo:} class
  \item \textbf{Package:} Premi.UserManger
  \item \textbf{Descrizione:} vista generale del package \code{UserManager}. Puo' contenere le view abbinate alle funzionalità di \code{User}
\end{itemize}
\subsubsection{Premi.UserManager.UserManagerCtrl}
\begin{itemize}
  \item \textbf{Nome:} UserManagerCtrl
  \item \textbf{Tipo:} class
  \item \textbf{Package:} Premi.UserManager
  \item \textbf{Descrizione:} controller generale del package \code{UserManager} dedicato a fornire alla view strumenti per l'interazione con l'utente
  \item \textbf{Relazioni con altri componenti:} si collega a \code{Premi.UserManager.View} per mostrare i dati dell'utente che va a prelevare tramite {Premi.UserManager.User}
\end{itemize}
\subsubsection{Premi.UserManager}
\begin{itemize}
  \item \textbf{Nome:} 
  \item \textbf{Tipo:} 
  \item \textbf{Package:} 
  \item \textbf{Descrizione:} 
  \item \textbf{Relazioni con altri componenti:} 
\end{itemize}
\subsubsection{Premi.UserManager}
\begin{itemize}
  \item \textbf{Nome:} 
  \item \textbf{Tipo:} 
  \item \textbf{Package:} 
  \item \textbf{Descrizione:} 
  \item \textbf{Relazioni con altri componenti:} 
\end{itemize}







