\newpage
\section{Definizione del Prodotto}

\subsection{Metodo e formalismo di specifica}
Verrà qui esposta l'architettura di Premi ad alto livello seguendo un approccio top-down$_G$: verranno prima descritti i package$_G$ e le loro dipendenze e successivamente le singole classi contenute al loro interno. I diagrammi delle classi e dei package$_G$ seguono il formalismo UML$_G$2.0 e la struttura dei package segue una prassi ("best practice$_G$") di AngularJS$_G$ che propone una suddivisione dei componenti per funzionalità dell'applicazione in alternativa alla classica suddivisione Model-View-Controller$_G$, più difficile da mantenere per applicazioni di medie o grandi dimensioni. Per ulteriori approfondimenti consultare la guida al sito \href{https://scotch.io/tutorials/angularjs-best-practices-directory-structure}{scotch.io} oppure il tutorial di \href{http://angular-meteor.com/tutorial/step_07}{urigo:angular-meteor}.
Si illustreranno poi i Design Pattern utilizzati nella fase di progettazione ad alto livello e si descriveranno le interazioni dell'utente con l'applicazione attraverso i diagrammi di attività$_G$.


\subsection{Presentazione dell'architettura generale del sistema}
I componenti sono stati suddivisi prima in base al loro contributo a specifiche funzionalità del software e solo successivamente per appartenenza ai ruoli del pattern MVC$_G$. Questo aumenta la chiarezza espositiva dei diagrammi, evita la creazione di package$_G$ contenenti un numero eccessivo di classi e aiuta a compiere verifiche mirate a singoli componenti. \\
È importante specificare che il framework AngularJS unisce view e controller attraverso una dichiarazione esterna a entrambi, che fa parte del meccanismo detto di \textit{routing} o di reindirizzamento dell'utente; view e controller inoltre non sanno di essere collegati tra loro e comunicano attraverso un oggetto chiamato \textit{\$scope}. Questo rende l'architettura sia di tipo Model-View-Controller$_G$ che di tipo Model-View-ViewModel$_G$. \\
Per motivi di leggibilità \$scope e routing non verranno rappresentati in modo esplicito nei diagrammi dei package e delle classi di questo documento, ma sono comunque da considerarsi impliciti nelle dipendenze tra i view e controller dei componenti.


\section{Diagrammi dei Package}
Di seguito vengono descritti componenti principali del sistema e le loro dipendenze.

-----------------------------pkg.jpg

L'applicazione è costituita da un solo package$_G$ principale chiamato \code{Premi}; al suo interno sono presenti:
\begin{itemize}
\item \code{Premi} racchiude tutti i package$_G$ e la view e il controller principali;
\item \code{Premi.UserManager} è il package$_G$ di gestione dei dati dell'utente;
\item \code{Premi.Viewer} racchiude gli elementi necessari alla visualizzazione della presentazione nei vari contesti previsti;
\item \code{Premi.Presentation} racchiude la struttura generale della presentazione;
\item \code{Premi.PresentationManager} contiene gli elementi necessari alla gestione delle presentazioni da parte dell'utente;
\item \code{Premi.Editor} è il package$_G$ dedicato alla modifica interna delle presentazioni; possiede al suo interno tre ulteriori package$_G$:
\begin{itemize}
\item \code{Premi.Editor.FrameEditor} si occupa di creare, modificare o cancellare i Frame$_G$ contenuti nella presentazione;
\item \code{Premi.Editor.InfographicEditor} posiziona Frame$_G$ o altri elementi all'interno di un poster;
\item \code{Premi.Editor.TrailsEditor} ordina i Frame per la creazione di uno o più percorsi di presentazione.
\end{itemize}
\end{itemize}
