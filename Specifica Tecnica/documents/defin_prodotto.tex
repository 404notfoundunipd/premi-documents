\newpage
\section{Definizione del Prodotto}

\subsection{Metodo e formalismo di specifica}
Verrà qui esposta l'architettura di Premi ad alto livello seguendo un approccio top-down$_G$: verranno prima descritti i package$_G$ e le loro dipendenze e successivamente le singole classi contenute al loro interno. I diagrammi delle classi e dei package$_G$ seguono il formalismo UML$_G$2.0 e la struttura dei package segue una prassi ("best practice$_G$") di AngularJS$_G$ che propone una suddivisione dei componenti per funzionalità dell'applicazione in alternativa alla classica suddivisione Model-View-Controller$_G$, più difficile da mantenere per applicazioni di medie o grandi dimensioni. Per ulteriori approfondimenti consultare la guida al sito \href{https://scotch.io/tutorials/angularjs-best-practices-directory-structure}{scotch.io} oppure il tutorial di \href{http://angular-meteor.com/tutorial/step_07}{urigo:angular-meteor}.
Si illustreranno poi i Design Pattern utilizzati nella fase di progettazione ad alto livello e si descriveranno le interazioni dell'utente con l'applicazione attraverso i diagrammi di attività$_G$.


\subsection{Presentazione dell'architettura generale del sistema}

Indentificazione dei componenti\\
(diagrammi dei componenti e delle classi ordinati per MVC)
- architetturali di alto livello
